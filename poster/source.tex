\RequirePackage{amsmath}
\documentclass[25pt, a0paper, landscape, margin=15.4mm]{tikzposter}
\def\textmu{}

%include agda.fmt

\usepackage{natbib, textgreek, bussproofs, epigraph, float,
            enumerate, url, graphicx, listings, xfrac}
\usepackage{amsmath}
\usepackage{amssymb}
\usepackage{filecontents}
\usepackage{color}
\usepackage{fancyvrb}
\usepackage{inconsolata}
\usepackage{tgpagella}
\usepackage{tikz-cd}
\usepackage{adjustbox}
\usepackage{enumitem}
% \sloppy

% \usepackage[usenames,dvipsnames]{xcolor}

\definecolor{IdrisRed}{RGB}{166, 0, 2}
\definecolor{IdrisBlue}{RGB}{0, 0, 182}
\definecolor{IdrisGreen}{RGB}{13, 84, 8}
\definecolor{IdrisPurple}{RGB}{126, 2, 236}

\newcommand{\IdrisData}[1]{\textcolor{IdrisRed}{#1}}
\newcommand{\IdrisType}[1]{\textcolor{IdrisBlue}{#1}}
\newcommand{\IdrisBound}[1]{\textcolor{IdrisPurple}{#1}}
\newcommand{\IdrisFunction}[1]{\textcolor{IdrisGreen}{#1}}
\newcommand{\IdrisMetavar}[1]{\textcolor{cyan}{#1}}
\newcommand{\IdrisKeyword}[1]{{\textbf{#1}}}
\newcommand{\IdrisImplicit}[1]{{\itshape \IdrisBound{#1}}}
\usepackage{setspace, microtype}
\fvset{commandchars=\\\{\},formatcom=\singlespacing, frame=single}
\newcommand{\ty}[1]{\IdrisType{\texttt{#1}}}
\newcommand{\kw}[1]{\IdrisKeyword{\texttt{#1}}}
\newcommand{\fn}[1]{\IdrisFunction{\texttt{#1}}}
\newcommand{\dt}[1]{\IdrisData{\texttt{#1}}}
\newcommand{\bn}[1]{\IdrisBound{\texttt{#1}}}
\newcommand{\cm}[1]{\textcolor{darkgray}{\texttt{#1}}}
\newcommand{\hole}[1]{\IdrisMetavar{\texttt{?}\IdrisMetavar{\texttt{#1}}}}
\newcommand{\Elab}{\ty{Elab}}
\newcommand{\String}{\ty{String}}
\newcommand{\TT}{\ty{TT}}
\newcommand{\Raw}{\ty{Raw}}
\newcommand{\Editorable}{\ty{Editorable}}
\newcommand{\TyDecl}{\ty{TyDecl}}
\newcommand{\FunDefn}{\ty{FunDefn}}
\newcommand{\FunClause}{\ty{FunClause}}


  % Math and code commands {{{
  \newcommand{\nop}[0]{} % used to reconcile vim folds and latex curly braces
  \renewcommand{\paragraph}[1]{\bigskip\textbf{#1}}
  \newcommand{\figrule}{\begin{center}\hrule\end{center}}
  \newcommand{\set}[1]{\left\{#1\right\}}
  \DeclareRobustCommand{\shamrock}{\raisebox{-.035em}{\includegraphics[width=.75em, height=.75em]{symbols/command}}\nop}
  \DeclareRobustCommand{\col}{\raisebox{-.035em}{\includegraphics[width=.30em, height=.75em]{symbols/colon}}\nop}
  \newcommand{\conc}[2]{#1 \text{<#2>}}
  \newcommand{\val}[2]{#1 \sym #2}
  % }}}

\newcommand{\T}[1]{\texttt{#1}}

\usepackage{more}

\title{Edit-Time Tactics in Idris}
\institute{Wesleyan University} % See Section 4.1
\author{Joomy Korkut}

\usetheme{Simple}  % See Section 5

\begin{document}
% \definecolor{myblue}{RGB}{89,65,242}
\definecolor{myblue}{RGB}{0, 128, 110}
\colorlet{titlebgcolor}{myblue}
\colorlet{innerblocktitlefgcolor}{myblue}
\colorlet{blocktitlefgcolor}{myblue}

\maketitle  % See Section 4.1

\begin{columns}  % See Section 4.4

\column{0.5}

  \block{Abstract\phantom{p}}{
  % \coloredbox[bgcolor=blue,fgcolor=white]{}

  \Large
  \setlength{\parindent}{1.5cm}
  \indent Metaprogramming is a technique that allows users to write programs that
  write programs. In dependently-typed languages such as \textbf{Idris}, recent work on
  \textbf{elaborator reflection}\cite{elabref} paved the way for new applications of
  metaprogramming by showing that it can be a substitute for \textbf{tactic-based proof
  languages}.  The goal of our work is to use elaborator reflection to write
  editor interaction actions in Idris.
  \medskip

  \indent Currently in Idris modes of editors such as Emacs, users can perform actions
  like type-checking holes, case-splitting, and lemma extraction.
  Implementations of all of these Idris editor actions are hard-coded in the
  compiler and they are written in Haskell. Our work will allow us to \textbf{rewrite
  them in Idris as metaprograms}, and to move them into an Idris library instead
  of having them embedded into the compiler.
  \medskip

  \indent  Furthermore, we can write our own tactics through elaborator
  reflection, and run them from the editor, i.e. in \textbf{edit-time}.
  This would extend the abilities of the editor interaction mode from the
  current built-in features to anything that can be done with tactics.
  In our work, we present the design and implementation of this feature in the
  Idris compiler.
  \medskip

  \indent{}  We also implement an intuitionistic theorem prover tactic that is meant to be
  an better alternative to the built-in proof search editor action, and an
  add-clause tactic that exemplifies how we can move some of the hard-coded
  features from the compiler to a library.
}


\column{0.50}
  \block
  {Crash course on the Idris compiler and Idris metaprogramming}{
  \Large
  \scalebox{1.5}{
\newcommand\mlnode[1]{
  \begin{minipage}{8cm}
    \linespread{1}\selectfont \begin{center} \small  #1 \end{center}
  \end{minipage}}
\begin{tikzcd}[ampersand replacement=\&]
  \mlnode{
    Haskell terms\\ \medskip \footnotesize such as the \ty{()} term\\
    \dt{()}
  }
  \arrow[rrr, "\text{\small reflection}" description, bend right=10]
  \&  \&  \&
  \mlnode{
    Haskell representation of Idris core language terms\\ \medskip \scriptsize
    such as the \ty{Term} term\\
    \texttt{\dt{P} (\dt{DCon} \dt{0} \dt{0} \dt{False}) (\dt{UN} \dt{"MkUnit"}) (\dt{P} (\dt{TCon} \dt{0} \dt{0 \dt{False}}) (\dt{UN} \dt{"Unit"}) \dt{Erased})}
  }
  \arrow[lll, "\text{\small reification}" description, bend right=10] \\
  \&  \&  \&\\
  \mlnode{
    Idris terms\\ \medskip \footnotesize such as the \ty{()} term\\
    \dt{()}
  }
  \arrow[r, "\text{\small elaboration}" description, bend right=50]
  \&
  \mlnode{
    Idris core terms\\ \medskip \footnotesize such as the \ty{Unit} term\\
    \dt{MkUnit}
  }
  \arrow[l, "\text{\small delaboration}" description, bend right=50]
  \arrow[rr, "\text{\small quotation}" description, bend right=35]
  \arrow[uurr, "\text{\small \emph{internally represented as}}" description, dashed, bend left=17]
  \& \&
  \mlnode{
    Idris representation of the Haskell representation of Idris core language terms\\
    \medskip \scriptsize such as the \TT\ term\\
    \texttt{\dt{P} (\dt{DCon} \dt{0} \dt{0}) (\dt{UN} \dt{"MkUnit"})\\
    (\dt{P} (\dt{TCon} \dt{0} \dt{0}) (\dt{UN} \dt{"Unit"}) \dt{Erased})}
  }
  \arrow[ll, "\text{\small unquotation}" description, bend right=35]
  \arrow[uu, leftrightarrow, "\text{\small \emph{correspondence}}" description, dashed]
\end{tikzcd}
}
}
\end{columns}

\begin{columns}
\column{0.25}
\block{The \ty{Editorable} interface\phantom{p}}{
  \Large
  \setlength{\parindent}{1.5cm}
  \indent Editors communicate with the compiler via \mbox{S-expressions}, so
  users should be able dictate how to convey a value of a given type.  We
  define an interface in Idris, which can define serialization and
  deserialization of the terms of a type into an \ty{SExp} and vice versa.

  % \vspace{-0.2cm}
  \hspace{-1.5cm}
  \scalebox{3.5}{
    \begin{minipage}{6cm}
      \VerbatimInput[frame=none]{Editorable.tex}
    \end{minipage}
  }
  \vspace{1.5cm}

  \indent Usually we can define \ty{Editorable} implementations in Idris
  itself. But for Idris types that represent the Haskell representation of
  Idris core terms, such as \ty{TT} and \ty{TyDecl}, we need to define
  \ty{Editorable} implementations primitively in order to be able to parse,
  elaborate, delaborate, and pretty print them.
}

\column{0.25}
\block{Type-checking with \ty{Editorable}\phantom{p}}{

}

\column{0.25}
\block{Toy example}{
\Large
  \setlength{\parindent}{1.5cm}
  \vspace{-1.7cm}
  \hspace{-3cm}
  \scalebox{3.5}{
    \begin{minipage}{6cm}
      \VerbatimInput[frame=none]{Toy.tex}
    \end{minipage}
  }
  \vspace{1.5cm}

  \indent We can run \fn{toy} from the editor with a hole name and
  some string representing a \TT\ term. If the type of the hole is \ty{Nat},
  then \fn{toy} returns the term it was given, otherwise fails.

  \indent Idris code cannot tell Emacs to replace the name under cursor with
  some expression; we have to write 5-10 lines of trivial Emacs Lisp code
  to glue the \Elab\ action to our editor.
  \mbox{\emph{(If we are using Emacs, of course.)}}
}

\column{0.25}
\block{Useful applications}{
\Large
  \vspace{-0.7cm}
  \begin{enumerate}[labelsep=0.5cm]
    \renewcommand{\labelenumi}{(\arabic{enumi})}
    \item Replace the existing built-in editor actions (written in Haskell as a
      part of the compiler) with edit-time tactics (written in Idris). In our
      work, we present a replacement for the ``add clause'' action.
    \item Write a proof search mechanism better than the built-in one, as an
      edit-time tactic. In our work, we present \textbf{Hezarfen}, a theorem
      prover tactic that decides intuitionistic propositional logic, generates
      \TT\ proof terms, and simplifies the proof terms as much as possible.
\end{enumerate}
}

\block[titleoffsety=3cm, bodyoffsety=3cm]
  {References}{
  \vspace{-0.5cm}
  \bibliographystyle{plain}
  \renewcommand{\section}[2]{}%
  \bibliography{paper}
}
\end{columns}


\end{document}

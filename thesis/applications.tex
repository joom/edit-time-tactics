\section{Applications} \label{sec:applications}

\subsection{Theorem Prover for Intuitionistic Propositional Logic}\label{ssec:hezarfen}

In this section, we describe the tactic Hezarfen\footnote{ The name is
  pronounced {[\textipa{hezaRf\ae n}]}, like ``has are fan'', and it means
    polymath in Turkish.  Tactic source code is available at
    \url{http://github.com/joom/hezarfen}.}, which can decide intuitionistic
propositional logic theorems. This is based on LJT\cite{ljt} and its Haskell
implementation called Djinn\cite{djinn}, which generates Haskell expressions
for a given type. However, Djinn is a standalone program that takes commands
interactively, and when it generates an expression it prints it on the screen.
We can improve this by having our
theorem prover generate Idris expressions of the type \path{Raw}, one of the
types used for the inner representation of the core language of Idris.  We can
then write a tactic that takes the goal type, which will of the \path{Raw}, and
tries to fill the hole with the generated term.  Let's prove a theorem using
this tactic:

\begin{Verbatim}
\IdrisFunction{noContradiction} : \IdrisFunction{Not} \IdrisType{(}\IdrisImplicit{p}\IdrisType{,} \IdrisFunction{Not} \IdrisImplicit{p}\IdrisType{)}
\IdrisFunction{noContradiction} = \IdrisKeyword{\%runElab} \IdrisFunction{hezarfenExpr}
\end{Verbatim}

Through the elaborator reflection, that will turn into this\footnote{
  The definition is not changed at all. If it were, it could simply be written
  as

\ttfamily{\IdrisFunction{noContradiction} \IdrisData{(}\IdrisBound{pf}\IdrisData{,} \IdrisBound{contra}\IdrisData{)} = \IdrisBound{contra} \IdrisBound{p}} } when we compile the file:
\begin{Verbatim}[commandchars=\\\{\}]
\IdrisFunction{noContradiction} : \IdrisFunction{Not} \IdrisType{(}\IdrisImplicit{p}\IdrisType{,} \IdrisFunction{Not} \IdrisImplicit{p}\IdrisType{)}
\IdrisFunction{noContradiction} = \textbackslash\IdrisBound{h20} =>
     \IdrisKeyword{let} \IdrisBound{i20} = \IdrisFunction{fst} \IdrisBound{h20} \IdrisKeyword{in} \IdrisKeyword{let} \IdrisBound{j20} = \IdrisFunction{snd} \IdrisBound{h20} \IdrisKeyword{in} \IdrisFunction{void} (\IdrisBound{j20} \IdrisBound{i20})
\end{Verbatim}




% explain the nested impl rule in depth
% and possibly one or two more, to show what they look like in general


\subsection{Red-Black Trees}\label{ssec:rbt}

% let's translate the same problem to Idris.

% Suppose we want to balance a red-black tree, where
% the right subtree is well-formed, represented with the type \path{WellRedTree}, but
% the left one is well-formed except the root, represented with the type
% \path{AlmostWellRedTree}. We also have a key and value that should be stored in
% the root. Our task is to form a well-formed red-black tree with these.

\subsection{Nonlinear Patterns}

In Idris, there is no way to include the equality of two variables as a condition to match the pattern in the first place.
In Haskell, you can add a guard after the pattern that checks if those two
given variables are equal, assuming that their type has an \path{Eq} instance.
In OCaml, you can do the same with the \path{when} keyword.
However, these are not ideal. Prolog users will wonder whether we can just name
those variables the same thing, in order to check if they are equal. The answer
is no, yet in Idris, if we have an equality proof about two things, we can use
the same name for both since they are already the same.\footnote{In Agda the first one will have a normal name, but the other one will have the same name prepended with a dot. This is called a ``dot pattern''.} That means we would
first have to generate an equality proof for two things, which is something we
can do with views in Idris.

Using nonlinear patterns, this is how we would write a list membership check
function:

\begin{figure}[ht]
\caption{Code with nonlinear patterns, that we are trying to simulate}
\label{code:nonlinear}
\begin{Verbatim}
\IdrisFunction{elem} : \IdrisType{Eq} \IdrisImplicit{a} => \IdrisImplicit{a} -> \IdrisType{List} \IdrisImplicit{a} -> \IdrisType{Bool}
\IdrisFunction{elem} _ \IdrisData{[]} = \IdrisData{False}
\IdrisFunction{elem} \IdrisBound{x} (\IdrisBound{x} \IdrisData{::} \IdrisBound{ys}) = \IdrisData{True}
\IdrisFunction{elem} \IdrisBound{x} (\IdrisBound{y} \IdrisData{::} \IdrisBound{ys}) = \IdrisData{False}
\end{Verbatim}
\end{figure}

As we said before, this is not valid Idris. However, this would work:

\begin{figure}[ht]
\caption{The code with view, that simulates nonlinear patterns}
\label{code:nonlinearWith}
\begin{Verbatim}
\IdrisFunction{elem} : \IdrisType{DecEq} \IdrisImplicit{a} => \IdrisImplicit{a} -> \IdrisType{List} \IdrisImplicit{a} -> \IdrisType{Bool}
\IdrisFunction{elem} _ \IdrisData{[]} = \IdrisData{False}
\IdrisFunction{elem} \IdrisBound{x} (\IdrisBound{y} \IdrisData{::} \IdrisBound{ys}) \IdrisKeyword{with} (\IdrisFunction{decEq} \IdrisBound{x} \IdrisBound{y})
  \IdrisFunction{elem} \IdrisBound{x} (\IdrisBound{x} \IdrisData{::} \IdrisBound{ys}) | \IdrisData{Yes} \IdrisBound{Refl} = \IdrisData{True}
  \IdrisFunction{elem} \IdrisBound{x} (\IdrisBound{y} \IdrisData{::} \IdrisBound{ys}) | \IdrisData{No} \IdrisBound{contra} = \IdrisData{False}
\end{Verbatim}
\end{figure}

Even though this is a workaround and not really nonlinear patterns, we can
write an \Elab\ editor action that runs on a nonlinear pattern, such as
\autoref{code:nonlinear} and converts that into a function that uses Idris
views, such as \autoref{code:nonlinearWith}.

This can be generalized to things that are not like Prolog-style patterns.
Brady describes a view that matches on the last element of
a list, as presented in \autoref{code:listlast}.

\begin{figure}[ht]
\caption{Brady's view to match the last element of a list.\cite{tdd}}
\label{code:listlast}
\begin{Verbatim}
\IdrisKeyword{data} \IdrisType{ListLast} : \IdrisType{List} \IdrisImplicit{a} -> \IdrisType{Type} \IdrisKeyword{where}
  \IdrisData{Empty} : \IdrisType{ListLast} \IdrisData{[]}
  \IdrisData{NonEmpty} : (\IdrisBound{xs} : \IdrisType{List} \IdrisImplicit{a}) -> (\IdrisBound{x} : \IdrisImplicit{a}) -> \IdrisType{ListLast} (\IdrisBound{xs} \IdrisFunction{++} \IdrisData{[}\IdrisBound{x}\IdrisData{]})

\IdrisFunction{listLast} : (\IdrisBound{xs} : \IdrisType{List} \IdrisImplicit{a}) -> \IdrisType{ListLast} \IdrisBound{xs}
\IdrisFunction{listLast} \IdrisData{[]} = \IdrisData{Empty}
\IdrisFunction{listLast} (\IdrisBound{x} \IdrisData{::} \IdrisBound{xs}) = \IdrisKeyword{case} \IdrisFunction{listLast} \IdrisBound{xs} \IdrisKeyword{of}
                       \IdrisData{Empty} => \IdrisData{NonEmpty} \IdrisData{[]} \IdrisBound{x}
                       \IdrisData{NonEmpty} \IdrisBound{xs} \IdrisBound{y} => \IdrisData{NonEmpty} (\IdrisBound{x} \IdrisData{::} \IdrisBound{xs}) \IdrisBound{y}
\end{Verbatim}
\end{figure}

Now, suppose we want to write a function to match the last element of a list.
If we developed such an \Elab\ editor action, we could write the code in
\autoref{code:safeLast1} and get the result in \autoref{code:safeLast2}.

\begin{figure}[ht]
\caption{Matching the last element of a list, what we are trying to simulate}
\label{code:safeLast1}
\begin{Verbatim}
\IdrisFunction{safeLast} : \IdrisType{List} \IdrisImplicit{a} -> \IdrisType{Maybe} \IdrisImplicit{a}
\IdrisFunction{safeLast} \IdrisData{[]} = \IdrisData{Nothing}
\IdrisFunction{safeLast} (\IdrisBound{xs} \IdrisFunction{++} \IdrisData{[}\IdrisBound{x}\IdrisData{]}) = \IdrisData{Just} \IdrisBound{x}
\end{Verbatim}
\end{figure}

\begin{figure}[ht]
\caption{The code that matches the last element of a list with a view}
\label{code:safeLast2}
\begin{Verbatim}
\IdrisFunction{safeLast} : \IdrisType{List} \IdrisImplicit{a} -> \IdrisType{Maybe} \IdrisImplicit{a}
\IdrisFunction{safeLast} \IdrisBound{l} \IdrisKeyword{with} (\IdrisFunction{listLast} \IdrisBound{l})
  \IdrisFunction{safeLast} \IdrisData{[]} | \IdrisData{Empty} = \IdrisData{Nothing}
  \IdrisFunction{safeLast} (\IdrisBound{xs} \IdrisFunction{++} \IdrisData{[}\IdrisBound{x}\IdrisData{]}) | (\IdrisData{NonEmpty} \IdrisBound{xs} \IdrisBound{x}) = \IdrisData{Just} \IdrisBound{x}
\end{Verbatim}
\end{figure}

If we want to run the same tactic on all of these, we have to write an \Elab\ action
that can
\begin{enumerate}
\item Look at a pattern and detect nonlinearities
\item Look at a pattern and detect we used functions instead of constructors inside
\item Look up the type of the interesting patterns
\item Search for a view function that could be used to achieve this pattern
\item Try to apply possible view functions by unifying the input pattern with the view type constructors
\end{enumerate}

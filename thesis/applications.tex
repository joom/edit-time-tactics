\chapter{Applications} \label{chap:applications}

\section{A tactic to replace the built-in ``add clause'' action}\label{sec:addClause}

We have seen in \autoref{chap:introduction} how the ``Add initial match clause
to type declaration'' editor action works. When the cursor is on the type
signature of a function that does not have any clauses, we can run this editor
action and get an initial clause for the function.

In this section we implement this editor action for top-level type declarations
without implicit arguments or interface constraints, using edit-time tactics.
The Idris code we need to write is in \autoref{code:addClause}.

\begin{figure}[ht]
\caption{Implementation of the edit-time tactic for ``add clause''.}
\label{code:addClause}
\begin{Verbatim}[framesep=2mm, label=\footnotesize{\normalfont{Idris}}, labelposition=topline]
\fn{collectTypes} : \ty{TT} -> (\bn{xs} : \ty{List} \ty{TT} \dt{**} \ty{NonEmpty} \bn{xs})
\fn{collectTypes} (\dt{Bind} _ (\dt{Pi} \bn{ty} _) \bn{t}) =
  (\bn{ty} \fn{::} \bn{fst} (\fn{collectTypes} \bn{t}) ** \dt{IsNonEmpty})
\fn{collectTypes} \bn{t} = (\dt{[}\bn{t}\dt{]} ** \dt{IsNonEmpty})

\kw{%editor}
\fn{addClause} : \ty{TTName} -> \ty{Elab} (\ty{FunClause} \ty{TT})
\fn{addClause} \bn{n} =
  \kw{do} (_, _, \bn{ty}) <- \fn{lookupTyExact} \bn{n}
     \kw{let} (\bn{collected} \dt{**} \bn{nonEmpty}) = \fn{collectTypes} \bn{ty}
     \kw{let} (\bn{argTys}, \bn{retTy}) = (\fn{init} \bn{collected}, \fn{last} \bn{collected})
     \bn{argNames} <- \fn{traverse} (\fn{const} \fn{fresh}) \bn{argTys}
     \kw{let} \bn{lhsUntyped} = \fn{foldl} \dt{RApp} (\dt{Var} \bn{n}) (\fn{map} \dt{Var} \bn{argNames})
     \bn{env} <- \fn{getEnv}
     (\bn{lhsTyped}, _) <- \fn{check} \bn{env} \bn{lhsUntyped}
     \bn{holeName} <- \fn{fresh}
     \kw{let} \bn{rhs} = \dt{Bind} \bn{holeName} (\dt{GHole} \bn{retTy}) (\dt{V} \dt{0})
     \fn{pure} (\bn{MkFunClause} \bn{lhsTyped} \bn{rhs})
\end{Verbatim}
\end{figure}

The \fn{collectTypes} function we write in Idris is very similar to the one we
defined in \autoref{code:collectTypes} for the compiler implementation.
This one, however, also returns a proof that the resulting list is not empty.
The function takes a list and you the same list with the last element removed,
namely \fn{init}, requires a proof that the list is non-empty.

The \fn{addClause} tactic only takes one input, which is the name of the function
we will add an initial clause for.
Using this name, we look up the type of that function, and get its components
using \fn{collectTypes}. We know that there is at least one member in that
list, so we name the last element \bn{retTy}, for the return type, and we name
the rest \bn{argTys}, since they represent types of the arguments.
For each of member of \bn{argTys}, we generate a new name using
\fn{fresh}.\footnote{\fn{fresh} is defined in Hezarfen, not in a standard
library. The standard way of creating fresh names is \fn{gensym}, but we wrote
wrapper function \fn{fresh} that does not generate \dt{MN}, and gives variables
readable names, usually one-letter.} We later use these names and make them
into variable terms, using \dt{Var}, and create a function application using
all of these names. This application is supposed to represent the left-hand
side in our final definition. The right-hand side is a hole term that has the
type \bn{retTy}.  This concludes our type declaration term.

The Emacs Lisp code we write for this exactly the same as the existing
add-clause editor action, so we only have to change the part that Emacs sends a
message to the compiler. Our message now should refer to \fn{addClause} instead
of the built-in add-clause editor action.

Let's see this \Elab\ action at work in \autoref{code:exampleAddClause1}.

\begin{figure}[ht]
\caption{Example function to run \fn{addClause} on.}
\label{code:exampleAddClause1}
\begin{Verbatim}[framesep=2mm, label=\footnotesize{\normalfont{Idris}}, labelposition=topline]
\fn{example} : (\bn{name} : \ty{String}) -> (\bn{age} : \ty{Nat}) -> \ty{IO} \ty{()}
\end{Verbatim}
\end{figure}

If we put the cursor on \fn{example} and execute the Emacs Lisp code somehow, which is often done via a shortcut, we will get the result in \autoref{code:exampleAddClause2}.

\begin{figure}[ht]
\caption{Result of running \fn{addClause} on the \fn{example} function.}
\label{code:exampleAddClause2}
\begin{Verbatim}[framesep=2mm, label=\footnotesize{\normalfont{Idris}}, labelposition=topline]
\fn{example} : (\bn{name} : \ty{String}) -> (\bn{age} : \ty{Nat}) -> \ty{IO} \ty{()}
\fn{example} \bn{a} \bn{b} = \hole{c}
\end{Verbatim}
\end{figure}

This concludes the rudimentary replacement the the existing add-clause section.
For simplicity purposes, we have not covered the type signatures with implicit
arguments or interface constraints. One can write a tactic to handle
those cases as well.

\section{Theorem prover for intuitionistic propositional logic}\label{sec:hezarfen}

In this section, we describe the tactic Hezarfen\footnote{ The name is
  pronounced {[\textipa{hezaRf\ae n}]}, like ``has are fan'', and it means
    polymath in Turkish.  Tactic source code is available at
    \url{http://github.com/joom/hezarfen}.}, which can decide intuitionistic
propositional logic theorems.\footnote{Similar to Coq's \texttt{tauto} tactic.}
This tactic will be based on Dyckhoff's LJT~\cite{ljt} and its Haskell
implementation Djinn~\cite{djinn}, which generates Haskell expressions
that have a given type.

Djinn is a standalone program that takes commands
interactively, and when it generates an expression it prints it on the screen.
Instead, we want to design Hezarfen as a library that provides an \Elab\ action
that can be used as a tactic in proofs, and also as a custom editor action that
helps us when the built-in proof search mechanism does not suffice.

As a part of this library, we want to define the \Elab\ action we used in
\autoref{sec:communication}, with the type
\texttt{\fn{prover} : \ty{TTName} -> \ty{Elab} \ty{TT}}.
This tactic takes the name of the hole it is supposed to fill, and gives back a
\TT\ term in the \Elab\ monad.

Since Hezarfen's proof terms are sent back to the editor and put back into the
source code, we should aim to make our proof terms as simple as possible.
Hence, we should implement both proof term generation and simplification.

\subsection{Proof term generation}

In Hezarfen, define two types \ty{Context} and \ty{Sequent} to help us
represent the proof rules as manipulations of the goal type.

\begin{figure}[ht]
\caption{Definitions of \ty{Context} and \ty{Sequent} for Hezarfen.}
\label{code:hezarfenTypes}
\begin{Verbatim}[framesep=2mm, label=\footnotesize{\normalfont{Idris}}, labelposition=topline]
\kw{data} \ty{Context} = \dt{Ctx} (\ty{List} (\ty{TTName}, \ty{Raw})) (\ty{List} (\ty{TTName}, \ty{Raw}))
\kw{data} \ty{Sequent} = \dt{Seq} \ty{Context} \ty{Raw}
\end{Verbatim}
\end{figure}

A context is two lists of pairs that consist of names mapped to \Raw\ terms
that represent the type of the term that the name refers to.
The reason we want to have two lists is that we want to distinguish between the
\emph{consumed} and \emph{unconsumed} bindings. Once we use up an entry in the second list,
we delete it from the second list. But we may want to add new bindings, which
we do on the first list. Suppose we have $(C \vee D) \supset B$ in our context.
When we are checking the premises we want to remove it from the second list
and add $C \supset B$ and $D \supset B$ to the first list. In Dyckhoff's
presentation, this rule looks like:

\begin{prooftree}
  \AxiomC{$C \supset B,\ D \supset B,\ \Gamma \Longrightarrow G$}
  \UnaryInfC{$(C \vee D) \supset B,\ \Gamma \Longrightarrow G$}
\end{prooftree}
\vspace{\baselineskip}

But in Hezarfen's source code, this rule is written as in \autoref{code:hezarfenTypes}.

\begin{figure}[ht]
\caption{The ``\ty{Either} implies'' case in Hezarfen}
\label{code:hezarfenTypes}
\begin{Verbatim}[framesep=2mm, label=\footnotesize{\normalfont{Idris}}, labelposition=topline]
\fn{breakdown'} : \ty{Sequent} -> \ty{Elab} \ty{Tm}
\fn{breakdown'} \bn{goal} = \kw{case} \bn{goal} \kw{of}
  \cm{-- numerous previous cases}
  \dt{Seq} (\dt{Ctx} \bn{g} ((\bn{n}, \kw{\`{}(}(\dt{Either} ~\bn{d} ~\bn{e}) -> ~\bn{b}\kw{)}) :: \bn{o})) \bn{c} =>
    \kw{let} (\bn{n1}, \bn{n2}, \bn{newgoal}) = !(\fn{appDisjImplL} (\dt{Ctx} \bn{g} \bn{o}) (\bn{d}, \bn{e}, \bn{b}, \bn{c})) \kw{in}
    \kw{let} (\bn{l1}, \bn{l2}) = (!\fn{fresh}, !\fn{fresh}) \kw{in}
    \fn{pure} \fn{$} \dt{RBind} \bn{n1} (\dt{Let} \kw{\`{}(}~\bn{d} -> ~\bn{b}\kw{)}
                      (\dt{RBind} \bn{l1} (\dt{Lam} \bn{d}) (\dt{RApp} (\dt{Var} \bn{n})
                        \kw{\`{}(}\dt{Left} {\bn{a}=~\bn{d}} {\bn{b}=~\bn{e}} ~(\dt{Var} \bn{l1})\kw{)})))
          \fn{$} \dt{RBind} \bn{n2} (\dt{Let} \kw{\`{}(}~\bn{e} -> ~\bn{b}\kw{)}
                      (\dt{RBind} \bn{l2} (\dt{Lam} \bn{e}) (\dt{RApp} (\dt{Var} \bn{n})
                        \kw{\`{}(}\dt{Right} {\bn{a}=~\bn{d}} {\bn{b}=~\bn{e}} ~(\dt{Var} \bn{l2})\kw{)})))
            !(\fn{breakdown} \dt{False} \bn{newgoal})
\end{Verbatim}
\end{figure}

Understanding this code fully is not necessary; our goal is to give the greater
picture. This piece of code checks our second list in the context to see if
there is a name with the type \texttt{(\ty{Either} \bn{d} \bn{e}) -> \bn{b}},
for some \bn{d}, \bn{e} and \bn{b}. If there is, using this name \bn{n}, we can
create two functions, one with the type \texttt{\bn{d} -> \bn{b}} and the other
with the type \texttt{\bn{e} -> \bn{b}}. We generate two fresh names so we can
name these functions, and then we create a proof term, in which we generate
lambda bindings for these two functions we define in terms of \bn{n}. The rest
of the proof proceeds recursively.

The remaining rules of proof term generation proceed similarly. For more detail
on what the generated proof terms are, see Dyckhoff's paper or Hezarfen's
source code.

% explain the nested impl rule in depth ???
% and possibly one or two more, to show what they look like in general

\subsection{Simplification}

In \texttt{Hezarfen.Simplify}, we define a function \fn{reduce} that simplifies
a given \Raw\ term into another \Raw\ term in the \Elab\ monad.\footnote{We
depend on the \Elab\ monad for fresh name generation.}
A rudimentary implementation of this function that does not include all the
simplification steps is given in \autoref{code:reduce}.

\begin{figure}[ht]
\caption{Rudimentary implementation of \fn{reduce} in Hezarfen.}
\begin{Verbatim}[framesep=2mm, label=\footnotesize{\normalfont{Idris}}, labelposition=topline]
\fn{reduce} : \ty{Raw} -> \ty{Elab} \ty{Raw}
\fn{reduce} \fn{t} = \kw{case} \bn{t} \kw{of}
  \cm{-- Eta reduction:  (\textbackslash{}x => f x) becomes f}
  \dt{RBind} \bn{n} (\dt{Lam} \bn{b}) (\dt{RApp} \bn{t'} (\dt{Var} \bn{n'})) =>
    \kw{if} \bn{n} \fn{==} \bn{n'}
      \kw{then} \fn{reduce} \bn{t'}
      \kw{else} \fn{pure} \fn{$} \dt{RBind} \bn{n} (\dt{Lam} \bn{b}) !(\fn{reduce} (\dt{RApp} !(\fn{reduce} \bn{t'}) (\dt{Var} \bn{n'})))

  \cm{-- (id x) becomes x}
  \dt{RApp} (\dt{RApp} (\dt{Var} \kw{\`{}\{}\fn{id}\kw{\}}) \bn{c}) \bn{x} => \fn{reduce} \bn{x}

  \dt{RBind} \bn{n} \bn{b} \bn{t'} => \fn{pure} \fn{$} \dt{RBind} \bn{n} \bn{b} !(\fn{reduce} \bn{t'})
  \dt{RApp} \bn{t1} \bn{t2} => \fn{pure} \fn{$} \dt{RApp} !(\fn{reduce} \bn{t1}) !(\fn{reduce} \bn{t2})
  _ => \fn{pure} \bn{t}
\end{Verbatim}
\end{figure}

In the full implementation, we do more complex simplifications, such as
simplifying \texttt{(\textbackslash{}\bn{x} => \fn{g} (\fn{f} \bn{x}))} into
\texttt{(\fn{g} \fn{.} \fn{f})}, removing unused \kw{let} bindings,
substituting a \kw{let} binding in the body if the binding is only used once,
etc.

To fully simplify a \Raw\ term, we repeatedly apply it to \fn{reduce} until
fixpoint, which should be improved in future work.

There is also recent work on writing code generating theorem provers that
are more modular and efficient since they depend on an intermediate proof
representation that is later reconstructed.\cite{theoremProverCodeGeneration}
Hezarfen directly deals with the untyped core language syntax tree, both for
the input terms that represent types, and for the proof term it returns, and
this causes some overhead in our tactic.

\chapter{Background}\label{chap:background}

\newcommand{\zip}{\fn{zip}}
\section{Idris}

Idris is a dependently typed functional programming language. In simple terms,
dependent types allow us to do computation in types, just like we can do
computation in terms.\cite{davidphd} Moreover, we can use the computation
in types to shape our definitions of computations in terms.

A helpful intuition for functional programmers is to think about how the
concept that functions are values was a initially a novel idea, and now in
dependently typed programming, we promote types to values as well.
Thus a function can now take a term as an argument and return a type as a
result.\cite{lambdacube,henk}
For those familiar with Haskell type families\cite{typefamilies}, this is
similar to that, but notice that type families are like functions from types to types,
while we can have functions from terms to types.

In \autoref{code:vect}, we define a data type of vectors. It is exactly
like lists, except now the length is stored within the type. In other
words, as we add elements, the length is computed in type level.

After that, we define a function that zips two vectors. Notice that only two
cases are enough to cover all possible paths of this function. If we were to
define a \zip\ function for lists, we would need four cases: both empty, both
non-empty, one empty and one non-empty, and one non-empty and one empty.
However, our \zip\ function takes two arguments that are of the same length
\bn{n}.  Therefore, we cannot have a case with one empty and one non-empty,
because that contradicts with the fact that both vectors are of the same
length. Notice that the \zip\ function returns a vector of length \bn{n}.
In other words, the input vectors and the resulting vector are guaranteed to be
of the same length.

\begin{figure}[ht]
\caption{Example of a dependently typed Idris code: vectors}
\label{code:vect}
\begin{Verbatim}[framesep=2mm, label=\footnotesize{\normalfont{Idris}}, labelposition=topline]
\IdrisKeyword{data} \IdrisType{Vect} : \IdrisType{Nat} -> \IdrisType{Type} -> \IdrisType{Type} \IdrisKeyword{where}
  \IdrisData{Nil} : \IdrisType{Vect} \IdrisData{0} \IdrisImplicit{elem}
  \IdrisData{(::)} : \IdrisImplicit{elem} -> \IdrisType{Vect} \IdrisImplicit{len} \IdrisImplicit{elem} -> \IdrisType{Vect} (\IdrisData{S} \IdrisImplicit{len}) \IdrisImplicit{elem}

\IdrisFunction{zip} : \IdrisType{Vect} \IdrisImplicit{n} \IdrisImplicit{a} -> \IdrisType{Vect} \IdrisImplicit{n} \IdrisImplicit{b} -> \IdrisType{Vect} \IdrisImplicit{n} \IdrisType{(}\IdrisImplicit{a}\IdrisType{,} \IdrisImplicit{b}\IdrisType{)}
\IdrisFunction{zip} \IdrisData{[]} \IdrisData{[]} = \IdrisData{[]}
\IdrisFunction{zip} (\IdrisBound{x} \IdrisData{::} \IdrisBound{xs}) (\IdrisBound{y} \IdrisData{::} \IdrisBound{ys}) = \IdrisData{(}\IdrisBound{x}\IdrisData{,} \IdrisBound{y}\IdrisData{)} \IdrisData{::} \IdrisFunction{zip} \IdrisBound{xs} \IdrisBound{ys}
\end{Verbatim}
\end{figure}

In \autoref{code:evenodd}, we define
two similar data types that ensure that the given natural number is even or
odd, and compute the number within the type at every step. Later we give a
function that takes a natural number and generates either a value of the type
that ensures evenness or one that ensures oddity. This corresponds to a proof that for all $n \in \mathbb{N}$, $n$ is even or $n$ is odd.

\begin{figure}[ht]
\caption{Example of a dependently typed Idris code, parity of natural numbers}
\label{code:evenodd}
\begin{Verbatim}[framesep=2mm, label=\footnotesize{\normalfont{Idris}}, labelposition=topline]
\IdrisKeyword{data} \IdrisType{Even} : \IdrisType{Nat} -> \IdrisType{Type} \IdrisKeyword{where}
  \IdrisData{EvenZ} : \IdrisType{Even} \IdrisData{0}
  \IdrisData{EvenSS} : \IdrisType{Even} \IdrisImplicit{n} -> \IdrisType{Even} (\IdrisData{S} (\IdrisData{S} \IdrisImplicit{n}))

\IdrisKeyword{data} \IdrisType{Odd} : \IdrisType{Nat} -> \IdrisType{Type} \IdrisKeyword{where}
  \IdrisData{Odd1} : \IdrisType{Odd} \IdrisData{1}
  \IdrisData{OddSS} : \IdrisType{Odd} \IdrisImplicit{n} -> \IdrisType{Odd} (\IdrisData{S} (\IdrisData{S} \IdrisImplicit{n}))

\IdrisKeyword{total}
\IdrisFunction{evenOrOdd} : (\IdrisBound{n} : \IdrisType{Nat}) -> \IdrisType{Either} (\IdrisType{Even} \IdrisBound{n}) (\IdrisType{Odd} \IdrisBound{n})
\IdrisFunction{evenOrOdd} \IdrisData{0} = \IdrisData{Left} \IdrisData{EvenZ}
\IdrisFunction{evenOrOdd} \IdrisData{1} = \IdrisData{Right} \IdrisData{Odd1}
\IdrisFunction{evenOrOdd} (\IdrisData{S} (\IdrisData{S} \IdrisBound{n})) = \IdrisKeyword{case} \IdrisFunction{evenOrOdd} \IdrisBound{n} \IdrisKeyword{of}
                        \IdrisData{Left} \IdrisBound{ev} => \IdrisData{Left} (\IdrisData{EvenSS} \IdrisBound{ev})
                        \IdrisData{Right} \IdrisBound{o} => \IdrisData{Right} (\IdrisData{OddSS} \IdrisBound{o})
\end{Verbatim}
\end{figure}

Unlike other dependently typed languages like Agda and Coq, Idris is not total
by default. This is because Idris prioritizes general purpose programming
rather than theorem proving. However, users can opt in to enable totality
checking either for the entire module or for specific functions.
We did the latter for \fn{evenOrOdd} by using the keyword \kw{total}.
Similarly, we could require the \zip\ function from the previous example to be
total if we wanted to.\footnote{Clearly Idris cannot decide whether a given
function is total, since that would solve the halting problem. Instead it
acknowledges the ones that are obviously terminating, and for all the other
ones, even if they are actually total, it throws a totality check error.}

Haskell's type classes and type class instances are called interfaces and
implementations in Idris, respectively. In Haskell there can only be one
instance for the same type class and the type, but in Idris for the same
interface and the type, there can be multiple implementations. You can name
implementations and specify the implementation you want to use by its name,
when you are writing a function. For our purposes, we will not use multiple
implementations.

\section{Elaborator reflection}\label{sec:elabref}

Idris programs are elaborated from high-level Idris syntax trees into a core
language called \texttt{TT}, and then type checked.\cite{idris}
The implementation of the Idris elaboration in the compiler is written as a
Haskell monad called \texttt{Elab}.
Recent work on elaborator reflection\cite{elabref} allowed Idris users to
access this monad from Idris itself, by implementing a primitive monad
\Elab\ in Idris itself, that can only be used for metaprogramming in compile
time.

\subsection{Reflected core language types}\label{ssec:reflectedTypes}

Since the \Elab\ monad in Idris needs to work with core language terms and
definitions, we have to use data types in Idris representing them.
We want to have a correspondence between the internal data type that represents
the core language syntax in the compiler, and the ones defined in Idris.

The process of converting the syntax tree representation in the compiler, into
a term in Idris representing the same syntax tree, is called reflection.
The other direction, converting an Idris term that represents a syntax tree,
into the syntax tree representation of the compiler is called reification.
This allows us to manipulate Idris syntax trees in Idris itself.

In other words, there are types in Idris itself, that match the syntax
tree types in the compiler; we will call the ones we defined ``reflected
types''.

The most important one these types is called \TT,
which is type of the core language typed terms, and its definition can be seen
in \autoref{code:ttDef}.

\begin{figure}[ht]
\caption{The reflected type \protect\TT\ in Idris.}
\label{code:ttDef}
\begin{Verbatim}[framesep=2mm, label=\footnotesize{\normalfont{Idris}}, labelposition=topline]
\IdrisKeyword{data} \IdrisType{TT} : \IdrisType{Type} \IdrisKeyword{where}
  \IdrisData{P} : \IdrisType{NameType} -> \IdrisType{TTName} -> \IdrisType{TT} -> \IdrisType{TT}
  \IdrisData{V} : \IdrisType{Int} -> \IdrisType{TT}
  \IdrisData{Bind} : \IdrisType{TTName} -> \IdrisType{Binder} \IdrisType{TT} -> \IdrisType{TT} -> \IdrisType{TT}
  \IdrisData{App} : \IdrisType{TT} -> \IdrisType{TT} -> \IdrisType{TT}
  \IdrisData{TConst} : \IdrisType{Const} -> \IdrisType{TT}
  \IdrisData{Erased} : \IdrisType{TT}
  \IdrisData{TType} : \IdrisType{TTUExp} -> \IdrisType{TT}
  \IdrisData{UType} : \IdrisType{Universe} -> \IdrisType{TT}
\end{Verbatim}
\end{figure}

As a quick summary:
\begin{itemize}
\item\dt{P} creates a variable term from a name and the type of the variable.
\item\dt{V} creates a de Bruijn variable.
(given integer $n$ representing the $n$th most recently introduced local variable)
\item\dt{Bind} creates any kind of binder (lambda, let etc.) with a term it binds on.
\item\dt{App} creates a function application.
\item\dt{TConst} creates a constant such as an integer, a character, a string etc.
\item\dt{Erased} creates a term that is not known. This is used for erasing the types we do not need later in the compilation.
\item\dt{TType} creates a type of types for a given universe.
\item\dt{UType} creates a uniqueness type for a given uniqueness universe.
\end{itemize}

This summary is meant to be an overview, so refer to \cite{idris} and
\cite{elabref} if this is not perfectly clear. For our purposes, we will mostly
be concerned with \dt{P}, \dt{Bind} and \dt{App}.

The other important type that is used in elaborator reflection is \ty{Raw},
which is the type of untyped core language terms, and its definition can be
seen in \autoref{code:rawDef}.

\begin{figure}[ht]
\caption{The reflected type \protect\ty{Raw} in Idris.}
\label{code:rawDef}
\begin{Verbatim}[framesep=2mm, label=\footnotesize{\normalfont{Idris}}, labelposition=topline]
\IdrisKeyword{data} \IdrisType{Raw} : \IdrisType{Type} \IdrisKeyword{where}
  \IdrisData{Var} : \IdrisType{TTName} -> \IdrisType{Raw}
  \IdrisData{RBind} : \IdrisType{TTName} -> \IdrisType{Binder} \IdrisType{Raw} -> \IdrisType{Raw} -> \IdrisType{Raw}
  \IdrisData{RApp} : \IdrisType{Raw} -> \IdrisType{Raw} -> \IdrisType{Raw}
  \IdrisData{RType} : \IdrisType{Raw}
  \IdrisData{RUType} : \IdrisType{Universe} -> \IdrisType{Raw}
  \IdrisData{RConstant} : \IdrisType{Const} -> \IdrisType{Raw}
\end{Verbatim}
\end{figure}

The constructors of \ty{Raw} are almost the same ones as \TT,
except a few of them are missing and variables do not have to be annotated with
their types.


The \ty{TTName} type is the type of names in the core language, its full definition can be seen in \autoref{code:ttnameDef}.

\begin{figure}[ht]
\caption{The reflected type \protect\ty{TTName} in Idris.}
\label{code:ttnameDef}
\begin{Verbatim}[framesep=2mm, label=\footnotesize{\normalfont{Idris}}, labelposition=topline]
\IdrisKeyword{data} \IdrisType{TTName} : \IdrisType{Type} \IdrisKeyword{where}
  \IdrisData{UN} : \IdrisType{String} -> \IdrisType{TTName}
  \IdrisData{NS} : \IdrisType{TTName} -> \IdrisType{List} \IdrisType{String} -> \IdrisType{TTName}
  \IdrisData{MN} : \IdrisType{Int} -> \IdrisType{String} -> \IdrisType{TTName}
  \IdrisData{SN} : \IdrisType{SpecialName} -> \IdrisType{TTName}
\end{Verbatim}
\end{figure}

As a quick summary:
\begin{itemize}
\item\dt{UN} represents the variable names without any namespace.
\item\dt{NS} represents the variable names with a given namespace. For example, the name \texttt{Prelude.Bool.True} is represented as
  \texttt{\dt{NS} (\dt{UN} \dt{"True"}) \dt{["Bool", "Prelude"]}}.
\item\dt{MN} represents machine generated names with a hint string and a fresh integer for that hint.
\item\dt{SN} represents special names, which are used for metavariables, implementations etc. We will not deal with them in this thesis.
\end{itemize}

As a quick way to refer to Idris names, there is a syntax
\texttt{\IdrisKeyword{\`{}\{\{}x\IdrisKeyword{\}\}}}
that would give you the term \dt{UN} \dt{"x"}.
Similarly, there is another syntax that checks whether a given name exists and
lets you refer to an existing name without having to specify its full
namespace: \texttt{\IdrisKeyword{\`{}\{}False\IdrisKeyword{\}}} would give you
\texttt{\dt{NS} (\dt{UN} \dt{"False"}) \dt{["Bool", "Prelude"]}}.

\medskip

There are many other types used in the reflection of the core language, but we
will not give their definitions here since they are not as common as \TT,
\ty{Raw}, and \ty{TTName}. However, it would be useful to at least list the
important ones and describe what they represent.

\begin{itemize}
\item\ty{TyDecl} represents type declarations.
\item\ty{DataDefn} represents data type definitions.
\item\ty{FunDefn} represents function definitions.
\item\ty{FunClause} represents a single clause in a function definition.
\end{itemize}

\subsection{Quotations}\label{ssec:quotation}

Writing \TT\ and \ty{Raw} terms by hand can get tedious, hence there is a
quotation syntax that elaborates a given expression into its corresponding
\TT\ or \ty{Raw} term.\cite{idrisQuotation}
The syntax \texttt{\`{}(\fn{e})}, where \fn{e} is an Idris expression, gives
us the typed or untyped core language syntax tree for \fn{e}. For example,
\texttt{\`{}(\fn{not} \dt{True})} gives us the following \TT\ term:

\begin{figure}[ht]
  \caption{The \TT\ term we get when we quote \texttt{\fn{not} \dt{True}}.}
\begin{Verbatim}[framesep=2mm, label=\footnotesize{\normalfont{Idris}}, labelposition=topline]
\IdrisData{App} (\IdrisData{P} \IdrisData{Ref}
       (\IdrisData{NS} (\IdrisData{UN} \IdrisData{"not"}) \IdrisData{[}\IdrisData{"Bool"}\IdrisData{,} \IdrisData{"Prelude"}\IdrisData{]})
       (\IdrisData{Bind} (\IdrisData{UN} \IdrisData{"__pi_arg"})
             (\IdrisData{Pi} (\IdrisData{P} (\IdrisData{TCon} \IdrisData{8} \IdrisData{0}) (\IdrisData{NS} (\IdrisData{UN} \IdrisData{"Bool"}) \IdrisData{[}\IdrisData{"Bool"}\IdrisData{,} \IdrisData{"Prelude"}\IdrisData{]}) \IdrisData{Erased})
                 (\IdrisData{TType} (\IdrisData{UVar} \IdrisData{"./Prelude/Bool.idr"} \IdrisData{71})))
             (\IdrisData{P} (\IdrisData{TCon} \IdrisData{8} \IdrisData{0}) (\IdrisData{NS} (\IdrisData{UN} \IdrisData{"Bool"}) \IdrisData{[}\IdrisData{"Bool"}\IdrisData{,} \IdrisData{"Prelude"}\IdrisData{]}) \IdrisData{Erased})))
    (\IdrisData{P} (\IdrisData{DCon} \IdrisData{1} \IdrisData{0})
       (\IdrisData{NS} (\IdrisData{UN} \IdrisData{"True"}) \IdrisData{[}\IdrisData{"Bool"}\IdrisData{,} \IdrisData{"Prelude"}\IdrisData{]})
       (\IdrisData{P} (\IdrisData{TCon} \IdrisData{0} \IdrisData{0}) (\IdrisData{NS} (\IdrisData{UN} \IdrisData{"Bool"}) \IdrisData{[}\IdrisData{"Bool"}\IdrisData{,} \IdrisData{"Prelude"}\IdrisData{]}) \IdrisData{Erased}))
\end{Verbatim}
\end{figure}

The \ty{Raw} term for the same expression is a bit more reasonable:
\begin{figure}[ht]
  \caption{The \Raw\ term we get when we quote \texttt{\fn{not} \dt{True}}.}
\begin{Verbatim}[framesep=2mm, label=\footnotesize{\normalfont{Idris}}, labelposition=topline]
(\IdrisData{RApp} (\IdrisData{Var} (\IdrisData{NS} (\IdrisData{UN} \IdrisData{"not"}) \IdrisData{[}\IdrisData{"Bool"}\IdrisData{,} \IdrisData{"Prelude"}\IdrisData{]}))
      (\IdrisData{Var} (\IdrisData{NS} (\IdrisData{UN} \IdrisData{"True"}) \IdrisData{[}\IdrisData{"Bool"}\IdrisData{,} \IdrisData{"Prelude"}\IdrisData{]})))
\end{Verbatim}
\end{figure}

Obviously, we would not want to write terms like these manually every time we want to return
the syntax tree for a simple function application.
At times like this, quotation saves us, for both expressions and patterns.

We can also give the type of the expression we want to elaborate, which becomes
necessary when Idris cannot infer the type. For \dt{True}, it is trivial to
infer that type is \ty{Bool}, but for \dt{5}, the type can be \ty{Int},
\ty{Integer}, \ty{Nat}, or anything that satisfies the \ty{Num} interface.
Therefore, we have to specify the type when we are quoting. The
syntax for that is \texttt{\`{}(\fn{e} : \ty{t})},
e.g. \texttt{\`{}(\dt{5} : \ty{Nat})}.

We also can do antiquotation. If we have some variable expression \fn{x} that has the
type \TT\ or \ty{Raw}, then we can construct a syntax tree using it within the
quotation, with the syntax \texttt{\`{}(\fn{not} ~\fn{x})}.
Note that antiquotation works for expressions, not just variables. The type of
expression or variable we have in the antiquotation has to match the type of
the quotation. In other words, only \TT\ expressions can be used in an
antiquotation in a \TT\ quotation, and mutatis mutandis for \ty{Raw}.

\subsection{\protect\Elab\ monad}

The elaborator reflection\cite{elabref} feature that has been added to the
Idris compiler recently provides a tool for metaprogramming with a monad called
\Elab.  This monad is implemented as a primitive and it can only be run during
compile time, or in the interactive proof shell.

Elaborator reflection adds a new declaration
\texttt{\IdrisKeyword{\%runElab}}\ \fn{e} to Idris,
where \fn{e} has the type \ty{Elab}\ \ty{()}.
This declaration runs the \Elab\ action and adds new type declarations,
function and data type definitions generated by the \Elab\ action generated by
\fn{e} to the context.

Elaborator reflection also adds a new expression
\texttt{\IdrisKeyword{\%runElab}}\ \fn{e} to Idris,
where \fn{e} has the type \ty{Elab}\ \ty{()}.
The type \ty{t} of the entire expression is started as the goal of the
\Elab\ action, and the tactics in \fn{e} must solve the goal
that has the type \ty{t}.
Like the declaration above, this expression also adds the newly generated
declarations and definitions to the context.

\Elab\ monad holds a proof state inside, which has a goal type, a proof term that is incrementally built up, a hole queue, a collection of open unification problems, and a supply of fresh names.\cite{elabref}
Tactics can change the proof state. Here are some examples that do that:
\begin{itemize}
\item\texttt{\IdrisFunction{claim} : \IdrisType{TTName} -> \IdrisType{Raw} -> \IdrisType{Elab} \IdrisType{()}}\\
Creates a new hole with a given name and a type.
\item\texttt{\IdrisFunction{fill} : \IdrisType{Raw} -> \IdrisType{Elab} \IdrisType{()}}\\
Create a guess to fill the current hole with a term. Fail if the types do not unify.
\item\texttt{\IdrisFunction{solve} : \IdrisType{Elab} \IdrisType{()}}\\
Try to finalize the guess in the hole. Fail if there is no guess.
\end{itemize}

There are a lot more tactics, which we will not list here. A more thorough list
can be found in \cite{elabref} and Idris documentation.

We also have access to \Elab\ actions that do not change the proof state, but give us access to the context or other compiler primitives:
\begin{itemize}
\item\texttt{\IdrisFunction{check} : \IdrisType{List} \IdrisType{(}\IdrisType{TTName}\IdrisType{,} \IdrisType{Binder} \IdrisType{TT}\IdrisType{)} -> \IdrisType{Raw} -> \IdrisType{Elab} \IdrisType{(}\IdrisType{TT}\IdrisType{,} \IdrisType{TT}\IdrisType{)}}\\
Type-checks a term under a given environment and gives the typed core term version of the \ty{Raw} term and the type of it as a typed core term.
\item\texttt{\IdrisFunction{normalise} : \IdrisType{List} \IdrisType{(}\IdrisType{TTName}\IdrisType{,} \IdrisType{Binder} \IdrisType{TT}\IdrisType{)} -> \IdrisType{TT} -> \IdrisType{Elab} \IdrisType{TT}}\\
Normalizes\footnote{\fn{normalise} is spelled the British way, since most Idris development happens in the UK.} a typed term under a given environment.
\item\texttt{\IdrisFunction{lookupTy} : \IdrisType{TTName} -> \IdrisType{Elab} (\IdrisType{List} \IdrisType{(}\IdrisType{TTName}\IdrisType{,} \IdrisType{NameType}\IdrisType{,} \IdrisType{TT}\IdrisType{)})}\\
Looks up the type of the given name and returns the ones it finds in a list, in case the name is ambiguous.
\end{itemize}

Observe that in most of these functions, for inputs we use \ty{Raw}, the
untyped core language terms, and results are in \ty{TT}, the typed core
language terms.  This is because untyped core language terms are easier to
write for the tactic users, and type-checking them in the elaborator is easy.

Now let's define a function using elaborator reflection.
Let's define a polymorphic identity function.


\begin{figure}[ht]
  \caption{A proof of the identity function using elaborator reflection in Idris.}
\begin{Verbatim}[framesep=2mm, label=\footnotesize{\normalfont{Idris}}, labelposition=topline]
\fn{id} : (\bn{a} : \ty{Type}) -> \bn{a} -> \bn{a}
\fn{id} = \kw{%runElab} (\kw{do} \fn{intro} \kw{`\{\{}ty\kw{\}\}}
                  \fn{intro} \kw{`\{\{}a\kw{\}\}}
                  \fn{fill} (\dt{Var} \kw{`\{\{}a\kw{\}\}})
                  \fn{solve})
\end{Verbatim}
\end{figure}

For anyone familiar with Coq, this will look very similar to a normal Coq
proof. First we take the type as an argument, and then a value of that type,
and we return the same value. Elaborator reflection proofs are a bit more
unpolished compared to Coq proofs because of the core language terms,
quotation and special name syntax, but it is essentially very similar,
hence the name ``tactics'' we use to refer to monadic \Elab\ actions.

Let's prove something not as trivial. This time we want to prove that
$(\forall n \in \mathbb{N})\ n = n + 0$, for the standard definition of
addition. Since that requires more complex tactics like induction, we will
import the Pruviloj\footnote{Pruviloj is the Idris library included in
the Idris distribution. It contains complex tactics written with elaborator
reflection.} library.\cite{davidphd}

\begin{figure}[ht]
  \caption{A proof that $(\forall n \in \mathbb{N})\ n = n + 0$ using elaborator reflection in Idris.}
\begin{Verbatim}[framesep=2mm, label=\footnotesize{\normalfont{Idris}}, labelposition=topline]
\fn{nPlusZero} : (\bn{n} : \ty{Nat}) -> \bn{n} = \fn{plus} \bn{n} \dt{0}
\fn{nPlusZero} = \kw{%runElab} (\kw{do} \fn{intro} \kw{`\{\{}n\kw{\}\}}
                         \fn{induction} (\dt{Var} \kw{`\{\{}n\kw{\}\}})
                         \fn{compute}
                         \fn{reflexivity}
                         \fn{compute}
                         \fn{attack}
                         \fn{intro} \kw{`\{\{}n1\kw{\}\}}
                         \fn{intro} \kw{`\{\{}indHyp\kw{\}\}}
                         \fn{rewriteWith} (\dt{Var} \kw{`\{\{}indHyp\kw{\}\}})
                         \fn{reflexivity}
                         \fn{solve})
\end{Verbatim}
\end{figure}

The proof proceeds as such: we first take in the argument \bn{n}, and then do
an induction on \bn{n}. Because of the way induction works in Pruviloj, we have
to simplify the goal using \fn{compute}\footnote{Its Coq equivalent would be
\texttt{simpl}.}.
And for the base case the goal is just proving \dt{0} \dt{=} \dt{0}.
For the inductive step, we have to restructure the goal with \fn{attack} and
then reintroduce the input and then introduce the induction hypothesis. Then we
rewrite the goal with the induction hypothesis and then the goal becomes
trivial. Understanding this proof completely is not crucial for this thesis,
but if you want to fully comprehend \fn{attack} and {solve}, you can refer to
\cite{elabref}.

Finally let's do an example for a declaration with elaborator reflection:

\begin{figure}[ht]
  \caption{A type declaration and new definition for \fn{n}, using elaborator reflection in Idris.}
\begin{Verbatim}[framesep=2mm, label=\footnotesize{\normalfont{Idris}}, labelposition=topline]
\kw{%runElab} (\kw{do} \fn{declareType} (\dt{Declare} \kw{\`{}\{\{}n\kw{\}\}} \dt{[]} \kw{`{}(}\ty{Nat}\kw{)})
             \fn{defineFunction} (\dt{DefineFun} \kw{\`{}\{\{}n\kw{\}\}}
                              \dt{[}\dt{MkFunClause} (\dt{Var} \kw{\`{}\{\{}n\kw{\}\}}) \kw{\`{}(}\dt{Z}\kw{)}\dt{]}))
\end{Verbatim}
\end{figure}

The example above first declares that \fn{n} will have the type \ty{Nat}.
Then it defines it as \texttt{\IdrisFunction{n} = \IdrisData{Z}}.

Now that we have seen different use cases for elaborator reflection, we can move on to the
design of the edit-time tactics feature.


% Previous versions of Idris had a tactic based prover\footnote{As of Idris
% 1.1.1 it is still available, with the warning that it will be deprecated in the
% future versions.}, which embedded the proof tactics in a Haskell monad in the
% implementation.\cite{elabref}

% what sorts of things stay, what sorts of things go away with elaboration
% mcbride dependent pattern matching? (Epigram paper?)
% standard eliminators?
% (a few constructions on constructors: injectivity and disjointness)
% icfp 2016 - Jesper Cockx, doing that without axiom K
% how to get rid of dep pattern matching and turning it into induction principle

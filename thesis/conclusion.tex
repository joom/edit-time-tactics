\chapter{Conclusion}\label{chap:conclusion}

\section{Future work}

\subsection{Proof simplification}

The ability to run tactics as editor actions has a consequence
that we have not explored much in this thesis.
Idris programs are usually meant to be executed, unlike Coq or
Agda programs, which are usually only meant to be type checked.\footnote{Of
  course there are backends for these languages, such as the OCaml
  backend for Coq and the Haskell backend for Agda.}
This means Idris programs will be compiled more often than Agda or Coq in the
long run, since not much code is added to proofs after its completion, but
practical programs tend to change in time. Therefore, minimizing compile time
is more of a priority for Idris.
Idris tactics generate proof terms in compile time, but the
compilation can take a long time for complex tactics\footnote{Similar problems
arise in Coq as well. For example, theorems that use the famous \texttt{omega}
tactic that decides Presburger arithmetic\cite{omega} take a long time to compile, and it
usually generates a huge proof term.},
not to mention that the implementation of elaborator
reflection in Idris has significant performance issues.\cite{leanmeta}
Yet we still want to utilize complex tactics to generate proofs or terms.
Using edit-time tactics, one would run a tactic once from the editor, generate
the proof term and serialize and send that to the editor and put it back in the
file.
If we think of the differences between the traditions of writing the proof
terms directly and writing tactics, the former more common in Agda and Idris
and the latter in Coq and Isabelle, this work will constitute a one way bridge
between the two, by making use of the elaborator reflection to create proof
terms in the editor in a smarter and quicker way.

The problem with that approach is that the generated proof terms can be (and
often are) gigantic and hideous, especially if generating a minimal proof term
is not a priority for the tactic we are using.
If there was a generic mechanism to simplify and minimize the generated proof
terms, and even write them in a way that makes use of dependent pattern
matching, then this could have been a more usable consequence of this work.
Ideally, we would want the artifact we are handing in to the reader of our
proofs to look just like what it would be if we had not used this system.
We leave that for future work.  However, even without proof simplification,
edit-time tactics still could be a last resort solution to long compile times
for tactics.

\subsection{Writing an editor action frontend in Idris}

We explained in \autoref{chap:introduction} that this thesis focuses on writing
the backend of an editor action in Idris, and that we still had to write some
Emacs Lisp (if we are using Emacs). However, Idris supports many different
code generation targets\cite{idriscodegen} seamlessly.

For example, since compiling to JavaScript is built-in, we can use JavaScript
code generation to write the editor interaction frontend for Visual Studio Code
and Atom.

There are also experimental projects on compiling Idris to Emacs
Lisp\footnote{Steven Shaw's work on compiling Idris to Emacs Lisp:
\url{http://github.com/steshaw/idris-elisp}} and VimL
(Vimscript)\footnote{Oskar Wickstr\"om and Soham Chowdhury's work on
compiling Idris to VimL:
\url{https://github.com/owickstrom/idris-vimscript}}. These projects are not
mature enough yet, but we believe they have the potential to inspire different
applications of metaprogramming, especially if the Idris modes of these editors
are written in Idris via their respective code generation targets.

\section{Final words}

In this thesis we extended the capabilities of the editor interaction mode of
Idris, by allowing users to define new editor actions in Idris itself. We did
so through a metaprogramming technique that was introduced to Idris recently by
Christiansen and Brady\cite{elabref}.

Editors communicate with the compiler via S-expressions, so we gave users the
power to dictate how a value of a given Idris type should exactly be
communicated; through the \ty{Editorable} interface users are now able to
define how a received S-expression should be parsed by the compiler, and how
the compiler should send the result as an S-expression. To achieve this, we
reflected the \ty{SExp} type to Idris, and extended elaborator reflection
by adding new \Elab\ primitives, with which we defined the \ty{Editorable}
implementations for Idris types representing the Haskell representation of
Idris core language terms.

Using this feature, we showed a simple \fn{toy} example, and then the
\fn{addClause} example that can replace an existing built-in editor action,
and Hezarfen, which is meant to be a better proof search mechanism then the
built-in one. We believe there is potential to write even more of the built-in
editor actions as edit-time tactics. We also believe that as more decision
procedures are coded up in Idris, edit-time tactics can become a more popular
feature.
Hopefully our work will get dependently-typed languages one step closer to the
state-of-the-art IDEs, and even give them an edge by allowing the reuse of the
existing metaprogramming mechanisms and tactic engineering efforts to write
editor actions.

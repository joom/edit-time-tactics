\section{Design} \label{sec:design}

We described in \autoref{sec:introduction} what a simple editor interaction
action like case splitting does.  What we want to do is to add another command
that would be recognized by our compiler, and specify how such a command would
run, and what arguments it should take to communicate with the editor
effectively.

The current Idris\footnote{This thesis is using Idris 1.1.1.} implementation of
the editor interaction mode is a part of the Idris compiler, and is written in
Haskell. When the editor is running,
it runs the \path{idris} executable with the \path{--ide-mode} flag, which
allows socket communication with the program through a machine-readable
syntax.\footnote{The Idris mode of Vim works differently, since Vim did not
support asynchronous jobs until 8.0. This is expected to change in the near
future.}
To be more precise, this program receives S-expressions\cite{mccarthy} as input
over the socket and sends back S-expressions as output. Let's see an example of
this. Suppose we are writing a predecessor function on natural numbers and we
want to case split on the argument.

\begin{Verbatim}
\IdrisFunction{pred} : \IdrisType{Nat} -> \IdrisType{Nat}
\IdrisFunction{pred} \IdrisBound{x} = \IdrisMetavar{?\IdrisMetavar{p}}
\end{Verbatim}

When we put the cursor on \path{x} and run case splitting, the following
communication happens in the background.

\begin{Verbatim}
-> ((:case-split 5 "x") 13)
<- (:return (:ok "pred Z = ?p_1{\textbackslash}npred (S k) = ?p_2{\textbackslash}n") 13)
\end{Verbatim}

Case split command requires two arguments, the line number and the variable
name to case split on. The line number is included in most editor actions that
change the code. The variable name is a string that hold the necessary data.
The number \path{13} that comes afterwards is how many rounds of communications
have been done so far. Notice that the response also carries the same number.
The response contains \path{:ok}, which means the case splitting succeeded,
and then a string that contains two lines split by the new line character.
The editor receives twose two lines and replaces that with the line we ran the
case split command on.

The most important thing to notice here is that the way the editor and the
compiler communicate is very ``stringly typed''\footnote{A term coined by Mark
Simpson, mentioned in a deleted Stack Overflow thread.}.  Since all of these
editor actions are primitives, this is initially fine. However, we now want to
create a single command that handles different kinds of actions. These actions
possibly take multiple arguments that hold different kinds of values, such as
names, type declarations, function definitions, or expressions.
The data itself can still be sent as a string by the editor, but having type
annotations would give a hint to the compiler about how to parse the given
string into parts of a syntax tree. In other words, the command should take a
list of strings, each tagged with a type annotation.

Then the question is, how will the editor know what to tag each string with?
It will not. For each non-trivial editor action, the user will have to extend
the Idris mode of the editor with a function that tags each string with a type
annotation. The resulting communication should look like this for case splitting.

\begin{Verbatim}
-> ((:elab-action :case-split 5 '((:name . "x"))) 13)
<- (:return (:ok '((:fun-defn . "pred Z = ?p_1")
                   (:fun-defn . "pred (S k) = ?p_2"))) 13)
\end{Verbatim}

Notice that the call to the editor states that we are calling the \Elab
action that does case splitting, and it takes an argument list, which happens
to be a singleton list that contains a pair of a tag and a string. The tag
states that the string should be handled as a variable name by the compiler.
The result from the editor returns that the action succeeded and then a list of
results.  In the result list we have two entries, each of which is a pair of a
tag and a string.  The tags in here state that the string represents a function
definition.

Now let's think of the other end of this editor action. What would the type of
such an \Elab action be?

% caseSplit : Int -> TTName -> Elab (List FunDefn)
% caseSplit

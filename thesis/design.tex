\chapter{Design}\label{chap:design}
The goal of this project is to allow users to write custom editor actions using
elaborator reflection.
The editor and the compiler have to communicate to do that; the editor must
send the name of the action and the necessary information, and the
compiler should then send the result back.
In this chapter, we will look at how the built-in editor actions work, and then
see what kind of restrictions this brings, and how this shapes the design of
edit-time tactics.

\section{Communication}\label{sec:communication}

The current Idris implementation\footnote{This thesis is using Idris 1.2.0.} of
the editor interaction mode is a part of the Idris compiler, and is written in
Haskell.
The editor runs an instance of the \texttt{idris} executable with the
\texttt{--ide-mode} flag, which
allows socket communication with the program through a machine-readable
syntax.\footnote{The Idris mode of Vim works differently, since Vim did not
support asynchronous jobs until version 8.0. This should change in the near future.}
To be more precise, the compiler receives S-expressions\cite{mccarthy} as input
over the socket and sends back S-expressions as output.

Let's revise the editor actions we looked at in \autoref{chap:introduction}.
We have the following piece of code, which will complete to the height function
on binary trees. The next step is to add a the initial function clause.

\begin{figure}[ht]
\caption{Initial \fn{height} function.}
\begin{Verbatim}[framesep=2mm, label=\footnotesize{\normalfont{Idris}}, labelposition=topline]
\IdrisFunction{height} : \IdrisType{Tree} \bn{a} -> \IdrisType{Nat}
\end{Verbatim}
\end{figure}

When we place the cursor on \fn{height} and run the action to add the initial
function clause, the editor sends the message on the first line in
\autoref{code:commAddClause} to the compiler, after doing the necessary
computation, the compiler responds with the message on the second line.

\begin{figure}[ht]
  \caption{The communication for add clause editor action on \fn{height}.}
  \label{code:commAddClause}
\begin{Verbatim}[framesep=2mm, label=\footnotesize{\normalfont{S-expression}}, labelposition=topline]
-> ((:add-clause 4 "height") 8)
<- (:return (:ok "height t1 = ?height_rhs") 8)
\end{Verbatim}
\end{figure}

Let's look at how these messages are formed. The built-in add clause command
needs to know the line number after which we are adding a clause, which is
\texttt{5}, and the name of the function we are adding a clause for, which is
\bn{height}.
Also, for communication purposes, we keep track of how many rounds of
communications have been done so far. The number \texttt{8} we have at the end
of the message signifies that this is the 8th correspondence so far.
Observe that the response also carries the same number.

The response contains \texttt{:ok}, which means the clause adding succeeded,
and then a string that contains a line of code.
When the editor receives that, it adds this new code to the next
line. Note that that last step is done by the front-end of the editor
interaction mode, i.e. in Emacs Lisp if we are in Emacs.

The next step in writing the function \fn{height} is to look at the possible
cases of the tree, i.e. a case-splitting action. This is the piece of code we
have before that action:

\begin{figure}[ht]
  \caption{\fn{height} function after adding an initial clause}
\begin{Verbatim}[framesep=2mm, label=\footnotesize{\normalfont{Idris}}, labelposition=topline]
\IdrisFunction{height} : \IdrisType{Tree} \bn{a} -> \IdrisType{Nat}
\IdrisFunction{height} \IdrisBound{t1} = \hole{height_rhs}
\end{Verbatim}
\end{figure}

When we place the cursor on \bn{t1} and run the case-splitting action, the
communication in \autoref{code:commCaseSplit} happens between the editor and
the compiler.

\begin{figure}[ht]
\caption{The communication for case-split editor action on \fn{height}.}
\label{code:commCaseSplit}
\begin{Verbatim}[framesep=2mm, label=\footnotesize{\normalfont{S-expression}}, labelposition=topline]
-> ((:case-split 5 "t1") 11)
<- (:return (:ok
     "height Empty = ?height_rhs_1{\textbackslash}n
      height (Node x t1 t2) = ?height_rhs_2{\textbackslash}n") 11)
\end{Verbatim}
\end{figure}

The built-in case-split command needs to know the line number on which we are
case-splitting, which is \texttt{5}, and the name of the argument we are
splitting, which is \bn{t1}.  The response contains \texttt{:ok}, which means
the case-splitting succeeded, and then a string that contains two lines split
by the new line character. When the editor receives this information, it
replaces the line the cursor was on before with the new code it received.

We have now seen how exactly the current editor action communication between
the editor and the compiler works through S-expressions.
Now we want to introduce an S-expression format that can capture any of these
commands; we want to generalize the ones we have seen so far.

Suppose we have the following \Elab\ action that we want to run from the editor:
\texttt{\fn{prover} : \ty{TTName} -> \ty{Elab} \ty{TT}}.
We will see in detail in \autoref{sec:hezarfen} how this tactic is
implemented. For now let's only look at how it is used. It takes a name of a
hole, and returns the proof term to fill the hole with.
Let's try to prove a trivial lemma with this tactic.

\begin{figure}[ht]
\caption{Type declaration for a simple theorem \fn{f}, with an incomplete definition.}
\begin{Verbatim}[framesep=2mm, label=\footnotesize{\normalfont{Idris}}, labelposition=topline]
\fn{f} : \ty{Either} \ty{Unit} \ty{Void}
\fn{f} = \hole{q}
\end{Verbatim}
\end{figure}

When we place the cursor on \hole{q} and tell the editor to run the \fn{prover}
tactic, we have to convey a couple things to the compiler: We have to tell that
we want to run an \Elab\ action named \fn{prover}, and we are running this on
the hole \hole{q}, and where exactly the cursor is (in line and column numbers)
when we run this editor action. Here is the communication we want to have for this:

\begin{figure}[ht]
\caption{The communication for the custom \fn{prover} action on the hole \hole{q}.}
\begin{Verbatim}[framesep=2mm, label=\footnotesize{\normalfont{S-expression}}, labelposition=topline]
-> ((:elab-edit "prover" '("q") 21 5) 22)
<- (:return (:ok "Left ()") 22)
\end{Verbatim}
\end{figure}

The editor tells the compiler that we want to run an \Elab\ action named
\fn{prover}. But the actions we may want to run take arguments, therefore we
pass a list S-expression \texttt{'("q")} that corresponds to the arguments
\fn{prover} takes.
In the response we get from the compiler, we get a piece of code that is
supposed to replace the hole. In \hyperref[sec:types]{the next section}, we
will discuss what determines the kind of S-expression we should send and
receive for a given type.

\section{Types of editor actions}\label{sec:types}

Users of our work will write \Elab\ actions that will then be called from the
editor via S-expressions that are sent to the compiler. We outlined above that
in the S-expression, we need to pass the arguments of the specific
\Elab\ action we want to call. Since we have to be able to send the arguments
in an S-expression and also send the result back in another S-expression, we
have limits on what kind of arguments an S-expression can take.
In other words, all arguments and the result of an \Elab\ editor action
have to serializable and deserializable.

Since serialization is done via S-expressions, we need to reflect the type of
S-expressions to Idris.

\begin{figure}[ht]
  \caption{The reflected type \ty{SExp} in Idris.}
  \label{code:definitionSExp}
\begin{Verbatim}[framesep=2mm, label=\footnotesize{\normalfont{Idris}}, labelposition=topline]
\kw{data} \ty{SExp} : \ty{Type} \kw{where}
  \dt{SExpList} : \ty{List} \ty{SExp} -> \ty{SExp}
  \dt{StringAtom} : \ty{String} -> \ty{SExp}
  \dt{BoolAtom} : \ty{Bool} -> \ty{SExp}
  \dt{IntegerAtom} : \ty{Integer} -> \ty{SExp}
  \dt{SymbolAtom} : \ty{String} -> \ty{SExp}
\end{Verbatim}
\end{figure}

An S-expression is either an atom of string, boolean, integer, or symbol (such
as \texttt{:example-symbol}), or a list of S-expressions.
For example, the S-expression \texttt{'("q")} will be represented in Idris, as
\dt{SExpList [StringAtom "q"]}.

Going back to the serialization problem, for the argument and return types or
our \Elab\ editor actions, we should specify how they should be converted to
\ty{SExp}s.  Defining certain functions for a given type sounds like the
perfect task for Idris interfaces\footnote{Haskell's type classes are called
interfaces in Idris, and type class instances are called implementations.}.

Therefore we define the interface \ty{Editorable} in Idris, which tells us
how to serialize a given type into an \ty{SExp} and how to deserialize it.
In \autoref{ssec:usingEditorable}, we will describe how implementations for
the \ty{Editorable} interface will be used by the compiler during the
communication between the editor and the compiler.

\begin{figure}[ht]
\caption{Definition of the \ty{Editorable} interface.}
\label{code:editorable}
\begin{Verbatim}[framesep=2mm, label=\footnotesize{\normalfont{Idris}}, labelposition=topline]
\kw{interface} \ty{Editorable} \bn{a} \kw{where}
  \fn{fromEditor} : \ty{SExp} -> \ty{Elab} \bn{a}
  \fn{toEditor} : \bn{a} -> \ty{Elab} \ty{SExp}
\end{Verbatim}
\end{figure}

We have two functions \fn{fromEditor} and \fn{toEditor} that must be defined
for a type that we want to make serializable.
\fn{fromEditor} allows us to deserialize a given S-expression into a value of the type \bn{a}.
\fn{toEditor} allows us to serialize a value of the type \bn{a} into an S-expression.
Notice that both of these functions return a value in the \Elab\ monad.
There are a few reasons for this:
\begin{itemize}
  \item The \Elab\ monad captures failure. Especially the \fn{fromEditor}
    function, which basically parses S-expressions, we want to fail if we are
    given an S-expression that is ill-formed for the type \bn{a}.
  \item The \Elab\ monad has an implementation for the \ty{Alternative}
    interface, which allows us to recover from failure if we need to.
  \item When an \Elab\ action fail with an explicit use of
    \texttt{\fn{fail} : \ty{List ErrorReporPart} -> \ty{Elab} \bn{a}},
    the user can give a detailed account of why it failed through the pretty
    printed errors. This is useful for giving graceful error messages to the
    user when a custom editor action fails.
  \item Both for serialization and deserialization, users may want to limit it
    to some of the constructors of a type. Being able to fail for certain
    constructors gives the users flexibility.
  \item Why we specifically need \Elab\ instead of any other monad that can
    fail, is that users may want to type check or normalize terms, or look up
    information about existing types and functions during serialization and
    deserialization as well. This is only available in the \Elab\ monad.
\end{itemize}

\subsection{\ty{Editorable} implementations in Idris}

Now that we have justified why putting \fn{fromEditor} and \fn{toEditor}
in the \Elab\ monad is necessary,
let's see how we can define an \ty{Editorable} implementation for \ty{String}.

\begin{figure}[ht]
\caption{\ty{Editorable} implementation for the type \ty{String}.}
\begin{Verbatim}[framesep=2mm, label=\footnotesize{\normalfont{Idris}}, labelposition=topline]
\kw{implementation} \ty{Editorable} \ty{String} \kw{where}
  \fn{fromEditor} (\dt{StringAtom} \bn{s}) = \fn{pure} \bn{s}
  \fn{fromEditor} \bn{x} =
    \fn{fail} \dt{[}\dt{TextPart} (\dt{"Can't parse the SExp "} \fn{++} \fn{show} \bn{x} \fn{++} \dt{" as a String"})\dt{]}
  \fn{toEditor} = \fn{pure} \fn{.} \dt{StringAtom}
\end{Verbatim}
\end{figure}

The definition of \fn{fromEditor} above tells us that if an S-expression is a
string atom, then we know how to get a \ty{String} from that, otherwise we
fail.  On the other hand, \fn{toEditor} never fails; if we have a \ty{String}
we can always construct a \dt{StringAtom} and put it in the \Elab\ monad.

Now, for a more complex example, let's look at how we can make \ty{List}s \ty{Editorable}.

\begin{figure}[ht]
  \caption{\ty{Editorable} implementation for the \ty{List} type.}
\begin{Verbatim}[framesep=2mm, label=\footnotesize{\normalfont{Idris}}, labelposition=topline]
\kw{implementation} \ty{Editorable} \bn{a} => \ty{Editorable} (\ty{List} \bn{a}) \kw{where}
  \fn{fromEditor} (\dt{SExpList} \bn{xs}) = \fn{traverse} \fn{fromEditor} \bn{xs}
  \fn{fromEditor} \bn{x} =
    \fn{fail} \dt{[}\dt{TextPart} (\dt{"Can't parse the SExp "} \fn{++} \fn{show} \bn{x} \fn{++} \dt{" as a List"})\dt{]}
  \fn{toEditor} \bn{xs} = \dt{SExpList} \fn{<$>} \fn{traverse} \fn{toEditor} \bn{xs}
\end{Verbatim}
\end{figure}

The definition of \fn{fromEditor} above tells us that if we have a list
S-expression, which holds an Idris list \bn{xs} of non-deserialized
S-expressions inside, we can apply \fn{fromEditor} to each of the S-expressions
in \bn{xs} and deserialize them all, and if they all succeed we can make a list
from their results and return that.
Similarly, the definition of \fn{toEditor} above tells us that if we have an
actual list \bn{xs}, we can \fn{traverse} the list to serialize them all, and
if they all succeed, we can make an S-expression out of the serialized
elements.

Now, let's look \autoref{code:ttnameEditorable} to see how we can serialize and
deserialize one of the types that we expect to use the most in our editor
actions, the type of names, \ty{TTName}.

\begin{figure}[ht]
\caption{\ty{Editorable} implementation for the type \ty{TTName}.}
\label{code:ttnameEditorable}
\begin{Verbatim}[framesep=2mm, label=\footnotesize{\normalfont{Idris}}, labelposition=topline]
\kw{implementation} \ty{Editorable} \ty{TTName} \kw{where}
  \fn{fromEditor} (\dt{StringAtom} \bn{s}) = \fn{namify} \bn{s}
    \kw{where}
      \fn{namify} : \ty{String} -> \ty{Elab} \ty{TTName}
      \fn{namify} \bn{s} =
        \kw{case} \fn{reverse} (\fn{map} \fn{pack} (\fn{splitOn} \dt{'.'} (\fn{unpack} \bn{s}))) \kw{of}
          \dt{[]} => \fn{fail} \dt{[TextPart "Empty string can't be a TTName"]}
          \dt{[}\bn{x}\dt{]} => \fn{pure} (\dt{UN} \bn{x})
          (\bn{x} \dt{::} \bn{xs}) => \bn{pure} (\dt{NS} (\dt{UN} \bn{x}) \bn{xs})
  \fn{fromEditor} \bn{x} =
    \fn{fail} \dt{[}\dt{TextPart} (\dt{"Can't parse the SExp "} \fn{++} \fn{show} \bn{x} \fn{++} \dt{" as a TTName"})\dt{]}

  \fn{toEditor} \bn{n} = \dt{StringAtom} \fn{<$>} \fn{stringify} \bn{n}
    \kw{where}
      \fn{stringify} : \ty{TTName} -> \ty{Elab} \ty{String}
      \fn{stringify} (\dt{UN} \bn{x}) = \fn{pure} \bn{x}
      \fn{stringify} (\dt{NS} \bn{x} \dt{[]}) = \fn{stringify} \bn{x}
      \fn{stringify} (\dt{NS} \bn{x} \bn{xs}) =
        \fn{pure} (\fn{concat} (\fn{intersperse} \dt{"."} (\fn{reverse} (\dt{""} \fn{::} \bn{xs})))
              \fn{++} !(\fn{stringify} \bn{x}))
      \fn{stringify} (\dt{MN} \bn{i} \bn{x}) = \fn{pure} (\dt{"__"} \fn{++} \bn{x} \fn{++} \fn{show} \bn{i})
      \fn{stringify} \bn{n'}@(\dt{SN} \bn{sn}) =
        \fn{fail} [ \dt{TextPart} \dt{"Don't know how to make"}
             , \dt{NamePart} \bn{n'}, \dt{TextPart} \dt{"into StringAtom"}]
\end{Verbatim}
\end{figure}

This is long code excerpt, and you do not have to follow the code entirely.
We will summarize the code below, but this excerpt should demonstrate that
we can define one of the most pivotal parts of serialization in Idris itself,
without writing any Haskell code.

Here is an overview of what the code in \autoref{code:ttnameEditorable} does:
When we want to deserialize a name, we require it to be a \dt{StringAtom}, and
then we split the string \bn{s} on the dots it contains. If there is no dot
inside, we make it a \ty{TTName} without a namespace using \dt{UN}, if there
are dots inside, we make it a \ty{TTName} with a namespace using \dt{NS}.
Similarly, if we have a \ty{TTName}, we check if we are given a namespace by
looking at the constructors of \ty{TTName}. If we are, then we make a
\ty{String} the name with dots in between, if we are not, then we merely make
\ty{String} from what we have.
For the machine generated names, for now, we make up a representation, that we
hope will not clash will other names.
We fail for the special names. Editor actions will ideally not use names with
\dt{MN} and \dt{SN}, so this is a temporary fallback solution.

Before we move on to the primitive \ty{Editorable} implementations,
we should point out that we have not made any function type \ty{Editorable}.
One can easily come up with a way to serialize pure and total functions with
finite domains by listing the results of the function for each of the inputs.
Alternatively, since editor actions are run interactively, we do have access to
the source code of functions, so one could explore serialization of functions
more seriously, but we will not entertain this thought in this thesis.

\subsection{Primitive \ty{Editorable} implementations}\label{ssec:primitiveEditorableDesign}

The existing elaborator reflection system are enough to define \ty{Editorable}
implementations for some of the most common types in Idris.
However, communication of Idris terms between the editor and the compiler is
not straightforward. The main reason for that there is a colossal gap between
the type of terms that our custom editor actions use, and the Idris terms we
write in our editor.
This gap is because Idris terms are written using the high-level surface
language, and \Elab\ actions deal with the core language terms. Therefore when
we send the code for a term from the editor to the compiler, we will send a
code in the surface syntax. Similarly, when the editor receives a piece of code
from the compiler, that also has to be in the surface syntax. This is because
when the editor takes some code from the file, or puts some code back to the
file, it does not know how to turn that into a core language term.
This conversion, also called elaboration as explained in
\autoref{sec:elabref}, can only be done in the Idris compiler.

If the editor can only deal with surface-syntax terms, and the \Elab\ actions
can only work with core language terms, the system we are designing must take
care of the conversion between the two.
Specifically, the \ty{Editorable} interface that we
\hyperref[code:editorable]{defined above} can be made responsible for this.

When the S-expression received by the compiler contains a string that is
supposed to be a piece of Idris code, \fn{fromEditor} should parse the string
into a surface-syntax code, and then elaborate that into a core language term.
Only after that can we run the \Elab\ editor action that we want to execute.

Similarly, when we finish running the \Elab\ action, \fn{toEditor} should
delaborate\footnote{Delaboration is the name in the Idris compiler for
converting core language code to surface syntax code. It is meant to be the
opposite of elaboration.} the core language code to a surface-syntax code, and
then it should pretty print it as a string. The resulting string can be sent
back from the compiler to the editor in an S-expression.

In \autoref{ssec:reflectedTypes}, we talked about the core language types that
are reflected in Idris.  We have already seen the definitions of the types
\ty{TT} and \ty{TTName}, and we have touched upon \ty{TyDecl}, \ty{DataDefn},
\ty{FunDefn}, and \ty{FunClause}.
We will not concern ourselves with \ty{Raw} right now, since solving the
problem for typed core language terms suffices to solve the same problem with
untyped one; we can always type-check a \Raw\ term into a \TT, or forget the
type of a \TT\ term into a \Raw\ one.

Notice that we have already given an \ty{Editorable} implementation of
\ty{TTName} \hyperref[code:ttnameEditorable]{above}, since names in the
surface syntax and core language are the same. There is no clever elaboration
for names.

However, for the other core language types, we should define \ty{Editorable}
implementations. As we discussed above, the editor can only send an receive
pieces of code in the surface syntax, which means we have to do parsing,
elaboration, delaboration and pretty printing in our \ty{Editorable}
implementation.
Unfortunately, it is not possible to write this implementation in Idris,
because Idris does not reflect the surface syntax, or parsing or pretty
printing with it.
Therefore, we will hard-code the implementations of \ty{TT}, \ty{TyDecl},
\ty{DataDefn}, \ty{FunDefn}, and \ty{FunClause} into the compiler.

To achieve this, we will have to extend the existing \Elab\ monad with
primitives that go through the steps we mentioned above. We will have to define
two primitives for each other those types, one for \fn{fromEditor} and one for
\fn{toEditor}. Instead of adding two primitives for each of those types, we can
only add two primitives and make them polymorphic, and add a constraint on what
types it can be used for. This would save us from adding new primitives to
\Elab\ every time we want to add a primitive \ty{Editorable} implementation.

What we mean by a constraint here is a predicate on types,
i.e. a \ty{Type}-indexed type family that describes which types can be used in
the primitives for \fn{fromEditor} and \fn{toEditor} we are about to define.

\begin{figure}[ht]
\caption{Definition of the \ty{HasEditorPrim} predicate in Idris.}
\begin{Verbatim}[framesep=2mm, label=\footnotesize{\normalfont{Idris}}, labelposition=topline]
\kw{data} \ty{HasEditorPrim} : \ty{Type} -> \ty{Type} \kw{where}
  \dt{HasTT}        : \ty{HasEditorPrim} \ty{TT}
  \dt{HasTyDecl}    : \ty{HasEditorPrim} \ty{TyDecl}
  \dt{HasDataDefn}  : \ty{HasEditorPrim} \ty{DataDefn}
  \dt{HasFunDefn}   : \ty{HasEditorPrim} (\ty{FunDefn} \ty{TT})
  \dt{HasFunClause} : \ty{HasEditorPrim} (\ty{FunClause} \ty{TT})
\end{Verbatim}
\end{figure}

Now we can use the predicate \ty{HasEditorPrim} to define our two new
\Elab\ primitives:

\begin{figure}[ht]
\caption{New \Elab\ primitives for serialization and deserialization that depend on \ty{HasEditorPrim}.}
\label{code:newElabPrims}
\begin{Verbatim}[framesep=2mm, label=\footnotesize{\normalfont{Idris}}, labelposition=topline]
\fn{prim__fromEditor} : \{\kw{auto} \bn{has} : \ty{HasEditorPrim} \bn{a}\} -> \ty{SExp} -> \ty{Elab} \bn{a}
\fn{prim__toEditor} : \{\kw{auto} \bn{has} : \ty{HasEditorPrim} \bn{a}\} -> \bn{a} -> \ty{Elab} \ty{SExp}
\end{Verbatim}
\end{figure}

Notice that their first argument is the predicate \ty{HasEditorPrim} applied to
the type \bn{a}, and it is an implicit argument that Idris should try to
automatically solve. This saves us from having to pass around a constructor
every time we want to use the primitive.

Using these two primitives, \ty{Editorable} implementation definitions for the
core language types all look alike:

\begin{figure}[ht]
\caption{\ty{Editorable} implementation for \TT, that depends on the new \Elab\ primitives.}
\begin{Verbatim}[framesep=2mm, label=\footnotesize{\normalfont{Idris}}, labelposition=topline]
\kw{implementation} \ty{Editorable} \ty{TT} \kw{where}
  \fn{fromEditor} \bn{x} = \fn{prim__fromEditor} \bn{x}
  \fn{toEditor} \bn{x} = \fn{prim__toEditor} \bn{x}
\end{Verbatim}
\end{figure}

We now know what primitive we want to have and we know what we want it to do.
We will discuss the implementation of this in the compiler in
\autoref{sec:primitiveEditorableImpl}.

\subsection{Using \ty{Editorable} in the compiler and the \kw{\%editor} modifier}\label{ssec:usingEditorable}

The motivation behind the \ty{Editorable} interface is twofold:
\begin{itemize}
\item to check if a given \Elab\ action is suitable to be used as an editor
  action.
\item to use the \fn{fromEditor} and \fn{toEditor} definitions to serialize
  and deserialize data when the action is run.
\end{itemize}

The first motivation means all argument and return types should have an
\ty{Editorable} implementation. If we want to write a toy editor action
\texttt{\fn{toy} : \ty{TTName} -> \ty{TT} -> \ty{Elab} \ty{TT}},
then the compiler must check if \ty{TTName}, \TT\ (from the second argument)
and \TT\ (from the return type) are all \Editorable. If any of the argument of
return types do not have an \Editorable\ implementation, then we should not be
able to run the given action from the editor. In fact, to make the distinction
clearer, we add a new function modifier syntax \kw{\%editor} to Idris.

\begin{figure}[ht]
\caption{Type declaration for the \fn{toy} editor action we want to define.}
\begin{Verbatim}[framesep=2mm, label=\footnotesize{\normalfont{Idris}}, labelposition=topline]
\kw{%editor}
\fn{toy} : \ty{TTName} -> \ty{TT} -> \ty{Elab} \ty{TT}
\end{Verbatim}
\end{figure}

When used before the type declaration of a function, Idris makes sure during
type-checking that this function's argument and return types all have an
\Editorable\ implementation. If they do not, then the user gets a type-error
that tells them that since they declared that \Elab\ action an editor action,
they must provide an \Editorable\ implementation for all argument and return
types, and the error tells them which one is missing.

The second motivation for \Editorable\ allows the compiler to communicate with
the editor via S-expressions. When the editor sends an S-expression like
\texttt{((:elab-edit "toy" '("q" "S Z") 35 5) 20)} to the
compiler with the intent of running an \Elab\ action,
the following steps should happen:
\begin{itemize}
  \item The S-expression we are given contains the name of the \Elab\ action,
    namely \fn{toy} We must lookup the name in order to resolve its namespace
    and also learn the type of the action.
  \item We check if \fn{toy} has been marked as an editor action using
    \kw{\%editor}. If not, we fail the editor action.
  \item We ``collect'' the components of the type of the \Elab\ action. For
    \fn{toy}, from the type signature \texttt{\ty{TTName} -> \ty{TT} ->
    \ty{Elab} \ty{TT}} we obtain the component list with three members: \ty{TTName}, \TT,
    and \Elab\ \TT.
    Since the last one is the return type for the action, we check the number
    of arguments we are given by the S-expression, and \texttt{'("q" "S Z")}
    has two elements, has exactly one fewer element than the component list we
    collected above.
  \item Observe that the argument members of the component list match the
    arguments list S-expression one to one. The argument components for
    \fn{toy} are \ty{TTName} and \TT, and we are given the S-expressions
    \texttt{"q"} and \texttt{"S Z"}. If we zip these lists, we get pairs of
    S-expressions and what type they will have when they are parsed.
    Parse them using the \fn{fromEditor} function for their respective
    \Editorable\ implementations.
  \item Then we will have a list of terms in the compiler that correspond to a
    \ty{TTName} and a \TT.  All we have done so far was to check if \fn{toy}
    was a valid editor action, and then to parse the inputs just so that we can
    run \fn{toy}. Run \fn{toy}.
  \item When we are done with that, we know we will get a term of the type \TT,
    because \fn{toy}'s type signature ends with \texttt{\Elab\ \TT}. That means,
    we can use \fn{toEditor} from the \Editorable\ implementation of \TT, which
    gives us an S-expression.
  \item We can now send that S-expression back to the editor.
\end{itemize}

\section{Examples in action}\label{sec:designExample}

\subsection{Successful example}

We now know how the compiler should work with the \Editorable\ interface.
To illustrate how it works in reality, let's finish the \fn{toy} editor action
we started above.

\begin{figure}[ht]
\caption{Implementation of the \fn{toy} action in Idris.}
\begin{Verbatim}[framesep=2mm, label=\footnotesize{\normalfont{Idris}}, labelposition=topline]
\kw{%editor}
\fn{toy} : \ty{TTName} -> \ty{TT} -> \ty{Elab} \ty{TT}
\fn{toy} \bn{n} \bn{t} = \kw{do} (_, _, \bn{ty}) <- \fn{lookupTyExact} \bn{n}
             \kw{case} \bn{ty} \kw{of}
               \kw{`(}\ty{Nat}\kw{)} => \fn{pure} \bn{t}
               _ => \fn{fail} \dt{[NamePart} \bn{n}, \dt{TextPart "is not a Nat!"]}
\end{Verbatim}
\end{figure}

Here is what we expect from this \Elab\ action: we will pass it two arguments,
the first one is a name of a hole, and the second one is a core language term.
If the type of the hole is \ty{Nat}, then we will return the term that we are
given, otherwise we will fail.

To be able to call this editor action from Emacs, we will have to write a bit
of Emacs Lisp:

\begin{figure}[ht]
\caption{Necessary Emacs Lisp code to run the \fn{toy} action.}
\begin{Verbatim}[framesep=2mm, label=\footnotesize{\normalfont{Emacs Lisp}}, labelposition=topline]
(\kw{defun} \fn{idris-elab-hole-arg} (\bn{action} \bn{args})
  \dt{"Run Elab action in editor, replace the hole with the result"}
  (\fn{interactive})
  (\kw{let} ((\bn{result} (\fn{car}
    (\fn{idris-eval}
      `(\dt{:elab-edit} ,\bn{action} ,\bn{args}
                   ,(\fn{idris-get-line-num}) ,(\fn{current-column}))))))
    (\fn{save-excursion}
      (\fn{apply} '\fn{delete-region} (\fn{idris-hole-start-end}))
      (\fn{insert} \bn{result}))))

(\kw{defun} \fn{idris-toy} ()
  \dt{"Run the toy Elab action in editor"}
  (\fn{interactive})
  (\kw{let} ((\bn{term} (\fn{read-string} \dt{"Enter a term:"})))
    (\fn{idris-elab-hole-arg} \dt{"toy"} `(,(\fn{idris-name-at-point}) ,\bn{term}))))
\end{Verbatim}
\end{figure}

The first function \fn{idris-elab-hole-arg} is a general function that can be
reused by other hole-based editor actions. It takes a name of an action and a
list, and sends them to the Idris executable, also adding the line and column
numbers. When it gets a response, it deletes the hole that the cursor is on,
and replaces it with the string/code that was just received from the compiler.

The second function \fn{idris-toy} is the little piece of code that runs the
\fn{toy} editor action we defined above. It asks the user to enter some code,
and when they do, it uses \fn{idris-elab-hole-arg} to send to Idris a message
that consists of the name under the cursor and that term the user just typed
in. Notice that the name under the cursor will be parsed as a \ty{TTName}, type
of the first argument of \fn{toy}, and the term we typed in will correspond to
\TT, type of the second argument of \fn{toy}.

Suppose we want to use this new editor action now.

\begin{figure}[ht]
\caption{Example \ty{Nat} declaration to run \fn{toy} on.}
\begin{Verbatim}[framesep=2mm, label=\footnotesize{\normalfont{Idris}}, labelposition=topline]
\fn{n} : \ty{Nat}
\fn{n} = \hole{q}
\end{Verbatim}
\end{figure}

We can place the cursor on \hole{q} and then run the \fn{idris-toy} function in
Emacs, possibly through a keyboard shortcut we assign to it.
By definition of \fn{idris-toy}, Emacs asks us to enter a term, suppose we
write \texttt{S Z} and press enter.
The following communication happens in the background with the compiler:

\begin{figure}[ht]
  \caption{Communication to run \fn{toy} on the hole \hole{q}.}
\begin{Verbatim}[framesep=2mm, label=\footnotesize{\normalfont{S-expression}}, labelposition=topline]
-> ((:elab-edit "toy" ("q" "S Z") 35 5)
<- (:return (:ok "1" nil) 20)
\end{Verbatim}
\end{figure}

We send the \Elab\ action name \fn{toy} and the list of arguments as an
S-expression. Idris sends back the result of the \Elab\ action when run with
the arguments \hole{q} as the name, and the core language term for \dt{S Z}.
Since the type of our hole is \ty{Nat} here, \fn{toy} just returns the same
term we give it, but it pretty prints it is as \dt{1}.

By definition of \fn{idris-elab-hole-arg}, the result \dt{1} replaces the hole
\hole{q} in the file. The resulting code is this:

\begin{figure}[ht]
\caption{End result of running \fn{toy} on the hole \hole{q}.}
\begin{Verbatim}[framesep=2mm, label=\footnotesize{\normalfont{Idris}}, labelposition=topline]
\fn{n} : \ty{Nat}
\fn{n} = \dt{1}
\end{Verbatim}
\end{figure}

\subsection{What \Editorable\ precludes}

Suppose we wanted to write a polymorphic editor action, that merely returned its argument back.

\begin{figure}[ht]
  \caption{A polymorphic editor action \fn{poly} in Idris that is supposed to fail.}
\begin{Verbatim}[framesep=2mm, label=\footnotesize{\normalfont{Idris}}, labelposition=topline]
\kw{%editor}
\fn{poly} : \bn{a} -> \ty{Elab} \bn{a}
\fn{poly} \bn{x} = \fn{pure} \bn{x}
\end{Verbatim}
\end{figure}

First of all, observe that the type declaration above actually is
\texttt{\{\bn{a} : \ty{Type}\} -> \bn{a} -> \ty{Elab} \bn{a}}, which means
there is an implicit argument in the beginning that has the type
\ty{Type}.\footnote{\ty{Type} is the type of types in Idris, similar to
\texttt{*} in Haskell and \texttt{Set} in Agda, but with a caveat:
Haskell's \texttt{*} has the type \texttt{*}\cite{eisenberg}, while Agda has universe
polymorphism and Idris has cumulativity. This means Agda and Idris manage to
avoid Girard's paradox, but Haskell does not even attempt that, ... not that
there's anything wrong with that.}

The problem with this is that, the editor has to tell the compiler what type of
result we want to get at the end. We do not yet have a way to communicate a
value of the type \ty{Type} between the editor and the compiler, i.e. we cannot
define an \Editorable\ implementation for \ty{Type} in Idris itself, because
Idris does not allow us to pattern match on types, since that breaks
parametricity and creates problems with type erasure.\cite{universePat}
We will not explore adding such an \Editorable\ implementation as a primitive.
We will instead disallow polymorphic editor actions.
When we try to compile \fn{poly}, we get the following error:

\begin{figure}[ht]
  \caption{Error message that shows why \fn{poly} fails.}
\begin{Verbatim}[framesep=2mm, label=\footnotesize{\normalfont{Idris error message}}, labelposition=topline]
You declared \fn{poly} to be an editor action, but there's no
\ty{Language.Reflection.Editor.Editorable} implementation for \ty{Type}
\end{Verbatim}
\end{figure}

Here is another custom editor action we want to preclude:

\begin{figure}[ht]
  \caption{A higher-order editor action \fn{funAction} in Idris that is supposed to fail.}
\begin{Verbatim}[framesep=2mm, label=\footnotesize{\normalfont{Idris}}, labelposition=topline]
\kw{%editor}
\fn{funAction} : (\ty{Nat} -> \ty{Nat}) -> \ty{Elab} \dt{()}
\fn{funAction} \bn{x} = \fn{pure} \dt{()}
\end{Verbatim}
\end{figure}

We do not have a way to serialize and deserialize functions of the type
\texttt{\ty{Nat} -> \ty{Nat}}, so therefore we should not allow \fn{funAction}
to be declared an editor action.
When we try to compile \fn{funAction}, we get the following error:

\begin{figure}[ht]
  \caption{Error message that shows why \fn{funAction} fails.}
\begin{Verbatim}[framesep=2mm, label=\footnotesize{\normalfont{Idris error message}}, labelposition=topline]
You declared \fn{funAction} to be an editor action, but there's no
\ty{Language.Reflection.Editor.Editorable} implementation for \ty{Nat} -> \ty{Nat}
\end{Verbatim}
\end{figure}

We have now seen the motivations behind the \Editorable\ interface and
why we defined it this way. We have also demonstrated that this definition
is useful for serialization and deserialization of the data used in editor
actions, and that for the unserializable kinds of data, such as functions,
\Editorable\ acts as a gatekeeper. In the next chapter, we will look at how
this design is implemented in the Idris compiler in Haskell.

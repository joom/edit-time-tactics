\section{Design}\label{sec:design}

% I haven't yet flowed you into the point
\TODO{Remember that we are trying to extend Idris's elaborator reflection
system to do ...}


We described in \autoref{sec:introduction} what a simple editor interaction
action like case splitting does.  What we want to do is to add another command
that would be recognized by our compiler, and specify how such a command would
run, and what arguments it should take to communicate with the editor
effectively.

\subsection{Communication}\label{ssec:communication}
\TODO{This section doesn't make such sense, I need at least two examples. I
need to address the challenges that make us generalize,  }

The current Idris\footnote{This thesis is using Idris 1.1.1.} implementation of
the editor interaction mode is a part of the Idris compiler, and is written in
Haskell. When the editor is running,
it runs the \path{idris} executable with the \path{--ide-mode} flag, which
allows socket communication with the program through a machine-readable
syntax.\footnote{The Idris mode of Vim works differently, since Vim did not
support asynchronous jobs until 8.0. This is expected to change in the near
future.}
To be more precise, this program receives S-expressions\cite{mccarthy} as input
over the socket and sends back S-expressions as output. Let's see an example of
this. Suppose we are writing a predecessor function on natural numbers and we
want to case split on the argument.

\begin{Verbatim}[framesep=2mm, label=\footnotesize{\normalfont{Idris}}, labelposition=topline]
\IdrisFunction{pred} : \IdrisType{Nat} -> \IdrisType{Nat}
\IdrisFunction{pred} \IdrisBound{x} = \IdrisMetavar{?\IdrisMetavar{p}}
\end{Verbatim}

When we put the cursor on \path{x} and run case splitting, the following
communication happens in the background.

\begin{Verbatim}[framesep=2mm, label=\footnotesize{\normalfont{S-expression}}, labelposition=topline]
-> ((:case-split 5 "x") 13)
<- (:return (:ok "pred Z = ?p_1{\textbackslash}npred (S k) = ?p_2{\textbackslash}n") 13)
\end{Verbatim}

\TODO{Revise the following paragraph: first describe the specific code as if you're reading the code to someone, then generalize it. A complete rewrite of this paragraph would be a good idea.}
The case split command requires two arguments, the line number and the variable
name to case split on. The line number is included in most editor actions that
change the code. The variable name is a string that hold the necessary data.
The number \path{13} that comes afterwards is how many rounds of communications
have been done so far. Notice that the response also carries the same number.
The response contains \path{:ok}, which means the case splitting succeeded,
and then a string that contains two lines split by the new line character.
The editor receives twose two lines and replaces that with the line we ran the
case split command on.

\TODO{The following paragraph is generalizing too early. It'd be nice to have another concrete example, then you can contrast the two examples to show the need for generalization}
The way the editor and the compiler communicate is
``stringly typed''\footnote{A term coined by Mark Simpson, mentioned in a
deleted Stack Overflow thread.\TODO{Explain the meaning}}. We do not have a way
to check if the arguments
to \path{:case-split} are of the right type. If that is the case, then we will
only know when we run that action.
We now want to create a single command that handles different kinds of
actions. These actions possibly take multiple arguments that hold different
kinds of values, such as names, type declarations, function definitions, or
expressions. The \Elab\ actions we will write are going to require those arguments
to be of the right ``type", and in the right order. When we are writing front
end code for your editor interaction mode, we have to take that into account.
We will not attempt to develop a way to detect errors about this before
runtime.

The resulting communication should look like this for case splitting.

\begin{Verbatim}[framesep=2mm, label=\footnotesize{\normalfont{S-expression}}, labelposition=topline]
-> ((:elab-action "caseSplit" '("pred" "pred x = ?p" "x")) 13)
<- (:return (:ok '("pred Z = ?p_1" "pred (S k) = ?p_2")) 13)
\end{Verbatim}

This does not look that different from the previous version, except now we are
calling a different command that runs the \Elab\ action \path{"caseSplit"}.

\subsection{Types of Editor Actions}\label{ssec:types}

\TODO{Be more specific, say "the Haskell implementation of the compiler that receives the message from Emacs"}
\TODO{"Users of our work will be writing a lot of functions like this. the problem with allowing users to do that is the following..."}
Now let's think of the other end of this editor action. What would the type of
such an \Elab\ action be? The communication we mentioned above should give a hint.

\begin{Verbatim}[framesep=2mm, label=\footnotesize{\normalfont{Idris}}, labelposition=topline]
\fn{caseSplit} : \ty{TTName} -> \ty{FunClause} -> \ty{TTName} -> \ty{Elab} (\ty{List} \ty{FunClause})
\end{Verbatim}

As a first cut, we can imagine an editor action that takes the name of the function
we are performing the case split on, the function clause we are splitting,
and the variable name that should occur in the aforementioned function clause,
and will be split by the editor action. As a result, our editor action should return
a list of function clauses, what the aforementioned function clause turns into
after the split.

\TODO{Show two examples of what works, show an example of what we want to preclude!}

The types in this editor action are reflection types. \TODO{WHAT ARE REFLECTION
TYPES, DUH} But what if we wanted to write
a editor action takes an argument of a non-reflection type, such as \ty{Nat}, then
how would the compiler know how to pass the editor action the right type of arguments?
More importantly, can we check in compile time if the editor knows how to pass
arguments to the editor action in question? An editor action should only deal
with the types that can be sent from and to the editor.
In other words, we want to have a subset of types, and to check in compile time
if the types in the type signature of the editor action are all in this subset.

If we were using Haskell, we could define a type family, and only have instances
for the types we want to allow. A type family in Haskell is a type-level
function, one might think since we are dealing with dependent types we can
already write type-level functions! However Idris does not allow us to pattern
match on types, because that breaks parametricity and type erasure.\cite{universePat}
We could opt for an alternative called the universe pattern, however we should
save the user the trouble of learning the universe we define. Moreover, any
universe that we will defined cannot be extended by the user. If the user later
wants to define a new type of their own and to use that in the editor action,
they should be able to do that.

The solution to our problem is to have an interface.
We can define a new interface \ty{Editorable}, which only has implementations
that we know how to communicate with the editor. For the user defined types,
the user is free to give an implementation for the \ty{Editorable} interface of
that type. Then what should such an interface consist of?
The way that we communicate with the editor right now is through S-expressions,
as we described earlier. Therefore we can require our types to have two
methods, one from the type to an S-expression, and one to go from an
S-expression to the type. In other words, one to serialize the type, and one
to the deserialize it. Let's call them \fn{toEditor} and \fn{fromEditor}, respectively.

\begin{Verbatim}[framesep=2mm, label=\footnotesize{\normalfont{Idris}}, labelposition=topline]
\kw{interface} \ty{Editorable} \bn{a} \kw{where}
  \fn{fromEditor} : \ty{SExp} -> \ty{Maybe} a
  \fn{toEditor} : \bn{a} -> \ty{SExp}
\end{Verbatim}

Then we would have implementations of this interface for the reflection types.

\TODO{the crash thing is confusing, maybe postulate the type class implementation?}

\begin{Verbatim}[framesep=2mm, label=\footnotesize{\normalfont{Idris}}, labelposition=topline]
\IdrisFunction{err} : \IdrisType{String}
\IdrisFunction{err} = \IdrisData{"Primitive instance of the Editorable interface"}

\kw{implementation} \ty{Editorable} \ty{TTName} \kw{where}
  \fn{fromEditor} = \fn{idris_crash} \fn{err}
  \fn{toEditor}   = \fn{idris_crash} \fn{err}
\end{Verbatim}

The implementations for \ty{TT}, \ty{TyDecl}, \ty{DataDefn}, \ty{FunDefn} and
\ty{FunClause} will also be the same as this.
The content that these types hold come to the compiler as a string and parsed
as a part of an abstract syntax tree. Therefore their conversions from and to
S-expressions should be handled by the compiler, because we do not have a way
to parse strings into syntax trees within Idris.

However, we are now free to define other instances of \ty{Editorable} that
might be useful, such as a \String\ instance and a \ty{List} instance.

\begin{Verbatim}[framesep=2mm, label=\footnotesize{\normalfont{Idris}}, labelposition=topline]
\kw{implementation} \ty{Editorable} \ty{String} \kw{where}
  \fn{fromEditor} (\dt{StringAtom} \bn{s}) = \dt{Just} \bn{s}
  \fn{fromEditor} _ = \dt{Nothing}
  \fn{toEditor} = \dt{StringAtom}
\end{Verbatim}

If we have an S-expression primitive string, then we can convert that into
an Idris \String. If we have a different kind of S-expression, we cannot.
Conversely, if we want to convert an Idris \String\ into an S-expression, we
just make it into a string atom. Other primitive instances like \ty{Integer},
\ty{Bool} etc.\ follow the same reasoning.

\begin{Verbatim}[framesep=2mm, label=\footnotesize{\normalfont{Idris}}, labelposition=topline]
\kw{implementation} \ty{Editorable} \bn{a} => \ty{Editorable} (\ty{List} \bn{a}) \kw{where}
  \fn{fromEditor} (SExpList xs) = sequence (map fromEditor xs)
  \fn{fromEditor} _ = \dt{Nothing}
  \fn{toEditor} \bn{xs} = \dt{SExpList} (\fn{map} \fn{toEditor} \bn{xs})
\end{Verbatim}

If we have an S-expression list, then we try to parse everything inside that
into S-expressions. If they all succeed, then we can parse the entire
S-expression into an Idris \ty{List}. If we do not have an S-expression list,
then we cannot parse the S-expression into an Idris \ty{List}.  When we are
converting a list into an S-expression, it is enough to convert everything in
the list to an S-expression separately, and then collect the separate
S-expressions inside a list S-expression. Other instances for the \ty{Pair},
\ty{Maybe}, \ty{Either} types follow the same reasoning.

We now have an interface that limits the types we can use in an editor action.
However, how do we make sure that the compiler checks this limitation.
There is no readable way to express this in the type level, therefore we will
invent a new compiler directive. Now we can declare a function to be an editor
action with the \texttt{\IdrisKeyword{\%editor}} keyword. Let's go back to the
case split action.

\TODO{Explain that this is new syntax!}

\begin{Verbatim}[framesep=2mm, label=\footnotesize{\normalfont{Idris}}, labelposition=topline]
\kw{%editor}
\fn{caseSplit} : \ty{TTName} -> \ty{FunClause} -> \ty{TTName} -> \ty{Elab} (\ty{List} \ty{FunClause})
\end{Verbatim}

Once we complete the definition, this will pass our type checker because
\ty{TTName}, \ty{FunClause} and \ty{List FunClause} all have implementations of
the \ty{Editorable} interface.

If we use a type that did not have an \ty{Editorable} implementation, then
we should get a type error.

\subsection{Connecting Types and Communication}\label{ssec:typesAndCommunication}

We now have defined a way to type check editor actions to see we can call them
from the editor in a sensible way. Let's juxtapose the type of \fn{caseSplit} and a communication that uses it and see how the arguments should be converted.
\begin{Verbatim}[framesep=2mm, label=\footnotesize{\normalfont{Idris}}, labelposition=topline]
\kw{%editor}
\fn{caseSplit} : \ty{TTName} -> \ty{FunClause} -> \ty{TTName} -> \ty{Elab} (\ty{List} \ty{FunClause})
\end{Verbatim}
\begin{Verbatim}[framesep=2mm, label=\footnotesize{\normalfont{S-expression}}, labelposition=topline]
-> ((:elab-action "caseSplit" '("pred" "pred x = ?p" "x")) 13)
<- (:return (:ok '("pred Z = ?p_1" "pred (S k) = ?p_2")) 13)
\end{Verbatim}

The first and thirds arguments to \fn{caseSplit} are of type \ty{TTName}. In
the S-expression, they are merely strings, but when they are passed to the
compiler we first have to resolve their namespaces and represent them as
\path{Name}s in the compiler, and then reflect them to
\path{Term}s\footnote{The type in the Haskell implementation of the compiler
that represents terms in the core language TT.} in the compiler
that would be of the type \ty{TTName} if they were reified.
Only then we can apply them to the \fn{caseSplit} action.

The second argument is of type \ty{FunClause}. Now we have actual code passed
as a string to the compiler. That string will then be parsed into a
\path{PClause}\footnote{The type in the Haskell implementation of the compiler
that represents surface language function clauses.} in the compiler, which will
then be elaborated into the core language and later represented as a
\path{RFunClause}\footnote{The type in the Haskell implementation of the
compiler that represents the reflected core language function clauses.} in the
compiler. Finally it will be reflected to a \path{Term} in the compiler that
would be of the type \ty{FunClause}. Then we can apply it to the
\fn{caseSplit} action.

When we apply the arguments to the \fn{caseSplit} action, we can run the
elaborator and get the term that it generates. In this case, we will get
a \path{Term} that will represent a list of \ty{FunClause}s.
We will first have to reify the \path{Term} that represents the list into an
actual Haskell list of \path{Term}s. Then we will first have to reify each of
them into \path{RFunClause}s in the compiler. The next step is to delaborate
each of them into \path{PClause}s in the compiler. Only then we can use the
pretty printer to convert them into strings that contain code.
Now we have a list of strings, which we can represent as an S-expression and
send back to the editor.

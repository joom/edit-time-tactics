\chapter{Implementation}\label{chap:implementation}

\section{Additions to the Idris standard library}\label{sec:stdlib}

\newcommand{\LR}{\texttt{Language.Reflection}}
\newcommand{\LRE}{\texttt{Language.Reflection.Elab}}
\newcommand{\LREd}{\texttt{Language.Reflection.Editor}}

Out of the types we skimmed through in \autoref{ssec:reflectedTypes}, \TT,
\Raw, and \ty{TTName} reside in the \texttt{Language.Reflection} module, and
\ty{TyDecl}, \ty{DataDefn}, and \ty{FunDefn} live in \LRE. This is because
Idris only has quotation of terms, as we reviewed in \autoref{ssec:quotation},
so the definitions we will need in quotations are in \LR. The ones that are not
needed by quotation but are needed for elaborator reflection are in \LRE.

Out of the types we defined in \autoref{sec:types}, \ty{SExp} should live in
\LRE\ since \Elab\ will depend on it. However, the new interface
\Editorable\ should live in \LREd.

Remember from \autoref{code:newElabPrims} that we added two new
\Elab\ primitives to our language, namely \fn{prim\_\_fromEditor} and
\fn{prim\_\_toEditor}.
The way \Elab\ primitives work in Idris is that there is a constructor in the
\Elab\ data type for each of them. Therefore we need to add those constructors
and then define \fn{prim\_\_fromEditor} and \fn{prim\_\_toEditor} in terms of
them.

\begin{figure}[ht]
\caption{New constructors for the new \Elab\ primitives, and function definitions based on them.}
  \label{code:newElabConstructors}
\begin{Verbatim}[framesep=2mm, label=\footnotesize{\normalfont{Idris}}, labelposition=topline]
\kw{export}
\kw{data} \ty{Elab} : \ty{Type} -> \ty{Type} \kw{where}
  \cm{-- the constructors we had before}
  \dt{Prim__FromEditor} : \{\bn{a} : \ty{Type}\} -> \ty{HasEditorPrim} \bn{a} -> \ty{SExp} -> \ty{Elab} \bn{a}
  \dt{Prim__ToEditor} : \{\bn{a} : \ty{Type}\} -> \ty{HasEditorPrim} \bn{a} -> \bn{a} -> \ty{Elab} \ty{SExp}

\kw{export}
\fn{prim__fromEditor} : \{\kw{auto} \bn{has} : \ty{HasEditorPrim} \bn{a}\} -> \ty{SExp} -> \ty{Elab} \bn{a}
\fn{prim__fromEditor} \{\bn{has} = \bn{x}\} \bn{sexp} = \dt{Prim__FromEditor} \bn{x} \bn{sexp}

\kw{export}
\fn{prim__toEditor} : \{\kw{auto} \bn{has} : \ty{HasEditorPrim} \bn{a}\} -> \bn{a} -> \ty{Elab} \ty{SExp}
\fn{prim__toEditor} \{\bn{has} = \bn{x}\} \bn{y} = \dt{Prim__ToEditor} \bn{x} \bn{y}
\end{Verbatim}
\end{figure}

We make the additions in \autoref{code:newElabConstructors} in \LRE.  Notice
that \kw{export} for \Elab\ does not export the constructors for the module,
and we do not want to export all the primitive actions without any limits; we
want to make \Elab\ more abstract. For that purpose, we define
\fn{prim\_\_fromEditor} and \fn{prim\_\_toEditor} that we actually can export.
Compared to the types of the constructors we defined, types of the functions we
defined are made more user-friendly by making the \ty{HasEditorPrim} argument
automatically inferrable by proof search, which is a helpful feature of Idris
usable via the \kw{auto} keyword.

We will inspect in \label{sec:primitiveEditorableImpl} how these new
constructors will be used in the compiler, but first we have to clarify some
terminology.

\newpage
\section{Reflection and reification in the compiler}

Before we delve into further details of the implementation, we should compare
and contrast certain terminology we will use in this chapter. The graph in
\autoref{reflectionGraph} describes the relationship between the different
kinds of languages and representations and the spells out the specific names
for moving from one to another.

\begin{figure}[ht]
\caption{The relationship between reflection, reification, quotation,
  unquotation, elaboration and delaboration.}
\label{reflectionGraph}
\newcommand\mlnode[1]{
  \begin{minipage}{3.3cm}
    \linespread{1}\selectfont \begin{center}\small #1 \end{center}
  \end{minipage}}
\begin{tikzcd}
  \mlnode{
    Haskell terms\\ \medskip \footnotesize such as the \ty{()} term\\
    \dt{()}
  }
  \arrow[rrr, "\text{\small reflection}" description, bend right=10]
  &  &  &
  \mlnode{
    Haskell representation of Idris core language terms\\ \medskip \scriptsize
    such as the \ty{Term}\footnotemark term\\
    \texttt{\dt{P} (\dt{DCon} \dt{0} \dt{0} \dt{False}) (\dt{UN} \dt{"MkUnit"}) (\dt{P} (\dt{TCon} \dt{0} \dt{0 \dt{False}}) (\dt{UN} \dt{"Unit"}) \dt{Erased})}
  }
  \arrow[lll, "\text{\small reification}" description, bend right=10] \\
  &  &  &\\
  \mlnode{
    Idris terms\\ \medskip \footnotesize such as the \ty{()} term\\
    \dt{()}
  }
  \arrow[r, "\text{\small elaboration}" description, bend right=50]
  &
  \mlnode{
    Idris core terms\\ \medskip \footnotesize such as the \ty{Unit} term\\
    \dt{MkUnit}
  }
  \arrow[l, "\text{\small delaboration}" description, bend right=50]
  \arrow[rr, "\text{\small quotation}" description, bend right=35]
  \arrow[uurr, "\text{\small \emph{internally represented as}}" description, dashed, bend left=25]
  & &
  \mlnode{
    Idris representation of the Haskell representation of Idris core language terms\\
    \medskip \scriptsize such as the \TT\ term\\
    \texttt{\dt{P} (\dt{DCon} \dt{0} \dt{0}) (\dt{UN} \dt{"MkUnit"})\\
    (\dt{P} (\dt{TCon} \dt{0} \dt{0}) (\dt{UN} \dt{"Unit"}) \dt{Erased})}
  }
  \arrow[ll, "\text{\small unquotation}" description, bend right=35]
  \arrow[uu, leftrightarrow, "\text{\small \emph{correspondence}}" description, dashed]
\end{tikzcd}
\end{figure}

\footnotetext{In the compiler source code in Haskell, both \ty{Term} and
  \ty{Type} are aliases for \ty{TT Name}. This should not be confused with the
  type of reflected names, \ty{TTName}, defined in Idris. In the compiler, the
  type of typed core language terms, \TT, is indexed by the type of names in
  that term. The type of names in the compiler is called \ty{Name}, therefore
  \texttt{TT Name} in the compiler is the type of core language terms with the
  usual names. We will stick with \ty{Term} and \ty{Type} in this thesis to
  avoid this confusion.}
In the previous sections, we have informally used the word ``reflection'' to
refer to the act of providing an Idris representation for a Haskell type we use
in the compiler.  That was handwavy but still accurate: In the graph we defined
reflection as moving from a Haskell term to the Haskell representation of an
Idris term. Since this holds for any Haskell term, you can also take a Haskell
term that is itself an internal Haskell representation of some Idris syntax.
When you reflect that, you get the internal representation of an Idris
representation of the Haskell representation of an Idris term.  Our previous
usage of ``reflection'' was only a special case of the word, but in this
section, we will use the generalized definition.

Let's give an example and see what we mean by this.
We defined the Idris type \ty{SExp} in \autoref{code:definitionSExp}.
When we will define \fn{fromEditor} and \fn{toEditor} in \autoref{sec:primitiveEditorableImpl},
we will need to convert Haskell terms of the Haskell type \ty{SExp} into the
internal representation of an Idris term of the Idris type \ty{SExp}, and vice
versa.

The conversion from the former to the latter is reflection. We write the
function \fn{reflectSExp} in \autoref{code:reflectSExp}.

\begin{figure}[ht]
\caption{The function to reflect Haskell terms of the type \ty{SExp} to the
internal representations of Idris terms of the Idris type \ty{SExp}.}
\label{code:reflectSExp}
\begin{Verbatim}[framesep=2mm, label=\footnotesize{\normalfont{Haskell}}, labelposition=topline]
\fn{reflectSExp} :: \ty{SExp} -> \ty{Raw}
\fn{reflectSExp} (\dt{StringAtom} \bn{s}) =
  \dt{RApp} (\dt{Var} (\fn{tacN} \dt{"StringAtom"})) (\dt{RConstant} (\dt{Str} \bn{s}))
\fn{reflectSExp} (\dt{SymbolAtom} \bn{s}) =
  \dt{RApp} (\dt{Var} (\fn{tacN} \dt{"SymbolAtom"})) (\dt{RConstant} (\dt{Str} \bn{s}))
\fn{reflectSExp} (\dt{BoolAtom} \bn{b}) =
  \dt{RApp} (\dt{Var} (\fn{tacN} \dt{"BoolAtom"})) (\fn{reflectBool} \bn{b})
\fn{reflectSExp} (\dt{IntegerAtom} \bn{i}) =
  \dt{RApp} (\dt{Var} (\fn{tacN} \dt{"IntegerAtom"})) (\dt{RConstant} (\dt{BI} \bn{i}))
\fn{reflectSExp} (\dt{SexpList} \bn{l}) =
  \dt{RApp} (\dt{Var} (\fn{tacN} \dt{"SExpList"}))
       (\fn{reflectList} (\dt{Var} (\bn{tacN} \dt{"SExp"})) (\fn{map} \fn{reflectSExp} \bn{l}))
\end{Verbatim}
\end{figure}

Observe that in the \fn{reflectSExp}, we are returning a \Raw\ term, a core
language term that is untyped. This mainly for convenience: it is easier to
write the syntax trees for untyped terms. We can always type-check them later.

To get a Haskell term of the type \ty{SExp} from an internal representation of
an Idris term that has the Idris type \ty{SExp}, we have to write a function
\fn{reifySExp} that reifies a given \ty{Term}, which you can see in
\autoref{code:reifySExp}

\begin{figure}[ht]
\caption{The function to reify the internal representations of Idris terms
  of the Idris type \ty{SExp} to Haskell terms of the type \ty{SExp}.}
\label{code:reifySExp}
\begin{Verbatim}[framesep=2mm, label=\footnotesize{\normalfont{Haskell}}, labelposition=topline]
\fn{reifySExp} :: \ty{Term} -> \ty{ElabD} \ty{SExp}
\fn{reifySExp} (\dt{App} _ (\dt{P} _ \bn{n} _) \bn{x})
  | \bn{n} \fn{==} \fn{tacN} \dt{"StringAtom"}  = \dt{StringAtom}  \fn{<$>} \fn{reifyString} \bn{x}
  | \bn{n} \fn{==} \fn{tacN} \dt{"SymbolAtom"}  = \dt{SymbolAtom}  \fn{<$>} \fn{reifyString} \bn{x}
  | \bn{n} \fn{==} \fn{tacN} \dt{"BoolAtom"}    = \dt{BoolAtom}    \fn{<$>} \fn{reifyBool} \bn{x}
  | \bn{n} \fn{==} \fn{tacN} \dt{"IntegerAtom"} = \dt{IntegerAtom} \fn{<$>} \fn{reifyInteger} \bn{x}
  | \bn{n} \fn{==} \fn{tacN} \dt{"SExpList"}    = \dt{SexpList}    \fn{<$>} \fn{reifyList} \fn{reifySExp} \bn{x}
\fn{reifySExp} \bn{tm} = \fn{fail} (\dt{"Not an SExp: "} \fn{++} \fn{show} \bn{tm})
\end{Verbatim}
\end{figure}

Observe in \fn{reifySExp} that we are returning
\ty{ElabD SExp}.\footnote{\ty{ElabD} is an alias for the \ty{Elab'} monad
combined with some special state, this does not concern us in this thesis.}
This is because reification, unlike reflection, can fail. This function is
designed to reify the internal representation of S-expressions. If the
\ty{Term} it receives does not correspond to one, reification should fail.

Now that we know what reflection and reification mean in the context of compiler
development, we can move on to how they are used in implementing primitive
implementations for \ty{Editorable}.

\section{Primitive \ty{Editorable} implementations}\label{sec:primitiveEditorableImpl}

In \autoref{ssec:primitiveEditorableDesign} we discussed why core language
types like \ty{TT}, \ty{TyDecl}, \ty{DataDefn}, \ty{FunDefn}, and
\ty{FunClause} must have primitive implementations of the
\Editorable\ interface. Recall that for each of these types, the editor can
only send a piece of code to the compiler as a string which contains code in
the surface language, however these types are in the core language. There are
many steps in between that we are missing, such as parsing, elaboration,
delaboration and pretty printing.

Now let's see how we implemented this in the Idris compiler. Tactic evaluation
in the compiler happens in \fn{runElabAction}, which is defined in the compiler
source code. Inside that, there is a helper function \fn{runTacTm} that takes a
typed term corresponding to a value in the \Elab\ monad of Idris, which should
correspond to a \Elab\ primitive constructor applied to arguments, i.e. when
the \Elab\ action term is in a normal form. Then we can check which primitive
constructor we are dealing with and behave accordingly.

\begin{figure}[ht]
\caption{The \fn{runElabAction} function in the compiler, how \Elab\ action terms are run under the hood.}
\label{code:runElabAction}
\begin{Verbatim}[framesep=2mm, label=\footnotesize{\normalfont{Haskell}}, labelposition=topline]
\fn{runElabAction} :: \ty{ElabInfo} -> \ty{IState} -> \ty{FC} -> \ty{Env}
              -> \ty{Term} -> [\ty{String}] -> \ty{ElabD} \ty{Term}
\fn{runElabAction} \bn{info} \bn{ist} \bn{fc} \bn{env} \bn{tm} \bn{ns} = \kw{do} \bn{tm}' <- \fn{eval} \bn{tm}
                                         \fn{runTacTm} \bn{tm}'

  \cm{-- some helper functions ...}

  \fn{runTacTm} :: \ty{Term} -> \ty{ElabD} \ty{Term}
  \fn{runTacTm} \bn{tac}@(\fn{unApply} -> (\dt{P} _ \bn{n} _, \bn{args}))
    | \bn{n} \fn{==} \fn{tacN} \dt{"Prim__Solve"}
    = \kw{do} ~[] <- \fn{tacTmArgs} \dt{0} \bn{tac} \bn{args}
          \fn{solve}
          \fn{returnUnit}
    \cm{-- other cases for other constructors}
\end{Verbatim}
\end{figure}

In \autoref{code:runElabAction} we can observe how \fn{runElabAction} is
defined in the compiler.
It takes many arguments, such as information for the elaborator, the internal
state of the compiler at that point, the source location of the action, the
environment under which the \Elab\ term should be evaluated to a normal form, the
term that represents the \Elab\ action term, and a namespace.  For our purposes,
we do not have to worry about all of these.

In the definition of \fn{runElabAction} we evaluate \bn{tm} to a normal form
of the \Elab\ action, and then we check in \fn{runTacTm} which form we want run.

If we have a non-neutral normal form of the \Elab\ action, that means when we decide
that what we have is a global variable term, which is a constructor, represented by
\dt{P}, or it is a series of applications to a variable term that is a
constructor. We can handle both situations with the
\texttt{\fn{unApply} :: \ty{Term} -> (\ty{Term}, [\ty{Term}])} function,
which dissects multiple curried function applications into the the term
that should be a function and the list of argument terms passed to
it.\footnote{A term that is not a function application will be treated as a
function application with no arguments passed.}

\autoref{code:runElabAction} demonstrates how an example \Elab\ action
term is treated by \fn{runTacTm}. For the primitive version of the \fn{solve}
action, which is represented by the \dt{Prim\_\_Solve} constructor in \Elab,
we first make sure that there are no given arguments, and then run
\texttt{\fn{solve} :: \ty{Elab'} \bn{aux} \ty{()}} in the internal elaborator
monad. If these all succeed, we return the reflected unit term back.

Now, in order to add more \Elab\ primitives, we have to implement similar cases
for \dt{Prim\_\_FromEditor} and \dt{Prim\_\_ToEditor} in \fn{runTacTm}, which
we started to do in \autoref{code:runElabAction}.

\begin{figure}[ht]
\caption{Adding cases for \dt{Prim\_\_FromEditor} and \dt{Prim\_\_ToEditor} in \fn{runTacTm} in the compiler.}
\label{code:runElabAction}
\begin{Verbatim}[framesep=2mm, label=\footnotesize{\normalfont{Haskell}}, labelposition=topline]
\fn{runTacTm} :: \ty{Term} -> \ty{ElabD} \ty{Term}
\fn{runTacTm} \bn{tac}@(\fn{unApply} -> (\dt{P} _ \bn{n} _, \bn{args}))
  \cm{-- other cases for other Elab primitives}
  | \bn{n} \fn{==} \bn{tacN} \dt{"Prim__FromEditor"}
  = \kw{do} ~[\bn{ty}, \bn{hasEditorPrim}, \bn{arg}] <- \fn{tacTmArgs} \dt{3} \bn{tac} \bn{args}
        \bn{ty'} <- \fn{eval} \bn{ty}
        \bn{arg'} <- \fn{eval} \bn{arg}
        \kw{case} \bn{ty'} \kw{of}
          \dt{P} _ \bn{tyN} _ | \bn{tyN} \fn{==} \fn{reflm} \dt{"TT"} ->
            \cm{-- now we want to convert an S-expression to TT}
          \dt{P} _ \bn{tyN} _ | \bn{tyN} \fn{==} \fn{tacN} \dt{"TyDecl"} ->
            \cm{-- now we want to convert an S-expression to TyDecl}
          \cm{-- other cases for other core language types}
  | \bn{n} \fn{==} \bn{tacN} \dt{"Prim__ToEditor"}
  = \kw{do} ~[\bn{ty}, \bn{hasEditorPrim}, \bn{arg}] <- \fn{tacTmArgs} \dt{3} \bn{tac} \bn{args}
        \bn{ty'} <- \fn{eval} \bn{ty}
        \bn{arg'} <- \fn{eval} \bn{arg}
        \kw{case} \bn{ty'} \kw{of}
          \dt{P} _ \bn{tyN} _ | \bn{tyN} \fn{==} \fn{reflm} \dt{"TT"} ->
            \cm{-- now we want to convert a TT into an S-expression}
          \dt{P} _ \bn{tyN} _ | \bn{tyN} \fn{==} \fn{tacN} \dt{"TyDecl"} ->
            \cm{-- now we want to convert a TyDecl into an S-expression}
          \cm{-- other cases for other core language types}
\end{Verbatim}
\end{figure}

Observe that we left most parts in here blank, let's zoom in to the \TT\ cases
  of \dt{Prim\_\_FromEditor} and \dt{Prim\_\_toEditor}. The actual code blocks
  for them refer to many helper functions that are difficult to follow; we will
  instead explain what they do in detailed prose.

For the \TT\ case of \dt{Prim\_\_FromEditor}, we have to do the following steps:
\begin{enumerate}
  \item We have an Idris term \bn{arg'} representing the S-expression passed to
    the \dt{Prim\_\_FromEditor}, we have to reify this using \fn{reifySExp} and
    get a Haskell term with the Haskell type \ty{SExp}.
  \item Remember from \autoref{ssec:primitiveEditorableDesign} that we should
    receive a string atom S-expression. We fail if the \ty{SExp} we get in the
    previous step is not a string atom. Otherwise we have access to the string
    \bn{s}.
  \item We parse the string \bn{s} into a abstract syntax tree term \bn{pterm} of
    the surface syntax, which is represented by the type \ty{PTerm} in the
    compiler. We therefore obtain the variable \bn{pterm}.
  \item Elaboration cannot resolve namespaces in the parsed surface syntax
    terms.  Using the current context, we write a function
    \mt{\fn{resolveNames} :: \ty{Context} -> \ty{PTerm} -> \ty{Either} \ty{Err} \ty{PTerm}}
    that traverses the the surface language abstract syntax tree using
    Uniplate\cite{uniplate}, and then at the \dt{PRef} terms, i.e. variable
    references, we search the context to find the matching fully namespaced
    name. If there is a unique matching name, we change the name. If there are multiple
    matching names, we return an error that there is ambiguity in the name. If there are
    no finds, we leave it as is and let elaboration handle the rest.
    We apply the current context and \bn{pterm} to this function and obtain
    \bn{pterm'}, the surface syntax term with resolved names.\footnote{
     We can do better when there is a known context and type, e.g. at a hole.
     Then we could elaborate in a local synthesized context to use types for
     overloading. This is a possible direction for future work.}
  \item Elaborate \bn{pterm'} into the core language and get \bn{t}.
  \item Now, remember that our function \fn{fromEditor} must return
    \ty{Elab TT} in this context. Therefore in the Haskell implementation, we
    want to return the Haskell representation of the Idris representation of
    the Haskell representation of the given code. In other words, we have to
    reflect \bn{t}, which is a \ty{Term}, and then get a \Raw\ term, which
    is a Haskell representation of the Idris representation of \bn{t}.
    We can call the reflected term \bn{reflected}.
  \item \fn{runTacTm} requires us to return a \ty{Term}, therefore we
    type-check \bn{reflected} and get a Haskell term of the type \ty{Term},
    which we can call \bn{tmReflected}.
  \item When we return a core syntax tree, we want to return it in normal form,
    therefore we normalise \bn{tmReflected} and return it.
\end{enumerate}
% David's comment: do we always want the normal form? why?

For the \TT\ case of \dt{Prim\_\_ToEditor}, we have to do the following steps:
\begin{enumerate}
  \item We have an Idris term \bn{arg'} representing the \TT\ term passed to
    the \dt{Prim\_\_ToEditor}, we have to reify this using \fn{reifyTT} and
    get a Haskell term with the Haskell type \ty{Term}.
    In other words, we are given the Haskell representation of the Idris
    representation of the Haskell representation of a term.
    We want to get the Haskell representation of that term, using reification.
  \item We now have a \ty{Term} term named \bn{v}. We delaborate and resugar
    \bn{v} and get the surface syntax version of it. We name the delaborated
    version \bn{pterm}, which has the type \ty{PTerm}.
  \item We pretty print \bn{pterm} and get a string \bn{s}.
  \item With \bn{s}, we can create \texttt{\dt{StringAtom} \bn{s}}, which is a
    Haskell term with the Haskell type \ty{SExp}. Using \fn{reflectSExp}, we
    can get the Haskell representation of an Idris term that has the Idris type
    \ty{SExp}, which we call \bn{tm}.
  \item When we return a core syntax tree, we want to return it in normal form,
    therefore we normalise \bn{tm} and return it.
\end{enumerate}

The primitive implementations for the other types, namely \ty{TyDecl},
\ty{DataDefn}, \ty{FunDefn}, and \ty{FunClause}, are not significantly
different from \ty{TT}. Hence we will not go over them.

In the next section, we will see how we will deal with the issue of local
contexts when we elaborate terms we receive from the editor.

\section{Extensions to elaboration and the internal Idris state}
\label{sec:extendIState}

In \autoref{ssec:usingEditorable}, we talked about how we wanted to deserialize
S-expressions into the arguments to the \Elab\ action we want to run.
As we seen in the previous section, in order to deserialize an S-expression
into a term of the type \TT, we have to parse, elaborate and reflect.
During elaboration, we cannot have any unbound variables, otherwise we get an
error.

Nevertheless, the editor can send to the compiler code pieces that contain local variables.
Handling the global context is easy, but when it comes to a local context, e.g. if the
code snippet we received uses a variable whose binding is introduced by
\kw{let}, \kw{case}, or a lambda elaboration is doomed to fail.
We need to find a way to get the local context at a given position, and then
pass it to elaboration.

If we restricted \Elab\ editor actions to only holes, this would not have been
a problem, since holes already provide a way to query the local context at
them. For other names and terms, however, we do not have a way.

To solve this problem, we extended the elaboration state with an interval map
in which keys are source locations, i.e. a pair of line and column numbers, and
values are local contexts, i.e. environments. We use the finger tree
implementation of interval maps in Haskell.\cite{fingertrees}

For those who are unfamiliar, an interval map is a data structure that maps
intervals to values. Every entry consists of the interval between two keys, and
a value associated with the interval. One can query the map with a single
key, and get the values in the map which are mapped by the intervals that
the key is in.

During elaboration, we still have access to the source locations of all terms.
Therefore we can register the interval between the start and end locations of
the term in the source code, and then map it to the local context.
Following the terminology we used in the compiler, we will call this map the
\textbf{source map}.

This task required significant refactoring, since the source map
needs to be preserved during elaboration of different parts, and also after
elaboration. Elaboration itself cannot change the internal Idris state
\ty{IState}; however, it has its own state \ty{ElabState}. When elaboration
finishes, we then update \ty{IState} with the additions.

In the next section, we will see how exactly \fn{fromEditor} and \fn{toEditor}
are used in the IDE mode of the compiler to communicate with the editor.

\section{Extensions to the Idris IDE mode}\label{sec:idemode}

Idris' IDE mode has an S-expression parser, which looks at the shape of an
S-expression and creates a IDE mode command value based on it. We extend this
parser by adding a line that says that if we see a symbol atom \dt{:elab-edit},
a string atom for the \Elab\ action name, an S-expression list, and two integer
atoms for line and column numbers (in that order), then we add a new IDE mode
command in the compiler.

These commands are passed around for a while in the compiler, however
eventually there are separate functions that deal with each editor command. We
write the function for edit-time tactics under the name \fn{elabEditAt} in
\autoref{code:elabEditAt}.

\begin{figure}[ht]
\caption{The function \fn{elabEditAt} that runs editor actions in the compiler.}
\label{code:elabEditAt}
\begin{Verbatim}[framesep=2mm, label=\footnotesize{\normalfont{Haskell}}, labelposition=topline]
\fn{elabEditAt}
  :: \ty{FilePath}   \cm{-- ^ The file name in which the Elab action is run}
  -> \ty{String}     \cm{-- ^ The name of the action}
  -> (\ty{Int}, \ty{Int}) \cm{-- ^ The line and column number the action is run on}
  -> [\ty{SExp}]     \cm{-- ^ The arguments to the action}
  -> \ty{Idris} \ty{()}
\fn{elabEditAt} \bn{filename} \bn{nameStr} \bn{pos} \bn{args} = ...
\end{Verbatim}
\end{figure}

Once again the actual code block for \fn{elabEditAt} refers to many helper
functions that are difficult to follow; we will instead explain what it does in
detailed prose. This will read like a much more detailed version of our
description in \autoref{ssec:usingEditorable} about how \ty{Editorable} works
in the compiler.

Here are the steps \fn{elabEditAt} takes when it is called:
\begin{enumerate}
  \item The variable \bn{nameStr}, which has the Haskell type \ty{String},
    holds the name of the \Elab\ action we want to run. However, we need
    to parse this string into a \ty{Name}, in order to get any namespace
    information that is provided inside \bn{nameStr}.
    If there is any namespace information, we use \dt{NS}, otherwise we use
    \dt{UN}.  We call the result \bn{name}, which has the type \ty{Name}.
  \item Now we have \bn{name}, yet we do not know whether it really refers to a
    existing \Elab\ action. If it does what is its type?
    We consult the context and find out its type, which we can call \bn{ty}, which
    has the Haskell type \ty{Type}. From the context, we also resolve the
    name of the \Elab\ action if we need to, we call the resolved name \bn{ns}.
  \item Using the resolved name \bn{ns}, we look up the flags declared for this function.
    We check if the it has been declared an editor action using the
    \kw{\%editor} modifier.  If not, we do not proceed.
  \item We are given a list of \ty{SExp}s, which is the list that the editor
    passes to the compiler to apply the editor action they want to call.
    However, we have to check whether this list contains the right number of
    arguments, and for that we need to know how many arguments the editor action should have.
    For that, we can write a function \fn{collectTypes} that takes a the
    \fn{Type} \bn{ty} and gives us a list of \fn{Type}s, which are the
    components of \bn{ty}, i.e.  curried arguments and also the return type.
    \begin{figure}[ht]
    \caption{The function \fn{collectTypes} to dissect a type signature to its components.}
    \label{code:collectTypes}
    \begin{Verbatim}[framesep=2mm, label=\footnotesize{\normalfont{Haskell}}, labelposition=topline]
    \fn{collectTypes} :: \ty{Type} -> [\ty{Type}]
    \fn{collectTypes} (\dt{Bind} _ (\dt{Pi} _ _ \bn{t1} _) \bn{t2}) = \bn{t1} \fn{:} \fn{collectTypes} \bn{t2}
    \fn{collectTypes} \bn{t} = [\bn{t}]
    \end{Verbatim}
    \end{figure}
    We apply \bn{ty} to \fn{collectTypes} and name the result \bn{collected},
    which is a list of \ty{Type}s.\footnote{Observe that this throws away the
    binding structure of the $\Pi$-type, which mean \kw{\%editor} actions
    cannot have dependent types. As we mentioned before about polymorphic
    editor actions, we believe monomorphic non-dependently-typed editor actions
    suffice for almost all editor actions. We leave further exploration for
    future work.}
  \item We check if the length of \bn{collected} is exactly one less than the
    length of \bn{args}, because the last element in \bn{collected} is the
    return type of the \Elab\ action.
  \item Using the interval map we defined in \autoref{sec:extendIState} and the
    line and column number the cursor in on, which are given to \fn{elabEditAt}
    as arguments, we query the local context at the cursor location, which we
    call \bn{env}.
  \item We zip \bn{collected} and \bn{args} and get a list of pairs of
    \ty{Type} and \ty{SExp}.
    For each of the pairs, we have to do the following and get a result for
    each \mbox{pair of \mt{(\bn{argTy}, \bn{sexp})}}:
    \begin{enumerate}[(a)]
      \item \label{item:fromEditor} Construct an Haskell representation of an
        Idris term that calls \fn{fromEditor}. Remember that \fn{fromEditor} is
        polymorphic and also has an interface constraint; therefore,
        constructing this as a \ty{Term} is not trivial.  As a shortcut, we can
        construct a \ty{PTerm} that explicitly applies the polymorphic type,
        which will be \bn{argTy} in this case.
      \item Elaborate the \ty{PTerm} we construct above, which gives us
        \bn{tm}, the Haskell representation of an Idris term that has the type
        represented by \bn{argTy}, but in the Idris \Elab\ monad.
      \item Since \bn{tm} is the Haskell representation of an \Elab\ action, we
        should run it using \fn{runElabAction}, under the local context
        \bn{env}. This gives us \bn{tm'}, the Haskell representation of a term
        that has the type represented by \bn{argTy}.
      \item We return \bn{tm'} as the result for the pair
        \texttt{(\bn{argTy}, \bn{sexp})}.
    \end{enumerate}
  \item We make a list of all the \bn{tm'}s returned for each pair.
  \item Remember that this list consists of the results parsing each
    S-expression into what type they represent. Therefore, its elements are
    actually Haskell representations of argument for the \Elab\ action we want
    to call. Hence, we construct a term by applying the elements of the list to
    the variable reference to \bn{ns}, which we can call \bn{app}.
  \item \label{item:extract}
    Now we have a Haskell representation of an \Elab\ action, because we
    applied all the arguments required.  From the last element of
    \bn{collected}, which is supposed to be the return type, we can find out
    what type the term we expect when we run \bn{app}.  The last element of
    \bn{collected} is a Haskell representation of an Idris type in the shape of
    \texttt{\ty{Elab} \bn{a}}, for some \bn{a}. We extract the Haskell
    representation for the Idris type \bn{a} from it.  We name the extracted
    representation \bn{lastTyInElab}.
  \item Using \fn{runElabAction} and \bn{lastTyInElab}, we run \bn{app}. This
    gives us \bn{res}, the Haskell representation of a term that has the Idris
    type that \bn{lastTyInElab} represents.
  \item We want to serialize this term before returning it as an S-expression.
    The only way we know how to serialize is through the \fn{toEditor} function
    in \ty{Editorable}, but remember that \fn{toEditor} is defined in Idris.
    We thus construct a Haskell representation of an Idris term that calls
    \fn{toEditor}, similar to how we constructed an application term for
    \fn{fromEditor} before in \autoref{item:fromEditor}.
    For convenience in implementation resolution, we first construct a
    \ty{PTerm}, then elaborate it. We call the elaborated term \bn{tm}.
  \item Since \bn{tm} is the Haskell representation of an \Elab\ action, we
    should run it using \fn{runElabAction}. We get the Haskell
    representation of a term that has the Idris type \ty{SExp}.
  \item We reify the Haskell representation of that term into a Haskell term
    that has the Haskell type \ty{SExp}, using \fn{reifySExp}, and get
    \bn{resSExp}.
  \item We send a message to the client/editor and report that the action is
    successful, and the result is \bn{resSExp}.
\end{enumerate}

\noindent This concludes the IDE mode section of implementing edit-time tactics.

\section{Extensions to the type-checker for the \kw{\%editor} modifier}\label{sec:extCheck}

The last remaining part of the implementation is an additional restriction to
type-checking.

If a function is declared an editor action with the \kw{\%editor} modifier, we
want to find out during type-checking if it is viable to use them as an editor
action, and an \Elab\ action is \emph{viable} is if the components of its type have
implementations of the \Editorable\ interface.\footnote{
  David Christiansen notes that this process could be internalized into Idris
  as a universe \`a la Tarski, instead of the \`a la Russell solution we
  present in our thesis. This approach can also be a starting point for
  dependently-typed editor actions.} We discussed this idea in the \fn{toy}
example in \autoref{ssec:primitiveEditorableDesign}, and also have seen the
type errors you can get in \autoref{ssec:preclude}. Now let's see how we
implement this feature.

To the \fn{elabType'} function in the compiler, we add the following:
\begin{enumerate}
  \item We use \fn{collectTypes} that we defined in \autoref{code:collectTypes} to get the components of the type signature, we will call this list of \ty{Type}s \bn{collected}.
  \item The last element in \bn{collected} is supposed to be return type. We
    extract \bn{lastTyInElab}, which is the Haskell representation of the Idris
    type in the Idris \Elab\ monad, the same way we did in
    \autoref{item:extract}.
  \item We replace the last element in \bn{collected} with \bn{lastTyInElab},
    and name the new list \bn{toCheck}.
  \item For every element \bn{ty} in \bn{toCheck}, we do the following:
    \begin{enumerate}[(a)]
      \item We want to find out if there is an \ty{Editorable} implementation
        for the Idris type \bn{ty} represents. The easy way to do that is to
        construct a \ty{PTerm} of a \fn{fromEditor} application and then
        elaborate it, as we did in \label{item:fromEditor} before, which should
        resolve implementation constraints, or give an error if it cannot.
      \item If there is no error, move on to the next \bn{ty} silently.
        If there is an error, that implies that there was no \ty{Editorable}
        implementation for the Idris type \bn{ty} represents. Then we send an
        error message like the ones in \autoref{ssec:preclude} back to the editor.
    \end{enumerate}
\end{enumerate}

This concludes the implementation of edit-time tactics in the Idris compiler.
In the next chapter, we will see real applications of edit-time tactics.

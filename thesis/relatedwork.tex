\chapter{Related work} \label{chap:relatedwork}

\section{In Haskell}

Template Haskell~\cite{th} is the main metaprogramming mechanism in Haskell.
It is similar to elaborator reflection in the sense that metaprograms are
defined in a monad called \ty{Q}, which allows metaprograms to create fresh
names and look up definitions.
Template Haskell metaprograms generate expressions and definitions, which are
among the capabilities of the \Elab\ monad in Idris.
However, there are significant differences;
quotations in Template Haskell return values in the \ty{Q} monad, and Template
Haskell does not try to reflect the elaboration infrastructure of
Haskell.\footnote{However, Haskell metaprogramming using the GHC core language
has been discussed in the GHC developers mail list, with credit to Idris:
\url{http://mail.haskell.org/pipermail/ghc-devs/2015-November/010402.html}}
Neither does it hold an internal proof state that can be changed by monadic
actions, nor it does try to provide an alternative way to implement tactics in
Haskell.\footnote{That being said, Siva Somayyajula has a rudimentary
implementation a tactic monad in Haskell based on the \ty{Q} monad:
\url{http://github.com/ssomayyajula/elab}}

Brian McKenna worked on expanding the definitions generated by Template Haskell
to source code, which then is pretty printed and put back into the source code
in Emacs using YASnippet.\footnote{His tweet with screenshots can be found at
\url{http://twitter.com/puffnfresh/status/935274097642057728} and the project
that enables this feature can be found at
\url{http://hackage.haskell.org/package/th-pprint}.}

On the IDE feature side of things, Alan Zimmerman and Matthew Pickering
developed \texttt{ghc-exactprint}\footnote{It can be found here:
\url{http://hackage.haskell.org/package/ghc-exactprint}}, which is a library
that helps IDE and tooling development by providing a way to automatically
refactor Haskell programs without changing a part of the program
unintentionally. As they put it, their library respects ``the identity
refactoring'', which is non-trivial if your system allows many different kinds
of transformations~\cite{ghc-exactprint-blog}.
There is also the Haskell IDE Engine project that aims integrate many Haskell
tools based on the GHC API to the editor workflow, by providing a backend for
editor modes.\footnote{The project can be found here:
\url{http://github.com/haskell/haskell-ide-engine}, and Alan Zimmerman's talk
at the Haskell Implementors' Workshop 2017 can be found here:
\url{http://youtu.be/-pjQcG94CxM}}

\section{In Agda}

There is a line of work on bringing more automated theorem proving,
proof automation and tactics, or metaprogramming to Agda.
Lindblad and Benke (2006) introduced a term search algorithm called
Agsy, a proof search mechanism that aims to save users' time by automating
parts of the proof that are straightforward but tedious to write~\cite{agsy}.
Agda has a derivative of this mechanism implemented as a part of its compiler.
Kokke and Swierstra (2015) used the Agda's prior reflection system to define a
new proof search mechanism in Agda itself~\cite{autoinagda}.
The Hezarfen tactic we discussed in \autoref{sec:hezarfen} is not as advanced
as their \fn{auto} function, yet in their paper, they discussed a feature
similar to edit-time tactics as future work:

\begin{quote}
  ``In the future, it may be interesting to explore how to integrate proof
  automation using the reflection mechanism better with Agda's IDE. For
  instance, we could create an IDE feature which replaces a call to
  \fn{auto} with the proof terms that it generates. As a result, reloading
  the file would no longer need to recompute the proof terms.''~\cite{autoinagda}
\end{quote}

In this thesis, we generalized their suggestion to all tactics, and specified
how the editor/IDE and the compiler should communicate with each other
in order to successfully call a tactic with inputs of the correct types.

There is also work on ``proof by reflection'' in Agda, which is different from our
usage of the word ``reflection'' so far.

\begin{quote}\label{quote:reflection}
``Reflection is an overloaded word in this context, since in programming language
  technology reflection is the capability of converting some piece of concrete
  code into an abstract syntax tree object that can be manipulated in the same
  system. Reflection in the proof technical sense is the method of mechanically
  constructing a proof of a theorem by inspecting its shape.''~\cite{reflInAgda}
\end{quote}

We have been concerned with the first meaning of ``reflection'' in this thesis,
however the work on the second meaning of this word is still relevant to proof
automation, and their ideas can be reused in our edit-time tactics. Work by van
der Walt and Swierstra showed compelling examples of proof by reflection in
Agda, such as a proof mechanism for boolean tautologies~\cite{pfByReflAgda}.

\section{In Coq}

Coq has a metaprogramming mechanism called
\texttt{template-coq}\footnote{It can be found here:
\url{https://github.com/Template-Coq/template-coq}} that is based on
Malecha's term reification~\cite{malecha-phd}. Recently a typed
version of this system is also introduced~\cite{typed-template-coq}.
However, we are not aware of any work on using template metaprograms in Coq to
write new features for the editor.

Aside from this, there is a large body of work on proof automation, proof
engineering and tactic languages in Coq.  Coq's original tactic language is
Ltac~\cite{ltac}, which is separate from its Coq's term language Gallina.
However, alternatives to Ltac have been developed, such as Mtac~\cite{mtac} and
MetaCoq~\cite{metacoq}.  Especially Mtac, which is a tactic language
for Coq that facilitates custom proof search by providing a monadic interface,
has inspired further research in the area, including Idris' elaborator
reflection~\cite{elabref}.

Chlipala's \emph{Certified Programming with Dependent Types}~\cite{cpdt} has
emerged as the canonical introductory textbook for proof engineering; it
explains the basics of tactic programming and even delves into proof search and
proof by reflection. Note that we use the word reflection in the proof
technical sense, \hyperref[quote:reflection]{as mentioned in the quote above}.

\section{In Lean}

Lean~\cite{lean}, which has a tactic metaprogramming system~\cite{leanmeta}
similar to Idris' elaborator reflection, also allows running tactics in
edit-time, and it does not require writing any code for the editor mode
frontend.\footnote{No Emacs Lisp if you are using Emacs.} The type of these
editor actions can be seen in \autoref{code:holecommand}.

\begin{figure}[H]
\caption{Definition of \ty{hole\_command} in Lean.}
\label{code:holecommand}
\begin{Verbatim}
\kw{meta} \kw{structure} \ty{hole_command} :=
  (\bn{name}   : \ty{string})
  (\bn{descr}  : \ty{string})
  (\bn{action} : \ty{list} \ty{pexpr} \ensuremath{\to} \ty{tactic} (\ty{list} (\ty{string} \ty{\ensuremath{\times}} \ty{string})))
\end{Verbatim}
\end{figure}

They provide the following documentation for \ty{hole\_command}:\footnote{From
the \href{https://github.com/leanprover/lean/blob/17fe3decaf8ae236f0d0ff51ac8fd7f6940acdee/library/init/meta/hole\_command.lean}{source
code}
of Lean 3.4.1.}

\begin{quote}
  ``The front-end (e.g., Emacs, VS Code) can invoke commands for holes
  \mt{\{! ... !\}} in a declaration. A command is a tactic that takes zero or
  more pre-terms in the hole, and returns a list of pair \mt{(\bn{s},
  \bn{descr})} where \bn{s} is a substitution and 'descr' is a short explanation
  for the substitution.  Each string \bn{s} represents a different way to fill
  the hole.  The frontend is responsible for replacing the hole with the
  string/alternative selected by the user.  This infrastructure can be used to
  implement auto-fill and/or refine commands. An action may return an empty
  list. This is useful for actions that just return information such as: the
  type of an expression, its normal form, etc.''
\end{quote}

In comparison to the edit-time tactics mechanism presented in our work, Lean's
system is very restrictive. It only allows editor action that run on holes, but
our system allows any kind of editor action as long as the user writes the
necessary glue code in the editor mode language. We already showed in
\autoref{code:elispToy}, what the glue code to fill a hole would look like in
Emacs Lisp. Another downside of Lean's system is that editor actions can only
have a single type, as opposed to our system, which allows any kind of
\Elab\ action as long as the components of the a type all have an
\Editorable\ implementation. Our system lets users write more expressive custom
editor actions.

\section{Others}

A question we received from Edward Morehouse was ``What are the prospects for
building editor interactions into a compiler from the start?  Is it feasible to
implement delaboration in such a way that the compiler could respond directly
with surface-syntax terms that fit in the current binding context?'' We believe
we should discuss the existing work on languages that are designed
with editor interaction in mind.

Building editor interactions in a compiler from the start is not a new idea,
both Idris and Agda have done this already. They did not, however, take
metaprogrammable editor interactions into account, and that is what our work
brings to Idris. We believe a path through Racket, a language-oriented
programming~\cite{racketManifesto, feltey2016languages} language would be an
interesting take on building a language around its editor interactions.
DrRacket~\cite{drracket}, Racket's IDE, makes writing editor interaction easy
for the languages defined in Racket. This not only eliminates a lot of
boilerplate code, but it also allows using Racket itself to define new editor
actions. There are already dependently-typed languages defined in Racket: one
example is Cur\footnote{It can be found here:
\url{http://github.com/wilbowma/cur}}~\cite{cur}, a proof assistant with
powerful metaprogramming tools. There is also Pudding\footnote{It can be found
here: \url{http://github.com/david-christiansen/pudding}}, a proof assistant in
development that uses Racket for specifications, proof automation, code
extraction and also extensions to the proof assistant itself.
Another one is Pie\footnote{It can be found
here: \url{http://github.com/the-little-typer/pie}}~\cite{theLittleTyper}, a
minimal language used for educational purposes.
We believe there is potential
for stronger editor interaction for these languages through metaprogramming.

Another path that is worth exploring more is structure editors. In the proof
assistant world, The Alfa proof editor~\cite{alfa} has established a proof
interface based on structure editor manipulating proof trees. More recently and
for a simpler type theory, the Hazel project~\cite{hazelnut,hazelEditor}
explored what a language designed around its editor would look like.
Specifically, they designed a structure editor and a type theory to deal with
incomplete programs in this setting. They also coined the term ``edit-time'' to
mean when the user is writing a program in the editor, and suggested
``edit-time tactics'' as future work\footnote{We learned this from their slides
and also personal communication with Cyrus Omar and Ravi Chugh.}, by which they
meant a separate language in which users can define editor actions, and a
library of predefined editor actions that the users can compose.

The second part of Morehouse's question was about implementing delaboration in
a way that the compiler could respond directly with surface-syntax terms that
fit in the current binding context.
Currently there is no way in Idris to write an editor action that returns a
surface-syntax term.\footnote{ The only way around returning surface-syntax
directly from an editor action is to return a \ty{String} that consists of the
code, but that is inelegant and we would like to avoid that.} The way
elaborator reflection is defined in Idris forces us to deal with core language
terms only, and for the rest we depend on the built-in delaboration.  There is
also no reflected Idris type that represents the surface syntax, since the
surface syntax can change quite often, maintaining its Idris representation
would be difficult, not to mention with every change it would likely break
users' code that depends on it.  Therefore, adding an Idris representation of
the surface-syntax is not planned.

Apart from Idris, it is possible to design a language that lets the users
define editor actions that return surface-syntax terms. We see two possible ways to do this:
\begin{enumerate}
  \item Not having a core language and surface-syntax distinction. This is not
    ideal if you have a large programming language, then the type-checking,
    evaluation, etc. have to be extended every time we want to add a new
    syntax. Not to mention that lacking features like implicit arguments is bad
    language ergonomics; elaboration is needed to resolve the implicit
    arguments~\cite{pollack}.
  \item Having a reflected type in your language that represents the
    surface-syntax terms, exposing the delaboration mechanism in your
    metaprogramming mechanism, and allowing splicing surface-syntax terms into
    programs. We are not aware of any work that does this.
\end{enumerate}

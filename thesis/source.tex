\documentclass[11pt, ma]{westhesis}
\setcounter{tocdepth}{4}
\setcounter{secnumdepth}{4}
\usepackage[utf8]{inputenc}
\usepackage{amsmath, amsfonts, amssymb, bussproofs, listings, float, upgreek, stmaryrd, epigraph, afterpage}
\usepackage[square,sort&compress,numbers]{natbib}
\usepackage[font=small]{caption}
\usepackage{inconsolata}
\usepackage{fancyvrb}
\usepackage[usenames,dvipsnames]{xcolor}
\usepackage[T1]{tipa}
\usepackage{graphicx}
\graphicspath{ {images/} }
% \usepackage{tgpagella}
\usepackage[hang, flushmargin]{footmisc}
\usepackage[backref=page,hyperfootnotes=false]{hyperref}
\usepackage{footnotebackref}
\hypersetup{colorlinks = true, allcolors = {blue}}
\definecolor{IdrisRed}{RGB}{166, 0, 2}
\definecolor{IdrisBlue}{RGB}{0, 0, 182}
\definecolor{IdrisGreen}{RGB}{13, 84, 8}
\definecolor{IdrisPurple}{RGB}{126, 2, 236}

% Idris
% in REPL run `:pp latex 80 <expr>` to get the colored verbatim
\newcommand{\IdrisData}[1]{\textcolor{IdrisRed}{#1}}
\newcommand{\IdrisType}[1]{\textcolor{IdrisBlue}{#1}}
\newcommand{\IdrisBound}[1]{\textcolor{IdrisPurple}{#1}}
\newcommand{\IdrisFunction}[1]{\textcolor{IdrisGreen}{#1}}
\newcommand{\IdrisMetavar}[1]{\textcolor{cyan}{#1}}
\newcommand{\IdrisKeyword}[1]{{\textbf{#1}}}
\newcommand{\IdrisImplicit}[1]{{\itshape \IdrisBound{#1}}}
\usepackage{setspace, microtype}
\fvset{commandchars=\\\{\},formatcom=\singlespacing, frame=single}
\newcommand{\ty}[1]{\IdrisType{\texttt{#1}}}
\newcommand{\kw}[1]{\IdrisKeyword{\texttt{#1}}}
\newcommand{\fn}[1]{\IdrisFunction{\texttt{#1}}}
\newcommand{\dt}[1]{\IdrisData{\texttt{#1}}}
\newcommand{\bn}[1]{\IdrisBound{\texttt{#1}}}
\newcommand{\hole}[1]{\IdrisMetavar{\texttt{?}\IdrisMetavar{\texttt{#1}}}}
\newcommand{\Elab}{\ty{Elab}}
\newcommand{\String}{\ty{String}}
\newcommand{\TT}{\ty{TT}}
\newcommand{\TyDecl}{\ty{TyDecl}}
\newcommand{\FunDefn}{\ty{FunDefn}}
\newcommand{\FunClause}{\ty{FunClause}}

\newcommand{\quadthree}{\qquad\quad}
\newcommand{\quadfour}{\quadthree\quad}
\newcommand{\quadfive}{\quadfour\quad}
\newcommand{\quadsix}{\quadfive\quad}
\newcommand{\quadseven}{\quadsix\quad}
\newcommand{\quadeight}{\quadseven\quad}
\newcommand{\quadten}{\quadfive\quadfive}
\newcommand{\Red}[1]{{\color{red} #1}}
\newcommand{\TODO}[1]{{\color{red}{[TODO: #1]}}}
\newcommand{\FYI}[1]{{\color{green}{[FYI: #1]}}}


% \allowdisplaybreaks

\department{Mathematics and Computer Science}
\submitdate{April 2018}
\advisor{Daniel R. Licata}
\title{Edit-Time Tactics in Idris}
\author{Joomy Korkut}

\theoremstyle{plain}
\newtheorem{theorem}{Theorem}[chapter]
\newtheorem{corollary}{Corollary}[theorem]
\newtheorem{lemma}[theorem]{Lemma}

\theoremstyle{definition}
\newtheorem{defn}{Definition}[chapter]

\newcommand{\T}[1]{\texttt{#1}}
\lstset{mathescape=true,basicstyle=\ttfamily}

\newcommand\numberthis{\addtocounter{equation}{1}\tag{\theequation}}

\begin{document}

\begin{abstract}
  Metaprogramming is a technique that allows users to write programs that
  write programs. In dependently-typed languages such as Idris, recent work on
  elaborator reflection paved the way for new applications of metaprogramming
  by showing that it can be a substitute for tactic-based proof languages.
  The goal of this work is to use elaborator reflection to write editor
  interaction actions of Idris.
  Currently in Idris modes of editors such as Emacs, users can perform actions
  like type-checking holes, case splitting, and lemma extraction.
  Implementations of all of these Idris editor actions are hard-coded in the
  compiler and they are written in Haskell. This work will allow us to rewrite
  them in Idris as metaprograms, and to move them into an Idris library instead
  of having them embedded into the compiler.
  Furthermore, we can write our own tactics through elaborator
  reflection, and run them from the editor, i.e. in edit-time.
  This would extend the abilities of the editor interaction mode from the
  current built-in features to anything that can be done with tactics.
  In this work, we present the design and implementation of this feature in the
  Idris compiler.
  We also implement an intuitionistic theorem prover tactic that is meant to be an
  better alternative to the built-in proof search editor action, and a case
  splitting tactic that exemplifies how we can move some of the hard-coded
  features from the compiler to a library.
\end{abstract}

\begin{dedication}
  \epigraph{``Soon enough, [mathematicians] are going to find themselves doing mathematics at the computer, with the aid of computer proof assistants. Soon, they won’t consider a theorem proven until a computer has verified it.''}{\textit{Vladimir Voevodsky}\cite{voevodskyQuote}}
  % https://blogs.scientificamerican.com/guest-blog/voevodskye28099s-mathematical-revolution/
\end{dedication}

\begin{acknowledgements}
First and foremost, I would like to thank Dan Licata, my research advisor.
Even though I picked a topic that he was not particularly interested in, he
gladly played along and gave very helpful advice whenever I got
stuck. Since this project also had a user interfacing aspect, his insights as a
long-time proof assistant user shaped my choices.
He had more impact on my research taste than anyone, and there is no doubt I
will miss his astute advice when I leave Wesleyan.

I am incredibly fortunate to have met David Christiansen last summer and talked
to him about his work on metaprogramming for dependent types.
He laid the foundation that makes this project possible in his
dissertation\cite{davidphd}, and pitched me the idea of using elaborator
reflection for interactive editing.
I could not have finished this work without him putting up with my incessant
stream of questions.  (My words, not his. He would have strongly disagreed
with my word choices.)

I would like to express my gratitude to James Lipton and Edward Morehouse for
reading and evaluating my thesis. I would also like to thank Cyrus Omar and
Matthew Wilson, for agreeing to read my thesis draft. Cyrus coined the term
``edit-time tactics'' and even though he did so in a different
context\cite{hazelnut}, I cannot think of a better way to describe this
project.

I want to thank Joseph Cutler and Mitchell Riley for the programming languages
discussions we had over the entire year; I am grateful for our little PL
community.

Finally, I am thankful to Yulia, Mehmet, Recep, Kivanc, Damlasu, Isin Ekin and
my family for their emotional support. This has not been an easy year for me,
but I made it through with their help and friendship. Thank you.
\end{acknowledgements}

\frontmatter
\maketitle
% \makededication
% \makeack
\makeabstract

\tableofcontents

\mainmatter

% \begin{figure}[ht]
% \caption{Some Idris code}
% \label{code:idrisExample}
% \begin{Verbatim}

% \end{Verbatim}
% \end{figure}

\section{Introduction} \label{sec:introduction}

Rich type systems give programmers a way to express their intentions
as types, statically ruling out many incorrect programs. But rich
types are useful for much more than preventing mistakes: the
information provided by informative types can be used by programming
tools to guide program construction, automating away tedious details
and freeing programmers to concentrate on the parts of their
problem that require human creativity.

Type-driven programming environments are necessarily built according
to language developers' assumptions about how programmers will use
them. These assumptions, however, can never hold for all members of a
diverse community working on a variety of problems. Unfortunately, the
interactive features of Idris and Agda are presently built in to their
respective compilers, and skill in dependently typed programming does
not imply the ability to extend the implementation of dependently
typed languages and maintain those extensions so that they continue to
work as compilers are improved.

The Idris elaborator~\citep{idris} translates programs written in
Idris into a smaller core type theory, called \textsf{TT}.
The elaborator is written
in Haskell, making use of an elaboration monad to track the
complicated state that is involved. The high-level Idris language is
extensible using \emph{elaborator reflection}~\citep{davidphd,
  elabref}, which directly exposes the elaboration monad to Idris
programs, so that Idris can be extended in itself. Concretely,
elaborator reflection extends Idris with a primitive monad
\Elab{}. Just as values in \IO{} describe effectful programs to be
executed by the run-time system, \Elab{} actions describe effectful
programs to be run during elaboration.

We have extended Idris's implementation of elaborator reflection with
new primitives that enable it to be used to construct \emph{editor
  actions}. These editor actions have access to the full power of
\Elab{}, but instead of running in the course of elaboration, they are
manually invoked by programmers to modify already-elaborated programs.
With these new primitives, it becomes possible to write
domain-specific editor actions for embedded domain-specific
languages~\citep{dsel} and to replace parts of the compiler with
customizable library code written in Idris. Even more importantly,
users who were previously stuck with whatever the developers provided
are now empowered to make not only their language, but also their
environment, their own.

% \subsection*{Contributions}

We make the following contributions in this paper:

\begin{itemize}[topsep=0pt] % , leftmargin=10pt]
\item We explore the features that are necessary to use
  \mbox{elaborator reflection} to implement editor actions.
\item We describe a concrete realization of this design, and the
  communication protocol that allows it to work in multiple
  interactive environments.
\item We describe a non-trivial editor action that invokes a
  theorem prover for intuitionistic propositional logic to
  interactively fill a hole in an incomplete
  program.
\item We demonstrate that editor actions written in Idris are
  sufficently powerful to replace parts of implementation by
  reimplementing a feature that constructs initial implementations of
  functions, based on their type signatures.
\end{itemize}
%there's space for two more lines on the first page


\subsection{Extending An Editor in Idris}

Dependently typed languages typically both allow programs to be
incomplete and provide support for making them more complete. A
limited version of this support could be a facility that substitutes
the unit constructor (written \dt{()}, as in Haskell) for a hole of
the unit type (also written \ty{()}, as in Haskell), and the
reflexivity constructor \dt{Refl} when the goal is a reflexive case of
the equality type.

\begin{figure}[b]
\begin{Verbatim}
\kw{%editor}
\fn{easy} : \ty{TTName} -> \ty{Elab} \ty{TT}
\fn{easy} \bn{n} =
  \kw{do} \bn{ty} <- \fn{getType} \bn{n}
     \kw{case} \bn{ty} \kw{of}
       \kw{`(}\ty{()} \kw{:} \ty{Type}\kw{)} \kw{=>}
         \fn{pure} \kw{`(}\dt{()} \kw{:} \ty{()}\kw{)}
       \kw{`(}\ty{(=)} \{A=\textasciitilde{}\bn{a}\} \{B=\textasciitilde{}\bn{b}\} \textasciitilde{}\bn{x} \textasciitilde{}\bn{y}\kw{)} \kw{=>}
         \kw{do} \fn{converts} \bn{a} \bn{b}
            \fn{converts} \bn{x} \bn{y}
            \fn{pure} \kw{`(}\dt{Refl} \{A=\textasciitilde{}\bn{a}\} \{x=\textasciitilde{}\bn{x}\}\kw{)}
       _ \kw{=>}
         \fn{fail} \dt{[}\dt{TextPart} \dt{"Cannot solve"}\dt{]}
\end{Verbatim}
\hrulefill
\begin{Verbatim}
(\kw{defun} \fn{idris-easy} ()
  \dt{"Invoke the first example action."}
  (\fn{interactive})
  (\fn{idris-elab-hole-arg}
   \dt{"easy"} (\fn{list} (\fn{idris-name-at-point}))))
\end{Verbatim}
  \caption{A simple editor action in Idris (top) and its \mbox{Emacs Lisp} support code (bottom)}
  \label{fig:motivating-example}
\end{figure}

Figure~\ref{fig:motivating-example} presents an implementation of this
editor action. The \kw{\%editor} keyword registers the declaration as
an editor action. Its type states that, when passed a representation
of a name from Idris's core language, it will produce a representation
of a term in Idris's core language, potentially having
elaboration-time side effects. It is passed a name because Idris holes
are identified by name.

The first step is to look up the type of the hole to be replaced,
using \fn{getType}, which takes a name and returns the type of the definition
associated with that name.
If the name is ambiguous, \fn{getType} fails.  Having discovered the name's
type, it then pattern-matches on said type, using Idris's quasiquotation
syntax~\citep{idrisQuotation}.

The first case to be considered is the unit type. In this pattern, a
type annotation is needed due to the Haskell-style overloading of the
double-parentheses. If the case is the unit type, the quoted form of
the unit constructor is returned with \fn{pure}, which is analogous to
Haskell's \fn{return}.

The second case to be considered is the equality type, which is
heterogeneous~\citep{mcbridephd} in the Idris standard library.  The
equality type requires two implicit arguments~\citep{pollack}, called
\bn{A} and \bn{B}, as well as explicit arguments \bn{x} and \bn{y}.
When \bn{A} and \bn{B} are the same type, and \bn{x} and \bn{y}
can be judged to be equal according to that type, \dt{Refl} proves the
equality. The \fn{converts} action checks whether two quoted terms are
judgmentally equal, and fails if they are not. Having checked that the
types and their inhabitants are equal, the second case returns \dt{Refl}.

The third and final case matches any other goal, and it
fails. Additional cases could be added on an \emph{ad hoc} basis, or a
more automatic approach could be taken. See \citet{davidphd} or
\citet{elabref} for a description of how to increase the level of
automation; this example is chosen to be easier to understand.

Each of Idris's editor actions requires a small amount of
editor-specific code to provide a user interface, and editor actions
written in Idris are no exception. With a suitable library, most
editing actions can be accommodated with fewer than five lines of
Emacs Lisp, and we expect the burden to be similar for other
extensible editors. Including in-editor documentation, this example
requires five lines of Emacs Lisp.

\begin{figure}[h]
\begin{BVerbatim}
\fn{ex1} : \ty{()}
\fn{ex1} = \hole{ex1_impl}

\fn{ex2} : \fn{not} \dt{False} = \dt{True}
\fn{ex2} = \hole{ex2_impl}

\fn{ex3} : \dt{False} = \dt{True}
\fn{ex3} = \hole{ex3_impl}
\end{BVerbatim}
\hspace{1em}
\begin{BVerbatim}
\fn{ex1} : \ty{()}
\fn{ex1} = \dt{()}

\fn{ex2} : \fn{not} \dt{False} = \dt{True}
\fn{ex2} = \dt{Refl}

\fn{ex3} : \dt{False} = \dt{True}
\fn{ex3} = \IdrisError{\hole{ex3_impl}}
\end{BVerbatim}

  \caption{Before and after invoking \fn{easy}}
  \label{fig:motivating-example-exec}
\end{figure}

Figure~\ref{fig:motivating-example-exec} displays the results of
executing this editor action on three holes. In the first two
examples, the program is completed automatically. In the third
example, however, an error is indicated because the underlying
\ty{Elab} action fails.

%%% Local Variables:
%%% mode: latex
%%% TeX-master: "source"
%%% End:

\section{Background}\label{sec:background}

\newcommand{\zip}{\fn{zip}}
\subsection{Idris}

Idris is a dependently typed functional programming language. In simple terms,
dependent types allow us to do computation in types, just like we can do
computation in terms.\cite{davidphd} Moreover, we can use the computation
in types to shape our definitions of computations in terms.

A helpful intuition for functional programmers is to think about how the
concept that functions are values was a initially a novel idea, and now in
dependently typed programming, we promote types to values as well.
Thus a function can now take a term as an argument and return a type as a
result.\cite{lambdacube,henk}
For those familiar with Haskell type families\cite{typefamilies}, this is
similar to that, but notice that type families are like functions from types to types,
while we can have functions from terms to types.

In \autoref{code:vect}, we define a data type of vectors. It is exactly
like lists, except now the length is stored within the type. In other
words, as we add elements, the length is computed in type level.

After that, we define a function that zips two vectors. Notice that only two
cases are enough to cover all possible paths of this function. If we were to
define a \zip\ function for lists, we would need four cases: both empty, both
non-empty, one empty and one non-empty, and one non-empty and one empty.
However, our \zip\ function takes two arguments that are of the same length
\bn{n}.  Therefore, we cannot have a case with one empty and one non-empty,
because that contradicts with the fact that both vectors are of the same
length. Notice that the \zip\ function returns a vector of length \bn{n}.
In other words, the input vectors and the resulting vector are guaranteed to be
of the same length.

\begin{figure}[ht]
\caption{Example of a dependently typed Idris code: vectors}
\label{code:vect}
\begin{Verbatim}[framesep=2mm, label=\footnotesize{\normalfont{Idris}}, labelposition=topline]
\IdrisKeyword{data} \IdrisType{Vect} : \IdrisType{Nat} -> \IdrisType{Type} -> \IdrisType{Type} \IdrisKeyword{where}
  \IdrisData{Nil} : \IdrisType{Vect} \IdrisData{0} \IdrisImplicit{elem}
  \IdrisData{(::)} : \IdrisImplicit{elem} -> \IdrisType{Vect} \IdrisImplicit{len} \IdrisImplicit{elem} -> \IdrisType{Vect} (\IdrisData{S} \IdrisImplicit{len}) \IdrisImplicit{elem}

\IdrisFunction{zip} : \IdrisType{Vect} \IdrisImplicit{n} \IdrisImplicit{a} -> \IdrisType{Vect} \IdrisImplicit{n} \IdrisImplicit{b} -> \IdrisType{Vect} \IdrisImplicit{n} \IdrisType{(}\IdrisImplicit{a}\IdrisType{,} \IdrisImplicit{b}\IdrisType{)}
\IdrisFunction{zip} \IdrisData{[]} \IdrisData{[]} = \IdrisData{[]}
\IdrisFunction{zip} (\IdrisBound{x} \IdrisData{::} \IdrisBound{xs}) (\IdrisBound{y} \IdrisData{::} \IdrisBound{ys}) = \IdrisData{(}\IdrisBound{x}\IdrisData{,} \IdrisBound{y}\IdrisData{)} \IdrisData{::} \IdrisFunction{zip} \IdrisBound{xs} \IdrisBound{ys}
\end{Verbatim}
\end{figure}

In \autoref{code:evenodd}, we define
two similar data types that ensure that the given natural number is even or
odd, and compute the number within the type at every step. Later we give a
function that takes a natural number and generates either a value of the type
that ensures evenness or one that ensures oddity. This corresponds to a proof that for all $n \in \mathbb{N}$, $n$ is even or $n$ is odd.

\begin{figure}[ht]
\caption{Example of a dependently typed Idris code, evenness and oddity}
\label{code:evenodd}
\begin{Verbatim}[framesep=2mm, label=\footnotesize{\normalfont{Idris}}, labelposition=topline]
\IdrisKeyword{data} \IdrisType{Even} : \IdrisType{Nat} -> \IdrisType{Type} \IdrisKeyword{where}
  \IdrisData{EvenZ} : \IdrisType{Even} \IdrisData{0}
  \IdrisData{EvenSS} : \IdrisType{Even} \IdrisImplicit{n} -> \IdrisType{Even} (\IdrisData{S} (\IdrisData{S} \IdrisImplicit{n}))

\IdrisKeyword{data} \IdrisType{Odd} : \IdrisType{Nat} -> \IdrisType{Type} \IdrisKeyword{where}
  \IdrisData{Odd1} : \IdrisType{Odd} \IdrisData{1}
  \IdrisData{OddSS} : \IdrisType{Odd} \IdrisImplicit{n} -> \IdrisType{Odd} (\IdrisData{S} (\IdrisData{S} \IdrisImplicit{n}))

\IdrisKeyword{total}
\IdrisFunction{evenOrOdd} : (\IdrisBound{n} : \IdrisType{Nat}) -> \IdrisType{Either} (\IdrisType{Even} \IdrisBound{n}) (\IdrisType{Odd} \IdrisBound{n})
\IdrisFunction{evenOrOdd} \IdrisData{0} = \IdrisData{Left} \IdrisData{EvenZ}
\IdrisFunction{evenOrOdd} \IdrisData{1} = \IdrisData{Right} \IdrisData{Odd1}
\IdrisFunction{evenOrOdd} (\IdrisData{S} (\IdrisData{S} \IdrisBound{n})) = \IdrisKeyword{case} \IdrisFunction{evenOrOdd} \IdrisBound{n} \IdrisKeyword{of}
                        \IdrisData{Left} \IdrisBound{ev} => \IdrisData{Left} (\IdrisData{EvenSS} \IdrisBound{ev})
                        \IdrisData{Right} \IdrisBound{o} => \IdrisData{Right} (\IdrisData{OddSS} \IdrisBound{o})
\end{Verbatim}
\end{figure}

Unlike other dependently typed languages like Agda and Coq, Idris is not total
by default. This is because Idris prioritizes general purpose programming
rather than theorem proving. However, users can opt in to enable totality
checking either for the entire module or for specific functions.
We did the latter for \fn{evenOrOdd} by using the keyword \kw{total}.
Similarly, we could require the \zip\ function from the previous example to be
total if we wanted to.\footnote{Clearly Idris cannot decide whether a given
function is total, since that would solve the halting problem. Instead it
acknowledges the ones that are obviously terminating, and for all the other
ones, even if they are actually total, it throws a totality check error.}

Haskell's type classes and type class instances are called interfaces and
implementations in Idris, respectively. In Haskell there can only be one
instance for the same type class and the type, but in Idris for the same
interface and the type, there can be multiple implementations. You can name
implementations and specify the implementation you want to use by its name,
when you are writing a function. For our purposes, we will not use multiple
implementations.

\subsection{Elaborator Reflection}\label{ssec:elabref}

Idris programs are elaborated from high-level Idris syntax trees into a core
language called \texttt{TT}, and then type checked.\cite{idris}
The implementation of the Idris elaboration in the compiler is written as a
Haskell monad called \texttt{Elab}.
Recent work on elaborator reflection\cite{elabref} allowed Idris users to
access this monad from Idris itself, by implementing a primitive monad \Elab\
in Idris itself, that can only be used for metaprogramming in compile time.

\subsubsection{Reflected Core Language Types}

Since this monad works with core language terms and definitions, the types in
the compiler representing the syntax trees for the core language are reflected
in Idris. In other words, there are now types in Idris itself, that match the
syntax tree types in the compiler. Most important one of all is called \TT,
which is type of the core language typed terms, and its definition can be seen
in \autoref{code:ttDef}.

\begin{figure}[ht]
\caption{The reflected type \protect\TT\ in Idris.}
\label{code:ttDef}
\begin{Verbatim}[framesep=2mm, label=\footnotesize{\normalfont{Idris}}, labelposition=topline]
\IdrisKeyword{data} \IdrisType{TT} : \IdrisType{Type} \IdrisKeyword{where}
  \IdrisData{P} : \IdrisType{NameType} -> \IdrisType{TTName} -> \IdrisType{TT} -> \IdrisType{TT}
  \IdrisData{V} : \IdrisType{Int} -> \IdrisType{TT}
  \IdrisData{Bind} : \IdrisType{TTName} -> \IdrisType{Binder} \IdrisType{TT} -> \IdrisType{TT} -> \IdrisType{TT}
  \IdrisData{App} : \IdrisType{TT} -> \IdrisType{TT} -> \IdrisType{TT}
  \IdrisData{TConst} : \IdrisType{Const} -> \IdrisType{TT}
  \IdrisData{Erased} : \IdrisType{TT}
  \IdrisData{TType} : \IdrisType{TTUExp} -> \IdrisType{TT}
  \IdrisData{UType} : \IdrisType{Universe} -> \IdrisType{TT}
\end{Verbatim}
\end{figure}

As a quick summary:
\begin{itemize}
\item\dt{P} creates a variable term from a name and the type of the variable.
\item\dt{V} creates a de Bruijn variable.
(given integer $n$ representing the $n$th most recently introduced local variable)
\item\dt{Bind} creates any kind of binder (lambda, let etc.) with a term it binds on.
\item\dt{App} creates a function application.
\item\dt{TConst} creates a constant such as an integer, a character, a string etc.
\item\dt{Erased} creates a term that is not known. This is used for erasing the types we do not need later in the compilation.
\item\dt{TType} creates a type of types for a given universe.
\item\dt{UType} creates a uniqueness type for a given uniqueness universe.
\end{itemize}

This summary is meant to be an overview, so refer to \cite{idris} and
\cite{elabref} if this is not perfectly clear. For our purposes, we will mostly
be concerned with \dt{P}, \dt{Bind} and \dt{App}.

The other important type that is used in elaborator reflection is \ty{Raw},
which is the type of untyped core language terms, and its definition can be
seen in \autoref{code:rawDef}.

\begin{figure}[ht]
\caption{The reflected type \protect\ty{Raw} in Idris.}
\label{code:rawDef}
\begin{Verbatim}[framesep=2mm, label=\footnotesize{\normalfont{Idris}}, labelposition=topline]
\IdrisKeyword{data} \IdrisType{Raw} : \IdrisType{Type} \IdrisKeyword{where}
  \IdrisData{Var} : \IdrisType{TTName} -> \IdrisType{Raw}
  \IdrisData{RBind} : \IdrisType{TTName} -> \IdrisType{Binder} \IdrisType{Raw} -> \IdrisType{Raw} -> \IdrisType{Raw}
  \IdrisData{RApp} : \IdrisType{Raw} -> \IdrisType{Raw} -> \IdrisType{Raw}
  \IdrisData{RType} : \IdrisType{Raw}
  \IdrisData{RUType} : \IdrisType{Universe} -> \IdrisType{Raw}
  \IdrisData{RConstant} : \IdrisType{Const} -> \IdrisType{Raw}
\end{Verbatim}
\end{figure}

The constructors of \ty{Raw} are almost the same ones as \TT,
except a few of them are missing and variables do not have to be annotated with
their types.


The \ty{TTName} type is the type of names in the core language, its full definition can be seen in \autoref{code:ttnameDef}.

\begin{figure}[ht]
\caption{The reflected type \protect\ty{TTName} in Idris.}
\label{code:ttnameDef}
\begin{Verbatim}[framesep=2mm, label=\footnotesize{\normalfont{Idris}}, labelposition=topline]
\IdrisKeyword{data} \IdrisType{TTName} : \IdrisType{Type} \IdrisKeyword{where}
  \IdrisData{UN} : \IdrisType{String} -> \IdrisType{TTName}
  \IdrisData{NS} : \IdrisType{TTName} -> \IdrisType{List} \IdrisType{String} -> \IdrisType{TTName}
  \IdrisData{MN} : \IdrisType{Int} -> \IdrisType{String} -> \IdrisType{TTName}
  \IdrisData{SN} : \IdrisType{SpecialName} -> \IdrisType{TTName}
\end{Verbatim}
\end{figure}

As a quick summary:
\begin{itemize}
\item\dt{UN} represents the variable names without any namespace.
\item\dt{NS} represents the variable names with a given namespace. For example, the name \texttt{Prelude.Bool.True} is represented as
  \texttt{\dt{NS} (\dt{UN} \dt{"True"}) \dt{["Bool", "Prelude"]}}.
\item\dt{MN} represents machine generated names with a hint string and a fresh integer for that hint.
\item\dt{SN} represents special names, which are used for metavariables, implementations etc. We will not deal with them in this thesis.
\end{itemize}

As a quick way to refer to Idris names, there is a syntax
\texttt{\IdrisKeyword{\`{}\{\{}x\IdrisKeyword{\}\}}}
that would give you the term \dt{UN} \dt{"x"}.
Similarly, there is another syntax that checks whether a given name exists and
lets you refer to an existing name without having to specify its full
namespace: \texttt{\IdrisKeyword{\`{}\{}False\IdrisKeyword{\}}} would give you
\texttt{\dt{NS} (\dt{UN} \dt{"False"}) \dt{["Bool", "Prelude"]}}.

\medskip

There are many other types used in the reflection of the core language, but we
will not give their definitions here since they are not as common as \TT,
\ty{Raw}, and \ty{TTName}. However, it would be useful to at least list the
important ones and describe what they represent.

\begin{itemize}
\item\ty{TyDecl} represents type declarations.
\item\ty{FunDefn} represents function definitions.
\item\ty{FunClause} represents a single clause in a function definition.
\item\ty{DataDefn} represents data type definitions.
\end{itemize}

\subsubsection{Quasiquotation}

Writing \TT\ and \ty{Raw} terms by hand can get tedious, hence there is a
quasiquotation syntax that elaborates a given expression into its corresponding
\TT\ or \ty{Raw} term.\cite{idrisQuotation}
The syntax \texttt{\`{}(\fn{e})}, where \fn{e} is an Idris expression, gives
us the typed or untyped core language syntax tree for \fn{e}. For example,
\texttt{\`{}(\fn{not} \dt{True})} gives us the following \TT\ term:
\begin{Verbatim}[framesep=2mm, label=\footnotesize{\normalfont{Idris}}, labelposition=topline]
\IdrisData{App} (\IdrisData{P} \IdrisData{Ref}
       (\IdrisData{NS} (\IdrisData{UN} \IdrisData{"not"}) \IdrisData{[}\IdrisData{"Bool"}\IdrisData{,} \IdrisData{"Prelude"}\IdrisData{]})
       (\IdrisData{Bind} (\IdrisData{UN} \IdrisData{"__pi_arg"})
             (\IdrisData{Pi} (\IdrisData{P} (\IdrisData{TCon} \IdrisData{8} \IdrisData{0}) (\IdrisData{NS} (\IdrisData{UN} \IdrisData{"Bool"}) \IdrisData{[}\IdrisData{"Bool"}\IdrisData{,} \IdrisData{"Prelude"}\IdrisData{]}) \IdrisData{Erased})
                 (\IdrisData{TType} (\IdrisData{UVar} \IdrisData{"./Prelude/Bool.idr"} \IdrisData{71})))
             (\IdrisData{P} (\IdrisData{TCon} \IdrisData{8} \IdrisData{0}) (\IdrisData{NS} (\IdrisData{UN} \IdrisData{"Bool"}) \IdrisData{[}\IdrisData{"Bool"}\IdrisData{,} \IdrisData{"Prelude"}\IdrisData{]}) \IdrisData{Erased})))
    (\IdrisData{P} (\IdrisData{DCon} \IdrisData{1} \IdrisData{0})
       (\IdrisData{NS} (\IdrisData{UN} \IdrisData{"True"}) \IdrisData{[}\IdrisData{"Bool"}\IdrisData{,} \IdrisData{"Prelude"}\IdrisData{]})
       (\IdrisData{P} (\IdrisData{TCon} \IdrisData{0} \IdrisData{0}) (\IdrisData{NS} (\IdrisData{UN} \IdrisData{"Bool"}) \IdrisData{[}\IdrisData{"Bool"}\IdrisData{,} \IdrisData{"Prelude"}\IdrisData{]}) \IdrisData{Erased}))
\end{Verbatim}

The \ty{Raw} term for the same expression is a bit more reasonable:
\begin{Verbatim}[framesep=2mm, label=\footnotesize{\normalfont{Idris}}, labelposition=topline]
(\IdrisData{RApp} (\IdrisData{Var} (\IdrisData{NS} (\IdrisData{UN} \IdrisData{"not"}) \IdrisData{[}\IdrisData{"Bool"}\IdrisData{,} \IdrisData{"Prelude"}\IdrisData{]}))
      (\IdrisData{Var} (\IdrisData{NS} (\IdrisData{UN} \IdrisData{"True"}) \IdrisData{[}\IdrisData{"Bool"}\IdrisData{,} \IdrisData{"Prelude"}\IdrisData{]})))
\end{Verbatim}

Obviously, we would not want to write terms like these manually every time we want to return
the syntax tree for a simple function application.
At times like this, quasiquotation saves us, for both expressions and patterns.

We can also give the type of the expression we want to elaborate, which becomes
necessary when Idris cannot infer the type. For \dt{True}, it is trivial to
infer that type is \ty{Bool}, but for \dt{5}, the type can be \ty{Int},
\ty{Integer}, \ty{Nat}, or anything that satisfies the \ty{Num} interface.
Therefore, we have to specify the type when we are quasiquoting. The
syntax for that is \texttt{\`{}(\fn{e} : \ty{t})},
e.g. \texttt{\`{}(\dt{5} : \ty{Nat})}.

We also can do antiquotation. If we have some variable expression \fn{x} that has the
type \TT\ or \ty{Raw}, then we can construct a syntax tree using it within the
quasiquotation, with the syntax \texttt{\`{}(\fn{not} ~\fn{x})}.
Note that antiquotation works for expressions, not just variables. The type of
expression or variable we have in the antiquotation has to match the type of
the quasiquotation. In other words, only \TT\ expressions can be used in an
antiquotation in a \TT\ quasiquotation, and mutatis mutandis for \ty{Raw}.

\subsubsection{\protect\Elab\ monad}

The elaborator reflection\cite{elabref} feature that has been added to the
Idris compiler recently provides a tool for metaprogramming with a monad called
\Elab.  This monad is implemented as a primitive and it can only be run during
compile time.

Elaborator reflection adds a new declaration
\texttt{\IdrisKeyword{\%runElab}}\ \fn{e} to Idris,
where \fn{e} has the type \ty{Elab}\ \ty{()}.
This declaration runs the \Elab\ action and adds new type declarations,
function and data type definitions generated by the \Elab\ action generated by
\fn{e} to the context.

Elaborator reflection also adds a new expression
\texttt{\IdrisKeyword{\%runElab}}\ \fn{e} to Idris,
where \fn{e} has the type \ty{Elab}\ \ty{()}.
The type \ty{t} of the entire expression is started as the goal of the
\Elab\ action, and the tactics in \fn{e} must solve the goal
that has the type \ty{t}.
Like the declaration above, this expression also adds the newly generated
declarations and definitions to the context.

\Elab\ monad holds a proof state inside, which has a goal type, a proof term that is incrementally built up, a hole queue, a collection of open unification problems, and a supply of fresh names.\cite{elabref}
Tactics can change the proof state. Here are some examples that do that:
\begin{itemize}
\item\texttt{\IdrisFunction{claim} : \IdrisType{TTName} -> \IdrisType{Raw} -> \IdrisType{Elab} \IdrisType{()}}\\
Creates a new hole with a given name and a type.
\item\texttt{\IdrisFunction{fill} : \IdrisType{Raw} -> \IdrisType{Elab} \IdrisType{()}}\\
Create a guess to fill the current hole with a term. Fail if the types do not unify.
\item\texttt{\IdrisFunction{solve} : \IdrisType{Elab} \IdrisType{()}}\\
Try to finalize the guess in the hole. Fail if there is no guess.
\end{itemize}

There are a lot more tactics, which we will not list here. A more thorough list
can be found in \cite{elabref} and Idris documentation.

We also have access to \Elab\ actions that do not change the proof state, but give us access to the context or other compiler primitives:
\begin{itemize}
\item\texttt{\IdrisFunction{check} : \IdrisType{List} \IdrisType{(}\IdrisType{TTName}\IdrisType{,} \IdrisType{Binder} \IdrisType{TT}\IdrisType{)} -> \IdrisType{Raw} -> \IdrisType{Elab} \IdrisType{(}\IdrisType{TT}\IdrisType{,} \IdrisType{TT}\IdrisType{)}}\\
Type-checks a term under a given environment and gives the typed core term version of the \ty{Raw} term and the type of it as a typed core term.
\item\texttt{\IdrisFunction{normalise} : \IdrisType{List} \IdrisType{(}\IdrisType{TTName}\IdrisType{,} \IdrisType{Binder} \IdrisType{TT}\IdrisType{)} -> \IdrisType{TT} -> \IdrisType{Elab} \IdrisType{TT}}\\
Normalizes\footnote{\fn{normalise} is spelled the British way, since most Idris development happens in the UK.} a typed term under a given environment.
\item\texttt{\IdrisFunction{lookupTy} : \IdrisType{TTName} -> \IdrisType{Elab} (\IdrisType{List} \IdrisType{(}\IdrisType{TTName}\IdrisType{,} \IdrisType{NameType}\IdrisType{,} \IdrisType{TT}\IdrisType{)})}\\
Looks up the type of the given name and returns the ones it finds in a list, in case the name is ambiguous.
\end{itemize}

Observe that in most of these functions, for inputs we use \ty{Raw}, the
untyped core language terms, and results are in \ty{TT}, the typed core
language terms.  This is because untyped core language terms are easier to
write for the tactic users, and type-checking them in the elaborator is easy.

Now let's define a function using elaborator reflection.
Let's define a polymorphic identity function.

\begin{Verbatim}[framesep=2mm, label=\footnotesize{\normalfont{Idris}}, labelposition=topline]
\fn{id} : (\bn{a} : \ty{Type}) -> \bn{a} -> \bn{a}
\fn{id} = \kw{%runElab} (\kw{do} \fn{intro} \kw{`\{\{}ty\kw{\}\}}
                  \fn{intro} \kw{`\{\{}a\kw{\}\}}
                  \fn{fill} (\dt{Var} \kw{`\{\{}a\kw{\}\}})
                  \fn{solve})
\end{Verbatim}

For anyone familiar with Coq, this will look very similar to a normal Coq
proof. First we take the type as an argument, and then a value of that type,
and we return the same value. Elaborator reflection proofs are a bit more
unpolished compared to Coq proofs because of the core language terms,
quasiquotation and special name syntax, but it is essentially very similar,
hence the name ``tactics'' we use to refer to monadic \Elab\ actions.

Let's prove something not as trivial this time. This time we want to prove that
$(\forall n \in \mathbb{N})\ n = n + 0$, for our definition of addition. Since that requires more complex tactics like induction, we will import the Pruviloj library.\cite{davidphd}

\begin{Verbatim}[framesep=2mm, label=\footnotesize{\normalfont{Idris}}, labelposition=topline]
\fn{nPlusZero} : (\bn{n} : \ty{Nat}) -> \bn{n} = \fn{plus} \bn{n} \dt{0}
\fn{nPlusZero} = \kw{%runElab} (\kw{do} \fn{intro} \kw{`\{\{}n\kw{\}\}}
                         \fn{induction} (\dt{Var} \kw{`\{\{}n\kw{\}\}})
                         \fn{compute}
                         \fn{reflexivity}
                         \fn{compute}
                         \fn{attack}
                         \fn{intro} \kw{`\{\{}n1\kw{\}\}}
                         \fn{intro} \kw{`\{\{}indHyp\kw{\}\}}
                         \fn{rewriteWith} (\dt{Var} \kw{`\{\{}indHyp\kw{\}\}})
                         \fn{reflexivity}
                         \fn{solve})
\end{Verbatim}

The proof proceeds as such: we first take in the argument \bn{n}, and then do
an induction on \bn{n}. Because of the way induction works in Pruviloj, we have
to simplify the goal using \fn{compute}\footnote{Its Coq equivalent would be
\texttt{simpl}.}.
And for the base case the goal is just proving \dt{0} \dt{=} \dt{0}.
For the inductive step, we have to restructure the goal with \fn{attack} and
then reintroduce the input and then introduce the induction hypothesis. Then we
rewrite the goal with the induction hypothesis and then the goal becomes
trivial. Understanding this proof completely is not crucial for this thesis,
but if you want to fully comprehend \fn{attack} and {solve}, you can refer to
\cite{elabref}.

Finally let's do an example for a declaration with elaborator reflection:

\begin{Verbatim}[framesep=2mm, label=\footnotesize{\normalfont{Idris}}, labelposition=topline]
\IdrisKeyword{%runElab} (\IdrisKeyword{do} \IdrisFunction{\IdrisFunction{declareType}} (\IdrisData{\IdrisData{Declare}} \IdrisKeyword{`\{\{}n\IdrisKeyword{\}\}} \IdrisData{\IdrisData{[]}} \IdrisKeyword{`(\IdrisType{\IdrisType{Nat\IdrisKeyword{)}}}})
             \IdrisFunction{\IdrisFunction{defineFunction}} (\IdrisData{\IdrisData{DefineFun}} \IdrisKeyword{`\{\{}n\IdrisKeyword{\}\}}
                               \IdrisData{\IdrisData{[\IdrisData{\IdrisData{MkFunClause}}}} (\IdrisData{\IdrisData{Var}} \IdrisKeyword{`\{\{}n\IdrisKeyword{\}\}}) \IdrisKeyword{`(\IdrisData{\IdrisData{Z\IdrisKeyword{)\IdrisData{\IdrisData{]}}}}}}))
\end{Verbatim}

The example above first declares that \fn{n} will have the type \ty{Nat}.
Then it defines it as \texttt{\IdrisFunction{n} = \IdrisData{Z}}.

Now that we have seen different use cases for elaborator reflection, we can move on to the
design of the edit-time tactics feature.


% Previous versions of Idris had a tactic based prover\footnote{As of Idris
% 1.1.1 it is still available, with the warning that it will be deprecated in the
% future versions.}, which embedded the proof tactics in a Haskell monad in the
% implementation.\cite{elabref}

% what sorts of things stay, what sorts of things go away with elaboration
% mcbride dependent pattern matching? (Epigram paper?)
% standard eliminators?
% (a few constructions on constructors: injectivity and disjointness)
% icfp 2016 - Jesper Cockx, doing that without axiom K
% how to get rid of dep pattern matching and turning it into induction principle

% \section{Design}\label{sec:design}

When the user invokes an editor action, the editor has to tell the Idris
compiler what to run. Since the editor and the compiler run in different processes,
for each interaction the editor has to send a message to the compiler,
and the compiler has to send a message back to the editor.
These messages are formatted as \citeauthor{mccarthy}'s \sexp{}s~\citep{mccarthy}.

The \Elab{} monad in Idris primitively keeps track of a state
involving a potentially incomplete expression, its type, and any new declarations generated as
side effects during elaboration.
When an \Elab{} script is executed, the incomplete expression is expected to have been completed.
Because these updates to the expression occur via side effects, elaborator reflection scripts have the type \mbox{\ty{Elab ()}}. Since the
desired metaprogramming effects are captured by the elaboration state, there is
nothing interesting to return.

However, \Elab{} scripts that are used as editor actions are not able to effect changes to the program by modifying the elaboration state, because the contents of the text editor are not part of the state.
Thus, editor actions return their results explicitly, and the serialized results are sent to the editor.

If an editor action needs to send back an expression to the editor, then the
action should have the return type \mbox{\ty{Elab TT}}, where \ty{TT} is the type of
quoted core Idris terms.
Similarly, if a user needs to define an action that creates a function definition,
then the action that does that should have the return type \mbox{\ty{Elab FunDefn}},
where \ty{FunDefn} is the type of quoted function definitions.
A simple editor action that only needs to send a number back
to the editor should return an \mbox{\ty{Elab Nat}}, where \ty{Nat} is the
type of natural numbers.

\subsection{The \Editorable{} Type Class}
\label{ssec:editorable}

The problem with allowing editor actions to return inhabitants of any
type is that the compiler cannot serialize values of arbitrary types
as \sexp{}s.  In order
to give users the power to define how each type should
be represented as \sexp{}s, we define a type class\footnote{In Idris,
  type classes are called \emph{interfaces} and instances are called
  \emph{implementations}.}  called \Editorable{}, which outlines what
the compiler needs to know about a type to be able to serialize and
deserialize values of that type.

\begin{figure}[H]
\begin{Verbatim}
\kw{interface} \ty{Editorable} \bn{a} \kw{where}
  \fn{fromEditor} : \ty{SExp} -> \ty{Elab} \bn{a}
  \fn{toEditor} : \bn{a} -> \ty{Elab} \ty{SExp}
\end{Verbatim}
\caption{Definition of the \Editorable{} type class.}
\label{code:editorable}
\end{figure}

Whenever users want to inform the compiler about the \sexp{}
representation of values of a type, they have to define an instance of the
\Editorable{} type class. Later when a user runs an editor action from an
editor, the \Editorable{} instances are used for communication via
\sexp{}s.

\subsubsection{Some \Editorable{} Instances}

The collection of atoms in Idris's \sexp{}s already includes many
primitive types, such as \ty{String}.  Deserializing a string succeeds
when provided with a string, and fails otherwise. The message thrown on failure can be a non-trivial list structure, which allows Idris's pretty printer to be used to render substrings, but here we elide the concrete messages and focus on the successful cases.
Serialization tags
the atom appropriately.

\begin{figure}[H]
\begin{Verbatim}
\kw{implementation} \ty{Editorable} \ty{String} \kw{where}
  \fn{fromEditor} (\dt{StringAtom} \bn{s}) = \fn{pure} \bn{s}
  \fn{fromEditor} \bn{x} = \fn{fail} \dt{[}\cm{\{- elided -\}}\dt{]}
  \fn{toEditor} \bn{x} = \fn{pure} (\dt{StringAtom} \bn{x})
\end{Verbatim}
\caption{\ty{String} instance of the \ty{Editorable} type class.}
\label{code:editorableString}
\end{figure}

Inductive types, such as \mbox{\ty{Maybe} \bn{a}}, can be represented as lists in which the first element is a tag specifying the chosen constructor.
For instance, \mt{\dt{Just "abc"}} can be represented as \mt{(\dt{:Just "abc"})}, a list
\sexp{} with a symbol atom as the first element and then the \sexp{}
representation of a string, and \dt{Nothing} can be represented as
the symbol \dt{:Nothing}. This can be implemented as follows:

\begin{figure}[H]
\begin{Verbatim}
\kw{implementation} \ty{Editorable} \bn{a}
            => \ty{Editorable} (\ty{Maybe} \bn{a}) \kw{where}
  \fn{fromEditor} (\dt{SExpList [SymbolAtom "Nothing"]}) =
    \fn{pure} \dt{Nothing}
  \fn{fromEditor} (\dt{SExpList [SymbolAtom "Just"}, \bn{x}\dt{]}) =
    \kw{do} \bn{x}' <- \fn{fromEditor} \bn{x}
       \fn{pure} (\dt{Just} \bn{x}')
  \fn{fromEditor} \bn{x} = \fn{fail} \dt{[}\cm{\{- elided -\}}\dt{]}
  \fn{toEditor} (\dt{Just} \bn{x}) =
    \kw{do} \bn{x'} <- \fn{toEditor} \bn{x}
       \fn{pure} (\dt{SExpList} \dt{[}\dt{SymbolAtom} \dt{"Just"}, \bn{x'}\dt{]})
  \fn{toEditor} \dt{Nothing} =
    \fn{pure} (\dt{SExpList [SymbolAtom "Nothing"]})
\end{Verbatim}
\label{code:editorableMaybe}
\caption{\ty{Maybe} instance of the \ty{Editorable} type class.}
\end{figure}

The idea that is introduced here can be used to define an \Editorable{}
instance for any data type that has exported constructors. Constructors that do
not take any argument are represented as symbol atoms, and the ones that do
take arguments are represented as a list \sexp{}, in which the first element is
a symbol atom and the other elements represent the arguments that the
constructor takes. We will call this the \emph{constructor-based \sexp{}
representation} of a type.

It is not, however, possible to use the constructor-based representation for every type.
In particular, functions and infinite coinductive datatypes do not, in general, admit finite serializations.

In other cases, the constructor-based representation requires too much work to encode and decode in editors.
For instance, Idris names have a rich structure, but users know them by their syntax rather than by their internal representation.
The \Editorable{} instance for the type of quoted Idris
names, \ty{TTName}, which appeared in \autoref{fig:motivating-example}, represents names using their user-facing syntax.
For instance, the Idris name \fn{Prelude.Bool.not}, which has the
data type representation \mt{\dt{NS "not" ["Bool"}, \dt{"Prelude"]}}, is
represented by a string atom \sexp{}, namely \dt{"Prelude.Bool.not"}.

\subsubsection{Primitive \Editorable{} Instances}
\label{sssec:primitiveEditorable}

The \sexp{} representations of quoted Idris code, such as \TT{}  and
\TyDecl{}, are the most challenging ones. These types mirror the internal
representation of Idris's core language, but they are ordinary inductive data
types defined in Idris, which means that the constructor-based representation
suffices to represent them.

However, that representation is not particularly convenient for
extending editors.  The constructor-based representation would be an
abstract syntax tree of the \textsf{TT} representation of an Idris
expression. Users, however, work with the concrete syntax of Idris
itself. When they use editor actions, they expect to see concrete
syntax put back into the file, and converting from \TT{} to concrete
Idris syntax requires a lot of code that should not be duplicated in
each editor when it already exists in the Idris compiler. Therefore,
for these core datatypes, the editor sends and receive concrete
syntax.

On the other hand, if the compiler receives concrete syntax and needs to run \Elab{}
actions on that, there are many missing steps in between, most important of
which is elaboration from a high-level language to the core language.
Similarly, if the compiler needs to send back concrete syntax after
running \Elab{} actions, then it needs to reverse all those steps.
In other words, there is a colossal gap between concrete syntax and the core
language that needs to be bridged, and this task can be delegated to the
\Editorable{} type class.

When the \sexp{} received by the compiler is a string atom that is a
piece of Idris code, i.e.\ concrete syntax, \fn{fromEditor} should parse that
string into a high-level language term, and then elaborate that into a core
language term. Only after that can the compiler run the \Elab\ editor action.
Similarly, when the \Elab\ action finishes, \fn{toEditor} should convert core
language terms into the high-level language terms, a process called
\emph{delaboration} in the Idris compiler.
Then, the compiler should invoke the pretty printer to get concrete syntax that
represents that term. The resulting string can be sent back from the compiler
to the editor as a string atom \sexp{}.

Bridging this gap requires an \Editorable{} instance for \TT{} that does parsing,
elaboration, conversion from the core language to the surface language, and pretty printing.
Rather than reimplementing this from scratch in Idris itself, we extended \Elab{}
to expose these features of the compiler as primitives, following \citeauthor{barzilayphd}'s program of \emph{direct reflection}~\citep{barzilayphd}.
In particular, these primitives are used to define the instances of \Editorable{}
for the core language types like \TT{}, \TyDecl{} and \FunDefn{}.
Hard-coding the \Editorable{} instances of \ty{TT}, \ty{TyDecl},
\ty{DataDefn}, \ty{FunDefn}, and \ty{FunClause} into the compiler
allows  by making use
of the already existing compiler implementations of the steps listed above.

To achieve this, the existing \Elab{} monad needs to be extended with
primitives that go through the steps mentioned above.  One \Elab{} primitive
for \fn{fromEditor} and one for \fn{toEditor} suffice; polymorphic
primitives constrained by an indexed family provide a principled
way to manage the primitive instances of \Editorable{}.

\begin{figure}[H]
\begin{Verbatim}
\kw{data} \ty{HasPrim} : \ty{Type} -> \ty{Type} \kw{where}
  \dt{HasTT}        : \ty{HasPrim} \ty{TT}
  \dt{HasTyDecl}    : \ty{HasPrim} \ty{TyDecl}
  \dt{HasDataDefn}  : \ty{HasPrim} \ty{DataDefn}
  \dt{HasFunDefn}   : \ty{HasPrim} (\ty{FunDefn} \ty{TT})
  \dt{HasFunClause} : \ty{HasPrim} (\ty{FunClause} \ty{TT})
\end{Verbatim}
\caption{Definition of the \ty{HasPrim} predicate in Idris.}
\end{figure}


\begin{figure}
\begin{Verbatim}
\fn{prim__fromEditor} : \ty{HasPrim} \bn{a} -> \ty{SExp} -> \ty{Elab} \bn{a}
\fn{prim__toEditor} : \ty{HasPrim} \bn{a} -> \bn{a} -> \ty{Elab} \ty{SExp}
\end{Verbatim}
  \caption{The new \Elab\ primitives.}
\label{code:newElabPrims}
\end{figure}

Figure~\ref{code:newElabPrims} uses \ty{HasPrim} to describe the new \Elab{} primitives.
Using these two primitives, the \Editorable{} instances for the core language
types all look alike; an example can be seen in figure~\ref{fig:editorablePrim}.

\begin{figure}
\begin{Verbatim}
\kw{implementation} \ty{Editorable} \ty{TT} \kw{where}
  \fn{fromEditor} \bn{x} = \fn{prim__fromEditor} \dt{HasTT} \bn{x}
  \fn{toEditor} \bn{x} = \fn{prim__toEditor} \dt{HasTT} \bn{x}
\end{Verbatim}
\caption{An \Editorable{} instance depending on the new primitives.}
\label{fig:editorablePrim}
\end{figure}

\subsection{How the Compiler Uses \Editorable{} for Communication}
\label{ssec:communication}

We have extended Idris's IDE protocol to support an additional message that
represents an invocation of a custom editor action.
This message includes the name of the custom action and a list of arguments,
and its reply contains Idris's response.

When the editor fires up Idris in IDE mode and loads the file, then it can send
a custom action message to Idris. If the compiler receives such a message from
the editor, it looks up the name and type of the editor action from the
context. From types of the arguments of the \Elab{} action, it can find the
necessary \Editorable{} instances and use the \fn{fromEditor} definitions in them
to parse the \sexp{}s into Idris values. If the number of arguments in the
action type and the argument list match, and all arguments can be parsed
without any errors, then the compiler can run the \Elab{} action, and use
\fn{toEditor} to serialize the output, and send it back to the editor.

The compiler can use \Elab{} actions whose arguments and return type have
\Editorable{} instances as custom editor actions. The \fn{easy} action from
figure \ref{fig:motivating-example} is an example of this, and its usage can be
seen in figure \ref{fig:motivating-example-exec}.
When the user puts the cursor on \hole{ex1\_impl} and invokes
\fn{idris-easy} in their Emacs session, Emacs sends a message to the compiler
that specifies that it wants to run \fn{easy}, and provides a list of
arguments, \mt{(\fn{list} \dt{"ex1\_impl"})}, which is a singleton list
containing a string atom. When Idris receives this message, it looks up the
name \fn{easy} from the context and finds out that it has the type
\mt{\ty{TTName} -> \ty{Elab TT}}. Therefore it looks up the \Editorable{}
instance of \ty{TTName} and runs its \fn{fromEditor} implementation on \dt{"ex1\_impl"}, which
results in the Idris name for \hole{ex1\_impl}. Then the compiler can execute
\fn{easy} and get a core language term \dt{()} as a result. Since core
language terms are represented by the \TT{} type, Idris has to find the
\Editorable{} instance of \TT{} and run its implementation of \fn{toEditor} on that term,
which produces an \sexp{} to be sent back to the editor.

\subsection{Using \Editorable{} in Type-Checking}
\label{ssec:typechecking}

The motivation behind the \ty{Editorable} type class is twofold:
\begin{enumerate}
\item to use the \fn{fromEditor} and \fn{toEditor} definitions to serialize
  and deserialize data before and after an \Elab{} action is run; and
\item to check whether a given \Elab{} action is suitable to be used as an
  editor action.
\end{enumerate}

The first motivation is already covered in the previous sections.
When Idris encounters a definition that is tagged with the \kw{\%editor}
keyword as an editor action, it first elaborates the type.
The next step is to check whether this type is suitable as an editor action.
It does this by ensuring that each argument type has an \Editorable{} instance,
that the return type has \Elab{} at its head, and that the argument to \Elab{}
is also \Editorable{}.
Note that this rules out dependent types for editor
actions---section~\ref{sssec:universeEncoding} discusses a potential way to lift
this restriction in the future.

%%% Local Variables:
%%% mode: latex
%%% TeX-master: "source"
%%% End:

% \section{Implementation Concerns}\label{sec:implementation}

\begin{figure*}[!b]
\begin{tikzcd}
  \mlnode{
    Haskell terms\\ \medskip \footnotesize such as the \ty{()} term\\
    \dt{()}
  }
  \arrow[rrr, "\text{\small reflection}" description, bend right=10]
  &  &  &
  \mlnode{
    Haskell representation of Idris core language terms\\ \medskip \scriptsize
    such as the \ty{Term} term\\
    \texttt{\dt{P} (\dt{DCon} \dt{0} \dt{0} \dt{False}) (\dt{UN} \dt{"MkUnit"}) (\dt{P} (\dt{TCon} \dt{0} \dt{0 \dt{False}}) (\dt{UN} \dt{"Unit"}) \dt{Erased})}
  }
  \arrow[lll, "\text{\small reification}" description, bend right=10] \\
  &  &  &\\
  \mlnode{
    Idris terms\\ \medskip \footnotesize such as the \ty{()} term\\
    \dt{()}
  }
  \arrow[r, "\text{\small elaboration}" description, bend right=50]
  &
  \mlnode{
    Idris core terms\\ \medskip \footnotesize such as the \ty{Unit} term\\
    \dt{MkUnit}
  }
  \arrow[l, "\text{\small delaboration}" description, bend right=50]
  \arrow[rr, "\text{\small quotation}" description, bend right=35]
  \arrow[uurr, "\text{\small \emph{internally represented as}}" description, dashed, bend left=25]
  & &
  \mlnode{
    Idris representation of the Haskell representation of Idris core language terms\\
    \medskip \scriptsize such as the \TT\ term\\
    \texttt{\dt{P} (\dt{DCon} \dt{0} \dt{0}) (\dt{UN} \dt{"MkUnit"})\\
    (\dt{P} (\dt{TCon} \dt{0} \dt{0}) (\dt{UN} \dt{"Unit"}) \dt{Erased})}
  }
  \arrow[ll, "\text{\small unquotation}" description, bend right=35]
  \arrow[uu, leftrightarrow, "\text{\small \emph{correspondence}}" description, dashed]
\end{tikzcd}
\caption{The relationship between reflection, reification, quotation,
  unquotation, elaboration and delaboration.}
\label{reflectionGraph}
\end{figure*}

% \TODO{The main goal of this section will be to provide enough information about the implementation so that someone else can add this feature to their dependently typed language. What were the big challenges? Think of things like the need for the local context to be preserved and how that gets associated with commands, as well as document the interesting challenges that arose.}

The overall design described in section~\ref{sec:design} is not
completely sufficient to implement extensible type-directed
editing. Some additional machinery proves to be necessary in practice.

\subsection{Local Contexts}
\label{ssec:localContext}

Expressions can be understood only in a local context that
explains the types, and sometimes the values, of their free
variables. Editor actions should have access to the local context of
bound variables in addition to the global definition context. The
design in section~\ref{sec:design}, however, has no means of providing
them with this local context.

For example, the editor might send an expression like \mt{\fn{not}
  \bn{a}} to the compiler, where \bn{a} is bound in the local context.
The compiler can parse this expression, but it cannot elaborate it,
because without the local context, \bn{a} is meaningless. In a call to
a custom editor action, expressions stand alone; they do not come with
a context. It can deal with \fn{not}, which is defined in the global
context, but when it comes to the local context, elaboration is doomed
to fail.

Editor actions have, thus far, been provided with their arguments
explicitly. However, local contexts do not have a concrete
representation in Idris's syntax, so they cannot be selected directly.

To solve this problem, we take advantage of the fact that lexical
contexts correspond to source spans. Each local binding form
has a defined scope; this scope corresponds to a production of the
abstract syntax tree that originates from a specific range of
positions in the editor buffer. We extended the protocol for custom
editor actions so that the editor sends a source position along with
the action name and its arguments.

Prior to our work, local context information was only available for holes, and
they were tied to names of the holes, not their source positions. To be able to
keep track of the correspondence between source positions of all expressions
and their local contexts, we extended the internal compiler state with an
interval map that connects ranges of source positions to the local
context that corresponds to the process of elaborating the expression
found in that range. We used the standard Haskell finger tree
implementation of interval maps~\cite{fingertrees}.
Entries in interval maps are accumulated during elaboration and saved, to be
used later in editor interaction.

When initializing the reflected elaboration monad prior to executing
an editor action, the local context is initialized with the one that
corresponds to the cursor location sent by the editor. The compiler
can use that information in the elaboration of terms depending on the
local context, such as \mt{\fn{not} \bn{a}}. This constrains
editor actions to have a single privileged source position; this
constraint has not proven difficult in practice, but it could be
lifted by making source positions \ty{Editorable} and providing any
number of them as ordinary arguments. Editor actions could then have a
primitive to enter the lexical scope corresponding to a source
position, and to check whether a source span is contained within a
particular scope.

We expect that the remembered association between source spans and
contexts will enable additional editor features, such as displaying
the local context as the user navigates a source file, but we have not
yet implemented these features.

\subsection{Hard-Coding \Editorable{} Instances}
\label{ssec:hardcodingEditorable}

Implementing hard-coded instances of the \Editorable{} type class in the
compiler is challenging to describe since there are many languages involved in
different ways. Idris's compiler is written in Haskell, hence there is a
Haskell data type that represents Idris syntax trees. Idris's
elaborator~\citep{idris} describes a core language that is smaller than Idris's
high-level language, there is also a Haskell data type that represents Idris
core syntax trees.

However, elaborator reflection~\citep{davidphd, elabref} provides new Idris
data types that correspond to the Haskell data types to represent Idris core
language terms.
Outlining how a metaprogramming feature is implemented introduces another layer of metadiscussion,
therefore it becomes difficult to use precise terminology.
Figure~\ref{reflectionGraph} describes the relationship between the different
languages and representations and spells out the specific names for moving
from one to another.

During the execution of \fn{prim\_\_fromEditor} in the elaborator, it is given
\bn{arg}, a Haskell representation of an Idris core language term representing
an \sexp{}, and \bn{ty}, a Haskell representation of an Idris type.
When the elaborator finishes running \fn{prim\_\_fromEditor}, it should create
the Haskell representation of an Idris core language term, which should have the
type \bn{ty}. The elaborator should have a case for each primitive
\Editorable{} implementation. For brevity, we will only consider the case in
which \bn{ty} corresponds to the Idris type \TT{}. Then the elaborator should

\begin{enumerate}[leftmargin=10pt]
  \item reify \bn{arg} to get a Haskell \sexp{} and make sure it is a string atom;
  \item parse the string inside the \sexp{} and get a high-level language term;
  \item traverse the high-level language term and resolve namespaces for names, for when it is unambiguous;
  \item elaborate the new high-level language term into a core language term, using the local context obtained through section~\ref{ssec:localContext};
  \item reflect the core language term, in order to create a core language term that represents an Idris term of the type \TT{}; and finally
  \item normalize the reflected term to get a syntax tree in canonical form, and return it.
\end{enumerate}

During the execution of \fn{prim\_\_toEditor} in the elaborator, it is given
\bn{ty}, a Haskell representation of an Idris type, and \bn{arg},
a Haskell representation of an Idris core language term representing
a term of the type \bn{ty}.
When the elaborator finishes running \fn{prim\_\_toEditor}, it should create
the Haskell representation of an Idris core language term representing an
\sexp{}. The \TT{} case for \fn{prim\_\_toEditor} should

\begin{enumerate}[leftmargin=10pt]
  \item reify \bn{arg} to get a Haskell representation of an Idris core language term;
  \item delaborate and resugar the core language term into a high-level language term;
  \item use pretty printing to get a string that is a piece of code;
  \item create a string atom \sexp{} with that string;
  \item reflect the \sexp{} to get a Haskell term representing an Idris core language term representing an \sexp{}; and finally
  \item normalize the reflected term to get a syntax tree in canonical form, and return it.
\end{enumerate}

\noindent
The other primitive instances behave similarly.

%%% Local Variables:
%%% mode: latex
%%% TeX-master: "source"
%%% End:

% \section{Applications} \label{sec:applications}

In this section, we present a custom domain-specific editor action, an editor
action that is meant to replace a built-in Idris IDE mode feature, and an
editor action that improves on Idris's proof search mechanism in a specific
logic.

\subsection{Regular Expression Simplification}

One of the most promising benefits of our work is that it allows authors of
embedded domain-specific languages~\cite{dsel} (or eDSLs) to write domain-specific
custom editor actions that assist eDSL users.
One such language that most programmers are already familiar with is regular
expressions.

\begin{figure}[ht]
\begin{Verbatim}
\kw{data} \ty{Regex} = \dt{Empty}
           | \dt{Epsilon}
           | \dt{Lit} \ty{Char}
           | \dt{Concat} \ty{Regex} \ty{Regex}
           | \dt{Or} \ty{Regex} \ty{Regex}
           | \dt{Star} \ty{Regex}
\end{Verbatim}
\caption{Definition of regular expressions.}
\label{code:regex}
\end{figure}

Mirroring the formal definition of regular expressions, the Idris definition of
regexes has constructors for $\varnothing$, $\varepsilon$, literal
characters, concatenation ($\cdot$), alternation ($|$) and the Kleene star ($*$),
as seen in \autoref{code:regex}.

The most common usage of regular expressions is to determine whether a string is in its language.
For that, users would have to write regular expression literals using the
\ty{Regex} data type. If the user wants to check whether the string \textit{``aaa''} is in
the language of the regular expression $a*$, then they would call
\mt{\fn{accepts} (\dt{Star} (\dt{Lit 'a'})) \dt{"aaa"}}.

However, there is no guarantee that the user would write the regular expression
in its simplest form. Especially for more complex regular expressions, it is
easy to overlook simpler versions. For instance, a user might write the regex
term \mt{\dt{Or} \dt{Epsilon} (\dt{Star} (\dt{Lit 'a'}))}, representing
$\varepsilon | a*$, instead \mt{\dt{Star} (\dt{Lit 'a'})}, representing $a*$.
A custom editor action can perform this simplification automatically.

For reasons of space, we will not explain the actual simplification algorithm, since there are
many external sources such as \citet{ortizRegex} and \citet{harperRegex}
 that do, and it is not essential to understanding how the custom
editor action works.
We take the function \mt{\fn{simplify} : \ty{Regex} -> \ty{Regex}} as a
given, and proceed to describe how it can be used to construct an editor
action.\footnote{Full source code is available at
\url{http://github.com/joom/edit-time-tactics/tree/master/code/regex}.}

The custom editor action to simplify regexes should consume a regex, returning a
potentially simpler regex. However, when the editor sends expressions to the compiler,
it sends them as strings containing snippets of code, which are then parsed and
elaborated in the compiler.  The editor action should therefore have the type
\mt{\TT{} -> \Elab{} \TT{}}. The input and output are not regular expressions;
they are ASTs of Idris code that represents regular expressions.

Simplification, however, is a function from \ty{Regex} to \ty{Regex}, not from
\ty{TT} to \ty{TT}.
At the same time, regular expression simplification is vastly easier to implement
on actual \ty{Regex}es, rather than on their quotations.
To implement the editor action, \autoref{code:regexQuote} defines two functions \fn{unquote} and \fn{quote}
that converts between \TT{} and the constructors of \ty{Regex}. The
conversion from \ty{Regex} to \ty{TT} in the \fn{quote} function cannot fail,
since all regexes must have a core language representation. However, the
conversion from \TT{} to \ty{Regex} can fail in the \fn{unquote} function,
since if \fn{unquote} can be given any core language term, including ones that
represent ill-typed Idris terms, ones that have types other than \ty{Regex}, or \ty{Regex}es
that contain free variables.

\begin{figure}[ht]
\begin{Verbatim}
\fn{unquote} : \ty{TT} -> \ty{Elab} \ty{Regex}
\fn{unquote} \qt{\dt{Empty}} = \fn{pure} \dt{Empty}
\fn{unquote} \qt{\dt{Epsilon}} = \fn{pure} \dt{Epsilon}
\fn{unquote} \qt{\dt{Lit} \antiqt\bn{c}} = \kw{do} \bn{c'} <- \fn{unquote} \bn{c}
                       \fn{pure} (\dt{Lit} \bn{c'})
\fn{unquote} \qt{\dt{Concat} \antiqt\bn{x} \antiqt\bn{y}} = \kw{do} \bn{x'} <- \fn{unquote} \bn{x}
                             \bn{y'} <- \fn{unquote} \bn{y}
                             \fn{pure} (\dt{Concat} \bn{x'} \bn{y'})
\fn{unquote} \qt{\dt{Or} \antiqt\bn{x} \antiqt\bn{y}} = \kw{do} \bn{x'} <- \fn{unquote} \bn{x}
                         \bn{y'} <- \fn{unquote} \bn{y}
                         \fn{pure} (\dt{Or} \bn{x'} \bn{y'})
\fn{unquote} \qt{\dt{Star} \antiqt\bn{x}} = \kw{do} \bn{x'} <- \fn{unquote} \bn{x}
                        \fn{pure} (\dt{Star} \bn{x'})
\fn{unquote} \bn{t} = \fn{fail} \dt{[}\cm{\{- elided -\}}\dt{]}

\fn{quote} : \ty{Regex} -> \ty{TT}
\fn{quote} \dt{Empty} = \qt{\dt{Empty}}
\fn{quote} \dt{Epsilon} = \qt{\dt{Epsilon}}
\fn{quote} (\dt{Lit} \bn{c}) = \qt{\dt{Lit} \antiqt(\fn{quote} \bn{c})}
\fn{quote} (\dt{Concat} \bn{x} \bn{y}) =
  \qt{\dt{Concat} \antiqt(\dt{quote} \bn{x}) \antiqt(\fn{quote} \bn{y})}
\fn{quote} (\dt{Or} \bn{x} \bn{y}) = \qt{\dt{Or} \antiqt(\fn{quote} \bn{x}) \antiqt(\fn{quote} \bn{y})}
\fn{quote} (\dt{Star} \bn{x}) = \qt{\dt{Star} \antiqt(\fn{quote} \bn{x})}
\end{Verbatim}
\caption{Functions to convert between values of the \ty{Regex} type and their representation in the core language.}
\label{code:regexQuote}
\end{figure}

The functions \fn{unquote}, \fn{quote}, and \fn{simplify} provide all the
building blocks needed to define the custom editor action, which are combined in
\autoref{code:regexElab} to make up the necessary \Elab{} action.

The Emacs Lisp support code required to run the custom editor action gets the
region selected in the editor by the user, sends it as the first argument for
the \fn{simplifyInEditor} editor action, receives a response from the compiler,
and replaces the code in the response with the selected region.

\begin{figure}[H]
\begin{Verbatim}
\kw{\%editor}
\fn{simplifyInEditor} : \ty{TT} -> \ty{Elab} \ty{TT}
\fn{simplifyInEditor} \bn{t} =
  \kw{do} \bn{r} <- \fn{unquote} \bn{t}
     \fn{pure} (\fn{quote} (\fn{simplify} \bn{r}))
\end{Verbatim}
\hrulefill
\begin{Verbatim}
(\kw{defun} \fn{idris-simplify-regex} ()
  \dt{"Replace selection with simplified regex."}
  (\fn{interactive})
  (\kw{let*} ((\bn{regexp} (\fn{buffer-substring-no-properties}
                  (\fn{region-beginning})
                  (\fn{region-end})))
         (\bn{result} (\fn{idris-elab-edit}
                  \dt{"simplifyInEditor"} \bn{regexp})))
    (\fn{replace-region}
     (\fn{region-beginning}) (\fn{region-end}) \bn{result})))
\end{Verbatim}
\caption{Definition of \Elab{} action for regex simplification, and the
  necessary Emacs Lisp support code to run.}
\label{code:regexElab}
\end{figure}


\begin{figure}[ht]
\begin{Verbatim}
\kw{if} \fn{accepts} (\select{\dt{Or} \dt{Epsilon} (\dt{Star} (\dt{Lit 'a'}))}) \dt{"aaa"}
  \kw{then} \cm{\{- elided -\}} \kw{else} \cm{\{- elided -\}}
\end{Verbatim}
  \vspace{1em}
\begin{Verbatim}
\kw{if} \fn{accepts} (\dt{Star} (\dt{Lit 'a'})) \dt{"aaa"}
  \kw{then} \cm{\{- elided -\}} \kw{else} \cm{\{- elided -\}}
\end{Verbatim}
\caption{Before and after invoking regular expression simplification}
  \label{fig:regex-example}
\end{figure}

\autoref{fig:regex-example} shows an example editor session using the regex
simplification editor action. The user selects a region containing an
expression and executes the Emacs Lisp function, which replaces the selected
expression with the simplified version of the same regex.

\subsection{Reimplementing the Built-In ``Add Clause'' Action}\label{sec:addClause}

Idris's editor modes support a built-in editor action called ``Add initial match
clause to type declaration.'' When the cursor is on the type signature of a
function that does not have any clauses, the user can run this editor action
and get an initial pattern clause for the function.
For instance, invoking the command on the declaration
\begin{Verbatim}
\fn{copy} : (\bn{n} : \ty{Nat}) -> \bn{a} -> \ty{Vect} \bn{n} \bn{a}
\end{Verbatim}
results in the clause
\begin{Verbatim}
\fn{copy} \bn{n} \bn{x} = \hole{copy_rhs}
\end{Verbatim}
which has a bound variable for each explicit argument in \fn{copy}'s type.

There is no longer any need to implement this feature in Haskell as
part of the compiler. This section describes the implementation of an
editor action in Idris itself that generates initial clauses for top-level
type declarations without implicit arguments or interface
constraints. A version that handles these additional features is
longer, but involves no additional concepts. The complete Idris code that
implements this editor action can be seen in \autoref{code:addClause}.

\begin{figure}[ht]
\begin{Verbatim}
\fn{collectTypes} : \ty{TT} -> (\ty{List} \ty{TT}, \ty{TT})
\fn{collectTypes} (\dt{Bind} _ (\dt{Pi} \bn{ty} _) \bn{t}) =
  \kw{let} \dt{(}\bn{xs}\dt{,} \bn{t'}\dt{)} = \fn{collectTypes} \bn{t} \kw{in}
  (\bn{ty} \fn{::} \bn{xs}, \bn{t'})
\fn{collectTypes} \bn{t} = \dt{([]}, \bn{t})

\kw{%editor}
\fn{addClause} : \ty{TTName} -> \ty{Elab} (\ty{FunClause} \ty{TT})
\fn{addClause} \bn{n} =
  \kw{do} \bn{ty} <- \fn{getType} \bn{n}
     \bn{env} <- \fn{getEnv}
     \bn{ty'} <- \fn{normalise} \bn{env} \bn{ty}
     \kw{let} \dt{(}\bn{argTys}\dt{,} \bn{retTy}\dt{)} = \fn{collectTypes} \bn{ty'}
     \bn{argNames} <- \fn{traverse} (\fn{const} \fn{fresh}) \bn{argTys}
     \kw{let} \bn{lhsUntyped} =
       \fn{foldl} \dt{RApp} (\dt{Var} \bn{n}) (\fn{map} \dt{Var} \bn{argNames})
     \dt{(}\bn{lhsTyped}\dt{,} _\dt{)} <- \fn{check} \bn{env} \bn{lhsUntyped}
     \bn{holeName} <- \fn{fresh}
     \kw{let} \bn{rhs} = \dt{Bind} \bn{holeName} (\dt{GHole} \bn{retTy}) (\dt{V} \dt{0})
     \fn{pure} (\fn{MkFunClause} \bn{lhsTyped} \bn{rhs})
\end{Verbatim}
\caption{Implementation of the editor action for ``add clause''.}
\label{code:addClause}
\end{figure}

The \fn{collectTypes} function takes a type and dissects it into components, and returns
a pair of the list of inputs and the output type. For instance,
calling \fn{collectTypes} with the input \mt{\kw{\`{}(}\ty{Nat} -> \ty{Bool} -> \ty{String}\kw{)}}
returns \mt{\dt{(}\dt{[}\kw{\`{}(}\ty{Nat}\kw{)}\dt{,}
\kw{\`{}(}\ty{Bool}\kw{)}\dt{]}\dt{,}\kw{\`{}(}\ty{String}\kw{)}\dt{)}}.

The \fn{addClause} action only takes one input, which is the name of
the function declaration for which an initial clause has been
requested.  Using this name, it looks up the type of that function,
normalizes the type, and gets its components using \fn{collectTypes}.
The list of input types is named \bn{argTys}, and the output type is
named \bn{retTy}. For each member of \bn{argTys}, it generates a
new user-accessible name using \fn{fresh}. A more featureful
implementation would attempt to preserve names from the type
signature, only generating fresh names when the user had not provided
a name or in the presence of shadowing.

A pattern match clause consists of a left-hand side, which is an
application of the function being defined to either constructors or
pattern variables, and a right-hand side, which is the expression that
results when the pattern on the left-hand side matches.  Having found
names for each pattern variable, the left hand side of the initial
clause is constructed by applying the function being defined, using
\dt{RApp}.  The \dt{Var} constructor injects names into terms.  The
right hand side of the initial clause should consist only of a hole
for the user to fill in, indicated by the \dt{GHole} term constructor.
Because Idris holes are binding forms, the de~Bruijn index \dt{0}
refers back to this new hole.


The Emacs Lisp code necessary to run \fn{addClause} as a custom editor
action is almost identical to the existing add-clause editor
action.  The only difference is that the call to the primitive
add-clause editor action in the IDE protocol is replaced by a call to
\fn{addClause}.

Using this editor action on the declaration
\begin{Verbatim}
\fn{copy} : (\bn{n} : \ty{Nat}) -> \bn{a} -> \ty{Vect} \bn{n} \bn{a}
\end{Verbatim}
results in an initial match clause
\begin{Verbatim}
\fn{copy} \bn{a} \bn{b} = \hole{c}
\end{Verbatim}
which was just as expected. However, this new version is much more
readily extensible by users.

\subsection{A Theorem Prover for Intuitionistic Propositional Logic}\label{sec:hezarfen}

In this section, we describe the procedure Hezarfen,\footnote{The name is
  pronounced ``has are fan'', and it means
    polymath in Turkish. Source code is available at
    \url{http://github.com/joom/hezarfen}.} which can decide intuitionistic
propositional logic theorems, similar to Coq's \texttt{tauto} tactic.
This procedure will be based on Dyckhoff's LJT~\cite{ljt} and its Haskell
implementation Djinn~\cite{djinn}, which generates Haskell expressions
for a given type.
Djinn is a standalone program that takes commands
interactively, and when it generates an expression it prints it on the screen.
Instead, Hezarfen is a library that provides an \Elab\ action
that can be used as a tactic in proofs, and a custom editor action to be run
when the built-in proof search mechanism does not suffice.

The prover consists of mutually recursive functions that try to break
the goal type down into components, recursively finds terms that satisfy the
components, and then glues them together based on the initial matched goal
type.

Later in the prover there is also a term simplifier, similar to Haskell's
pointfree style
converter.\footnote{\url{http://hackage.haskell.org/package/pointfree}} It performs
\mbox{$\eta$-reduction}, removes unused \kw{let} bindings, and similar
simplification steps repeatedly until it reaches a fixed point.
However, this simplifier is tailored for Hezarfen's proof terms; it is not
general-purpose.  The necessity of further work on a general purpose one is
discussed in section~\ref{sssec:simplification}.

\begin{figure}[h]
\begin{Verbatim}
\fn{comm} : (\bn{a}, \bn{b} : \ty{Type}) -> \ty{Either} \bn{a} \bn{b} -> \ty{Either} \bn{b} \bn{a}
\fn{comm} = \hole{comm_impl}
\end{Verbatim}
  \vspace{1em}
\begin{Verbatim}
\fn{comm} : (\bn{a}, \bn{b} : \ty{Type}) -> \ty{Either} \bn{a} \bn{b} -> \ty{Either} \bn{b} \bn{a}
\fn{comm} = \textbackslash\bn{x}, \bn{y} => \fn{either} \dt{Right} \dt{Left}
\end{Verbatim}
\caption{Before and after invoking Hezarfen}
  \label{fig:hezarfen-example}
\end{figure}

Figure~\ref{fig:hezarfen-example} displays the results of
executing this editor action on a hole:
Hezarfen finds a term with the desired type in which \fn{either} is a non-dependent eliminator for \ty{Either}.
Observe that the type of \fn{comm} corresponds to the proposition
$(a \lor b) \to (b \lor a)$, and by finding the term, Hezarfen proves
the proposition.

%%% Local Variables:
%%% mode: latex
%%% TeX-master: "source"
%%% End:

% \section{Future Work}

The story of dependently typed languages that can be reprogrammed in
themselves is only just beginning. Further developments can increase
the convenience and reliability of Idris's editor actions.

\subsection{Proof Simplification}
\label{sssec:simplification}

\citet{elabref} showed that elaborator reflection can be used as a
tactic language for interactive theorem proving. It is possible to use
\Elab{} tactics to define custom editor actions and reuse existing
proof automation efforts directly from the editor.

\Elab{} tactics generate a proof term during elaboration, but the
artifact is only a call to the tactic, which allows users to ignore
the proof terms generated by the tactics. However gigantic or hideous
the proof terms are, readers of the code will only see that the
tactics satisfy the goal, while the proof term
itself remains hidden. Many well-known proof automation procedures, such as Coq's
\mt{omega}~\cite{omega}, make use of this fact to hide large, complicated proof terms.  However, when using
\Elab{} tactics to define custom editor actions, the result of the
action is an expression that is visible to the user. Thus, brevity and
readability are desirable qualities in the proof terms generated by
those tactics.  Requiring all tactic authors to simplify their own
expressions qualities is burdensome, and it hampers the reuse of
existing tactics.  If there were a generic mechanism to simplify and
minimize generated proof terms, and even write them in a way that
makes use of dependent pattern matching, then existing tactics would become
much more useful for implementing editor actions. Ideally, a finished program that was written with
custom editor actions based on proof automation should be indistinguishable from one written without.

\subsection{A Universe of Actions}
\label{sssec:universeEncoding}

Section~\ref{ssec:typechecking} described how the Idris compiler
checks whether all components of an editor action type have an
instance of the \Editorable{} type class. However, it is not necessary
to implement this as an additional step during elaboration: it would suffice to
encode the allowed types of editor actions as a universe à la
Tarski~\cite{genericDep}.  The universe would include only those functions whose domains
have \Editorable{} instances and whose ranges are in the universe, as well as
other types that have \Editorable{} instances.
Figure~\ref{code:universe} demonstrates an implementation of this universe.


\begin{figure}[H]
\begin{Verbatim}
\kw{data} \ty{Act} : \ty{Type} \kw{where}
  \dt{Done} : (\bn{a} : \ty{Type}) -> \ty{Editorable} \bn{a} => \ty{Act}
  \dt{Arg} : (\bn{a} : \ty{Type}) -> \ty{Editorable} \bn{a} =>
     (\bn{ty} -> \ty{Act}) -> \ty{Act}

\fn{actTy} : \ty{Act} -> \ty{Type}
\fn{actTy} (\dt{Done} \bn{ty}) = \ty{Elab} \bn{ty}
\fn{actTy} (\dt{Arg} \bn{ty} \bn{f}) = (\bn{v} : \bn{ty}) -> \fn{actTy} (\bn{f} \bn{v})
\end{Verbatim}
\caption{Universe encoding of types feasible to be treated as editor actions.}
\label{code:universe}
\end{figure}

Figure~\ref{code:universe-example} shows how the type of the \fn{easy} editor
action from figure~\ref{fig:motivating-example} would change with this encoding.

\begin{figure}[H]
\begin{Verbatim}
\fn{easy} : \fn{actTy} (\dt{Arg} \ty{TTName} (\textbackslash\bn{n} => \dt{Done} \TT))
\fn{easy} \bn{n} = \cm{\{- elided, same as before -\}}
\end{Verbatim}
\caption{\fn{easy} rewritten as a universe encoded editor action.}
\label{code:universe-example}
\end{figure}

Observe that \mt{\fn{actTy} (\dt{Arg} \ty{TTName} (\dt{Done} \TT))} evaluates
to \mt{\ty{TTName} -> \Elab{} \TT}, therefore the definition of \fn{easy} does
not have to change.

The most important outcome of this change would be the increase in the
expressiveness of editor action types. The current implementation rules out
dependently typed editor actions, while this universe encoding would allow
them. Figure ~\ref{code:universe-dependent} shows a
hypothetical editor action that takes a vector of some length that contains
function names and returns a vector of the \emph{same} length that contains the types
found for the function names.

\begin{figure}[H]
\begin{Verbatim}
\fn{getTypes} : \fn{actTy} (\dt{Arg} \ty{Nat} (\textbackslash\bn{n} =>
                    \dt{Arg} (\ty{Vect} \bn{n} \ty{TTName}) (\textbackslash{}_ =>
                      \dt{Done} (\ty{Vect} \bn{n} (\ty{Maybe} \ty{TT})))))
\fn{getTypes} \bn{n} \bn{v} =
  \fn{for} \bn{v} (\textbackslash\bn{n} =>
    \kw{do} \bn{l} <- \fn{lookupTy} \bn{n}
       \kw{case} \bn{l} \kw{of}
         \dt{[(}_\dt{,} _\dt{,} \bn{ty}\dt{)]} => \fn{pure} (\dt{Just} \bn{ty})
         _ => \fn{pure} \dt{Nothing})
\end{Verbatim}
\caption{A dependently typed editor action that would be possible with the universe encoding.}
\label{code:universe-dependent}
\end{figure}

% I'm not sure if we need this paragraph... we can remove it later
However, writing editor actions with dependent data types would require writing
more complex \Editorable{} instances.  Figure ~\ref{code:editorable-vect} shows the
\Editorable{} instance for length-indexed vectors, which uses lists to denote
vectors and hence has to check if the lengths match in every deserialization.
\citet{foundations-dep-interop} provide a solution to this problem that could be
adopted in Idris.

\begin{figure}[H]
\begin{Verbatim}
\kw{implementation} \ty{Editorable} \bn{a}
            => \ty{Editorable} (\ty{Vect} \bn{n} \bn{a}) \kw{where}
  \fn{fromEditor} \{\bn{a}\} \{\bn{n}\} (\dt{SExpList} \bn{l}) =
    \kw{do} \bn{l'} <- \fn{traverse} (\fn{fromEditor} \{a = \bn{a}\}) \bn{l}
             \fn{<|>} \fn{fail} \dt{[}\cm{\{- elided -\}}\dt{]}
       \kw{case} \fn{decEq} (\fn{length} \bn{l'}) \bn{n} \kw{of}
         \dt{Yes} \bn{pf} =>
           \fn{pure} (\fn{replace} \{P = \textbackslash{}\bn{k} => \ty{Vect} \bn{k} \bn{a}\}
                         \bn{pf} (\fn{fromList} \bn{l'}))
         \dt{No} _ => \fn{fail} \dt{[}\cm{\{- elided -\}}\dt{]}
  \fn{fromEditor} _ = \fn{fail} \dt{[}\cm{\{- elided -\}}\dt{]}
  \fn{toEditor} \bn{v} = \fn{toEditor} (\fn{toList} \bn{v})
\end{Verbatim}
\caption{\Editorable{} instance for length-indexed vectors.}
\label{code:editorable-vect}
\end{figure}

\subsection{Surface-Language Syntax}

Editor actions presently accept and produce representations of \TT{},
rather than high-level Idris, which greatly simplifies the
implementation and maintenance of editor actions. For many
applications, this does not matter, because the \emph{meaning} of an
expression is more important than how it is written. In some cases,
however, this lack of expressive power might be a problem. For
instance, it is presently impossible to define an editor action that
converts a use of idiom brackets~\citep{Applicative} into the
equivalent \kw{do}-notation, as both expressions have the same
representation in the core language. In the future, it would be
interesting to explore representations of the syntax of high-level
Idris that are robust in the face of change and extension.



\section{Conclusion}\label{sec:conclusion}

In this paper, we extended the capabilities of the editor interaction mode of
Idris by allowing users to define new editor actions in Idris itself. We did
so through a metaprogramming technique that was introduced to Idris recently by
Christiansen and Brady~\cite{elabref}.

Editors communicate with the compiler via S-expressions, so we gave
users the power to dictate how a value of a given Idris type should
exactly be communicated; through the \ty{Editorable} interface users
are now able to define how a received S-expression should be parsed by
the compiler, and how the compiler should send the result as an
S-expression. To achieve this, we reflected the \ty{SExp} type to
Idris, and extended elaborator reflection by adding new \Elab\
primitives, with which we defined the \ty{Editorable} implementations
for Idris types representing the Haskell representation of Idris core
language terms. This demonstrates the value of directly reusing the
compiler's implementations.

We have demonstrated editor actions such as simple proof searches and
a DSL-specific action, as well as a demonstration of rewriting part of
Idris in itself. We hope that Hezarfen will eventually be a better
proof search than the built in action. We believe there is potential
to replace even more of the built-in editor actions with custom editor
actions written in Idris, such as case-splitting and lifting a hole
into a lemma. We can also add new general editor actions such as
renaming a binder, renaming a function within a file, pruning unused
arguments in a function, and so forth.

As the library of elaborator actions grows, more building blocks will
be available for custom editor actions. Even today, however, authors
of libraries and DSLs can include custom editor actions with their
packages, giving library and DSL authors access to power that was
previously reserved for compiler implementors.

If we are serious about type-driven interactive programming, we need
to give users the power to control not only their programming
language, but also their programming environment. Idris's editor
actions are one small step towards that goal.

%%% Local Variables:
%%% mode: latex
%%% TeX-master: "source"
%%% End:

\bibliography{paper}
\bibliographystyle{plain}
\end{document}

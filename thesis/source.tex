\documentclass[11pt, ma]{westhesis}
\setcounter{tocdepth}{4}
\setcounter{secnumdepth}{4}
\usepackage[utf8]{inputenc}
\usepackage{amsmath, amsfonts, amssymb, bussproofs, listings, float, upgreek, stmaryrd, epigraph, afterpage}
\usepackage[square,sort&compress,numbers]{natbib}
\usepackage[font=small]{caption}
\usepackage{fancyvrb}
\usepackage[usenames,dvipsnames]{xcolor}
% \usepackage[usenames,dvipsnames,monochrome]{xcolor} %black and white
\usepackage[T1]{tipa}
\usepackage{graphicx}
\graphicspath{ {images/} }
\usepackage{tgpagella}
\usepackage{inconsolata}
\usepackage{ragged2e}
\usepackage[hang, flushmargin]{footmisc}
\usepackage[backref=page,hyperfootnotes=false]{hyperref}
\usepackage{footnotebackref}
\definecolor{IdrisRed}{RGB}{166, 0, 2}
\definecolor{IdrisBlue}{RGB}{0, 0, 182}
\definecolor{IdrisGreen}{RGB}{13, 84, 8}
\definecolor{IdrisPurple}{RGB}{126, 2, 236}
\hypersetup{
  colorlinks = true,
  linkcolor = {blue},
  citecolor = {blue},
  urlcolor = {blue},
  anchorcolor = {blue}
}
\usepackage{tikz-cd}
\usepackage{adjustbox}
\usepackage{enumerate}

% Idris
% in REPL run `:pp latex 80 <expr>` to get the colored verbatim
\newcommand{\IdrisData}[1]{\textcolor{IdrisRed}{#1}}
\newcommand{\IdrisType}[1]{\textcolor{IdrisBlue}{#1}}
\newcommand{\IdrisBound}[1]{\textcolor{IdrisPurple}{#1}}
\newcommand{\IdrisFunction}[1]{\textcolor{IdrisGreen}{#1}}
\newcommand{\IdrisMetavar}[1]{\textcolor{cyan}{#1}}
\newcommand{\IdrisKeyword}[1]{{\textbf{#1}}}
\newcommand{\IdrisImplicit}[1]{{\itshape \IdrisBound{#1}}}
\usepackage{setspace, microtype}
\fvset{commandchars=\\\{\},formatcom=\singlespacing, frame=single}
\newcommand{\ty}[1]{\IdrisType{\texttt{#1}}}
\newcommand{\kw}[1]{\IdrisKeyword{\texttt{#1}}}
\newcommand{\fn}[1]{\IdrisFunction{\texttt{#1}}}
\newcommand{\dt}[1]{\IdrisData{\texttt{#1}}}
\newcommand{\bn}[1]{\IdrisBound{\texttt{#1}}}
\newcommand{\cm}[1]{\textcolor{darkgray}{\texttt{#1}}}
\newcommand{\hole}[1]{\IdrisMetavar{\texttt{?}\IdrisMetavar{\texttt{#1}}}}
\newcommand{\Elab}{\ty{Elab}}
\newcommand{\String}{\ty{String}}
\newcommand{\TT}{\ty{TT}}
\newcommand{\Raw}{\ty{Raw}}
\newcommand{\Editorable}{\ty{Editorable}}
\newcommand{\TyDecl}{\ty{TyDecl}}
\newcommand{\FunDefn}{\ty{FunDefn}}
\newcommand{\FunClause}{\ty{FunClause}}

\newcommand{\quadthree}{\qquad\quad}
\newcommand{\quadfour}{\quadthree\quad}
\newcommand{\quadfive}{\quadfour\quad}
\newcommand{\quadsix}{\quadfive\quad}
\newcommand{\quadseven}{\quadsix\quad}
\newcommand{\quadeight}{\quadseven\quad}
\newcommand{\quadten}{\quadfive\quadfive}
\newcommand{\Red}[1]{{\color{red} #1}}
\newcommand{\TODO}[1]{{\color{red}{[TODO: #1]}}}
\newcommand{\FYI}[1]{{\color{green}{[FYI: #1]}}}
\newcommand{\forceindent}{\hspace{\parindent}}
\sloppy


% \allowdisplaybreaks

\department{Mathematics and Computer Science}
\submitdate{May 2018}
\advisor{Daniel R. Licata}
\title{Edit-Time Tactics in Idris}
\author{Joomy Korkut}

\theoremstyle{plain}
\newtheorem{theorem}{Theorem}[chapter]
\newtheorem{corollary}{Corollary}[theorem]
\newtheorem{lemma}[theorem]{Lemma}

\theoremstyle{definition}
\newtheorem{defn}{Definition}[chapter]

\newcommand{\T}[1]{\texttt{#1}}
\lstset{mathescape=true,basicstyle=\ttfamily}

\newcommand\numberthis{\addtocounter{equation}{1}\tag{\theequation}}

\numberwithin{section}{chapter}
\numberwithin{figure}{chapter}

\begin{document}

\begin{abstract}
  \forceindent Metaprogramming is a technique that allows users to write
  programs that write programs. In dependently-typed languages such as Idris,
  recent work on elaborator reflection\cite{elabref} paved the way for new
  applications of metaprogramming by showing that it can be a substitute for
  tactic-based proof languages. The goal of our work is to use elaborator
  reflection to write editor interaction actions in Idris.

  Currently in Idris modes of editors such as Emacs, users can perform actions
  like type-checking holes, case-splitting, and lemma extraction.
  Implementations of all of these Idris editor actions are hard-coded in the
  compiler and they are written in Haskell. Our work will allow us to rewrite
  them in Idris as metaprograms, and to move them into an Idris library,
  instead of having them embedded into the compiler.

  Furthermore, we can write our own tactics through elaborator
  reflection, and run them from the editor, i.e. in edit-time.
  This would extend the abilities of the editor interaction mode from the
  current built-in features to anything that can be done with tactics.
  In our work, we present the design and implementation of this feature in the
  Idris compiler.

  We also implement an intuitionistic theorem prover tactic that is meant to be an
  better alternative to the built-in proof search editor action, and an
  add-clause tactic that exemplifies how we can move some of the hard-coded
  features from the compiler to a library.
\end{abstract}

\begin{dedication}
  \epigraph{``What do compilers do? They manipulate programs! Making it easy
  for users to manipulate their own programs, and also easy to interlace their
  manipulations with the compiler’s manipulations, creates a powerful new
  tool.''}{\textit{Tim Sheard and Simon Peyton Jones}\cite{th}}
\end{dedication}

\begin{acknowledgements}
\forceindent First and foremost, I would like to thank Dan Licata, my research
  advisor.  Even though the topic I picked was not very close to the work he
  does, he played along and gave me very helpful advice whenever I got stuck.
  Since this project also had a user interface aspect, his insight as a
  long-time user of proof assistants shaped my decisions.  He has had more
  impact on my research taste than anyone, and I am indebted to him for his
  kindness and patience.

I am incredibly fortunate to have met David Christiansen last summer and talked
to him about his work on metaprogramming for dependent types.
In his dissertation\cite{davidphd} he laid the foundation that makes this
project possible, and he later pitched me the idea of using elaborator
reflection for interactive editing.
I could not have finished this work without him putting up with my incessant
stream of questions. (My words, not his.)

I would like to express my gratitude to James Lipton and Edward Morehouse for
reading and evaluating my thesis. I would also like to thank Cyrus Omar for
coining the term ``edit-time tactics''. Even though he did so in a different
context\cite{hazelnut}, I cannot think of a better way to describe this
project. I would like to thank Matthew Wilson for reading my draft many
times, and helping me put my project in perspective with our conversations
about the history and future of IDEs and programming.

I want to thank Joseph Cutler and Mitchell Riley for the programming languages
discussions we had over the year; I am grateful for our little PL community.

Finally, I am thankful to Yulia, Mehmet, Recep, Kivanc, Isin Ekin and
my family for their emotional support. This has not been an easy year for me,
but I made it through thanks to all of you.
\end{acknowledgements}

\frontmatter
\maketitle
\makeabstract
\makededication
\makeack

\tableofcontents

\newcommand{\nocontentsline}[3]{}
\bgroup\let\addcontentsline=\nocontentsline
\renewcommand\numberline[1]{#1.\ }
\listoffigures
\egroup

\mainmatter

% \begin{figure}[ht]
% \caption{Some Idris code}
% \label{code:idrisExample}
% \begin{Verbatim}

% \end{Verbatim}
% \end{figure}

\section{Introduction}
\label{sec:introduction}

% \begin{figure}[ht]
% \caption{Some Idris code}
% \label{code:idrisExample}
% \begin{Verbatim}
% \IdrisKeyword{data} \IdrisType{(=)} : \IdrisImplicit{A} -> \IdrisImplicit{B} -> \IdrisType{Type} \IdrisKeyword{where}
%   \IdrisData{Refl} : \IdrisImplicit{x} \IdrisType{=} \IdrisImplicit{x}
% \end{Verbatim}
% \end{figure}


\cite{idris}

\chapter{Background}\label{chap:background}

\newcommand{\zip}{\fn{zip}}
\section{Idris}

Idris is a dependently typed functional programming language. In simple terms,
dependent types allow us to do computation in types, just like we can do
computation in terms~\cite{davidphd}. Moreover, we can use the computation
in types to shape our definitions of computations in terms.

A helpful intuition for functional programmers is to think about how the
concept that functions are values was initially a novel idea, and now in
dependently typed programming, we promote types to values as well.
Thus a function can now take a term as an argument and return a type as a
result~\cite{lambdacube,henk}.
For those familiar with Haskell type families~\cite{typefamilies}, this is
similar to that, but notice that type families are like functions from types to types,
while dependent types allow us to have functions from \emph{terms} to types.

In \autoref{code:vect}, we define a data type of vectors. It is exactly
like lists, except now the length is stored within the type. In other
words, as we add elements, the length is computed at the type level.

After that, we define a function that zips two vectors. Notice that only two
cases are enough to cover all possible paths of this function. If we were to
define a \zip\ function for lists, we would need four cases: both empty, both
non-empty, one empty and one non-empty, and one non-empty and one empty.
However, our \zip\ function takes two arguments that are of the same length
\bn{n}.\footnote{Both the \ty{Nat} named \bn{n} and the \ty{Type} named \bn{a} here are
implicit quantifiers.}  Therefore, we cannot have a case with one empty and one
non-empty, because that contradicts the fact that both vectors are of the same
length. Notice that the \zip\ function returns a vector of length \bn{n}.  In
other words, the input vectors and the resulting vector are guaranteed to be of
the same length.

\begin{figure}[H]
\caption{Example of a dependently typed Idris code: vectors}
\label{code:vect}
\begin{Verbatim}[framesep=2mm, label=\footnotesize{\normalfont{Idris}}, labelposition=topline]
\IdrisKeyword{data} \IdrisType{Vect} : \IdrisType{Nat} -> \IdrisType{Type} -> \IdrisType{Type} \IdrisKeyword{where}
  \IdrisData{Nil} : \IdrisType{Vect} \IdrisData{0} \IdrisImplicit{elem}
  \IdrisData{(::)} : \IdrisImplicit{elem} -> \IdrisType{Vect} \IdrisImplicit{len} \IdrisImplicit{elem} -> \IdrisType{Vect} (\IdrisData{S} \IdrisImplicit{len}) \IdrisImplicit{elem}

\IdrisFunction{zip} : \IdrisType{Vect} \IdrisImplicit{n} \IdrisImplicit{a} -> \IdrisType{Vect} \IdrisImplicit{n} \IdrisImplicit{b} -> \IdrisType{Vect} \IdrisImplicit{n} (\ty{Pair} \bn{a} \bn{b})
\IdrisFunction{zip} \IdrisData{[]} \IdrisData{[]} = \IdrisData{[]}
\IdrisFunction{zip} (\IdrisBound{x} \IdrisData{::} \IdrisBound{xs}) (\IdrisBound{y} \IdrisData{::} \IdrisBound{ys}) = \IdrisData{(}\IdrisBound{x}\IdrisData{,} \IdrisBound{y}\IdrisData{)} \IdrisData{::} \IdrisFunction{zip} \IdrisBound{xs} \IdrisBound{ys}
\end{Verbatim}
\end{figure}

In \autoref{code:evenodd}, we define
two similar data types that ensure that the given natural number is even or
odd, and compute the number within the type at every step. Later, we give a
function that takes a natural number and generates either a value of the type
that ensures evenness or one that ensures oddness. This corresponds to a proof that for all $n \in \mathbb{N}$, $n$ is even or $n$ is odd.

\begin{figure}[H]
\caption{Example of a dependently typed Idris code, parity of natural numbers}
\label{code:evenodd}
\begin{Verbatim}[framesep=2mm, label=\footnotesize{\normalfont{Idris}}, labelposition=topline]
\IdrisKeyword{data} \IdrisType{Even} : \IdrisType{Nat} -> \IdrisType{Type} \IdrisKeyword{where}
  \IdrisData{EvenZ} : \IdrisType{Even} \IdrisData{0}
  \IdrisData{EvenSS} : \IdrisType{Even} \IdrisImplicit{n} -> \IdrisType{Even} (\IdrisData{S} (\IdrisData{S} \IdrisImplicit{n}))

\IdrisKeyword{data} \IdrisType{Odd} : \IdrisType{Nat} -> \IdrisType{Type} \IdrisKeyword{where}
  \IdrisData{Odd1} : \IdrisType{Odd} \IdrisData{1}
  \IdrisData{OddSS} : \IdrisType{Odd} \IdrisImplicit{n} -> \IdrisType{Odd} (\IdrisData{S} (\IdrisData{S} \IdrisImplicit{n}))

\IdrisKeyword{total}
\IdrisFunction{evenOrOdd} : (\IdrisBound{n} : \IdrisType{Nat}) -> \IdrisType{Either} (\IdrisType{Even} \IdrisBound{n}) (\IdrisType{Odd} \IdrisBound{n})
\IdrisFunction{evenOrOdd} \IdrisData{0} = \IdrisData{Left} \IdrisData{EvenZ}
\IdrisFunction{evenOrOdd} \IdrisData{1} = \IdrisData{Right} \IdrisData{Odd1}
\IdrisFunction{evenOrOdd} (\IdrisData{S} (\IdrisData{S} \IdrisBound{n})) = \IdrisKeyword{case} \IdrisFunction{evenOrOdd} \IdrisBound{n} \IdrisKeyword{of}
                        \IdrisData{Left} \IdrisBound{ev} => \IdrisData{Left} (\IdrisData{EvenSS} \IdrisBound{ev})
                        \IdrisData{Right} \IdrisBound{o} => \IdrisData{Right} (\IdrisData{OddSS} \IdrisBound{o})
\end{Verbatim}
\end{figure}

Unlike other dependently typed languages like Agda and Coq, Idris is not total
by default. This is because Idris prioritizes general purpose programming
rather than theorem proving. However, users can opt in to totality
checking either for the entire module or for specific
declarations.\footnote{Both for functions and data type definitions, since the
latter are checked for strict positivity.}
We did the latter for \fn{evenOrOdd} by using the keyword \kw{total}.
Similarly, we could require the \zip\ function from the previous example to be
total if we wanted to.\footnote{Clearly Idris cannot decide whether an arbitrary
function is total, since that would solve the halting problem. Instead it
acknowledges the ones that are obviously total, and for all the other
ones, even if they are actually total, it throws a totality check error.}

Haskell's type classes and type class instances are called \emph{interfaces} and
\emph{implementations} in Idris, respectively. In Haskell there can only be one
instance for the same type class and type, but in Idris there can be
multiple implementations for the same interface and type. You can name
implementations and specify the implementation you want to use by its name
when you are writing a function. For our purposes, we will not use multiple
implementations.

\section{Elaborator reflection}\label{sec:elabref}

Idris programs are elaborated from high-level Idris syntax trees into a core
language called \ty{TT}, and then type checked~\cite{idris}.
The implementation of the Idris elaboration in the compiler is written as a
Haskell monad called \ty{Elab}.
Recent work on elaborator reflection~\cite{elabref} allowed Idris users to
access this monad from Idris itself, by implementing a primitive monad
\Elab\ in Idris itself, that can only be used for metaprogramming in compile
time.

\subsection{Reflected core language types}\label{ssec:reflectedTypes}

Since the \Elab\ monad in Idris needs to work with core language terms and
definitions, we have to use data types in Idris that represent them.
We want to have a correspondence between the internal data type that represents
the core language syntax in the compiler and the ones defined in Idris.

In this thesis, we will use the term \emph{reflection} to refer to ``the
capability of converting some piece of concrete code into an abstract syntax
tree object that can be manipulated in the same
system''~\cite{reflInAgda}.\footnote{Note that this is somewhat at odds with
the usage of the term ``reflection'' in normalization by evaluation.} In other
words, we will define Idris types that let us work with Idris syntax trees
within Idris.  We will call these types ``reflected types'' for now. We give a
diagram in \autoref{reflectionGraph} to explain the relationship between the
metalanguage Haskell, the object language Idris, the core language of Idris and
the conversions from one to another.  The diagram provides an overall look,
which is not required to understand this section, but finishing this section is
necessary to fully comprehend the diagram. Nevertheless, looking at the diagram
before reading this section might be helpful to some readers.

The most important reflected type is called \TT, which represents the core
language's typed terms, and its definition can be seen in \autoref{code:ttDef}.

\begin{figure}[H]
\caption{The reflected type \protect\TT\ in Idris.}
\label{code:ttDef}
\begin{Verbatim}[framesep=2mm, label=\footnotesize{\normalfont{Idris}}, labelposition=topline]
\IdrisKeyword{data} \IdrisType{TT} : \IdrisType{Type} \IdrisKeyword{where}
  \IdrisData{P} : \IdrisType{NameType} -> \IdrisType{TTName} -> \IdrisType{TT} -> \IdrisType{TT}
  \IdrisData{V} : \IdrisType{Int} -> \IdrisType{TT}
  \IdrisData{Bind} : \IdrisType{TTName} -> \IdrisType{Binder} \IdrisType{TT} -> \IdrisType{TT} -> \IdrisType{TT}
  \IdrisData{App} : \IdrisType{TT} -> \IdrisType{TT} -> \IdrisType{TT}
  \IdrisData{TConst} : \IdrisType{Const} -> \IdrisType{TT}
  \IdrisData{Erased} : \IdrisType{TT}
  \IdrisData{TType} : \IdrisType{TTUExp} -> \IdrisType{TT}
  \IdrisData{UType} : \IdrisType{Universe} -> \IdrisType{TT}
\end{Verbatim}
\end{figure}

As a quick summary:
\begin{itemize}
\item\dt{P} creates a variable term from a name, as defined in
  \autoref{code:ttnameDef}, and the type of the variable.
\item\dt{V} creates a de Bruijn variable.
(given integer $n$ representing the $n$th most recently introduced local variable)
\item\dt{Bind} creates any kind of binder (lambda, let etc.) with a term it binds on.
\item\dt{App} creates a function application.
\item\dt{TConst} creates a constant such as an integer, a character, a string etc.
\item\dt{Erased} creates a term that is not known. This is used for erasing the types we do not need later in the compilation.
\item\dt{TType} creates a type of types for a given universe.
\item\dt{UType} creates a uniqueness type for a given uniqueness universe.
\end{itemize}

This summary is meant to be an overview, so refer to \citet{idris} and
\citet{elabref} if this is not perfectly clear. For our purposes, we will mostly
be concerned with \dt{P}, \dt{Bind} and \dt{App}.

The other important type that is used in elaborator reflection is \ty{Raw},
which is the type of untyped core language terms, and its definition can be
seen in \autoref{code:rawDef}.

\begin{figure}[H]
\caption{The reflected type \protect\ty{Raw} in Idris.}
\label{code:rawDef}
\begin{Verbatim}[framesep=2mm, label=\footnotesize{\normalfont{Idris}}, labelposition=topline]
\IdrisKeyword{data} \IdrisType{Raw} : \IdrisType{Type} \IdrisKeyword{where}
  \IdrisData{Var} : \IdrisType{TTName} -> \IdrisType{Raw}
  \IdrisData{RBind} : \IdrisType{TTName} -> \IdrisType{Binder} \IdrisType{Raw} -> \IdrisType{Raw} -> \IdrisType{Raw}
  \IdrisData{RApp} : \IdrisType{Raw} -> \IdrisType{Raw} -> \IdrisType{Raw}
  \IdrisData{RType} : \IdrisType{Raw}
  \IdrisData{RUType} : \IdrisType{Universe} -> \IdrisType{Raw}
  \IdrisData{RConstant} : \IdrisType{Const} -> \IdrisType{Raw}
\end{Verbatim}
\end{figure}

The constructors of \ty{Raw} are almost the same ones as \TT,
except a few of them are missing and variables do not have to be annotated with
their types. This makes \Raw\ terms easier to type by hand, if necessary.
Therefore \TT\ terms will usually be treated as the output from the type
checker, and \Raw\ terms will be the input to the type checker.

The \ty{TTName} type is the type of names in the core language, its full definition can be seen in \autoref{code:ttnameDef}.

\begin{figure}[H]
\caption{The reflected type \protect\ty{TTName} in Idris.}
\label{code:ttnameDef}
\begin{Verbatim}[framesep=2mm, label=\footnotesize{\normalfont{Idris}}, labelposition=topline]
\IdrisKeyword{data} \IdrisType{TTName} : \IdrisType{Type} \IdrisKeyword{where}
  \IdrisData{UN} : \IdrisType{String} -> \IdrisType{TTName}
  \IdrisData{NS} : \IdrisType{TTName} -> \IdrisType{List} \IdrisType{String} -> \IdrisType{TTName}
  \IdrisData{MN} : \IdrisType{Int} -> \IdrisType{String} -> \IdrisType{TTName}
  \IdrisData{SN} : \IdrisType{SpecialName} -> \IdrisType{TTName}
\end{Verbatim}
\end{figure}

As a quick summary:
\begin{itemize}
\item\dt{UN} represents user-provided variable names without any namespace.
\item\dt{NS} represents variable names with a given namespace. For example, the name \texttt{Prelude.Bool.True} is represented as
  \mbox{\texttt{\dt{NS} (\dt{UN} \dt{"True"}) \dt{["Bool", "Prelude"]}}}.
\item\dt{MN} represents machine generated names with a hint string and a fresh integer for that hint.
\item\dt{SN} represents special names, which are used for metavariables, implementations etc. We will not deal with them in this thesis.
\end{itemize}

As a quick way to refer to Idris names, there is a syntactic sugar
\texttt{\IdrisKeyword{\`{}\{\{}x\IdrisKeyword{\}\}}}
that would give you the term \dt{UN} \dt{"x"}.
Similarly, there is another syntactic sugar that checks whether a given name
exists and lets you refer to an existing name without having to specify its
full namespace: \texttt{\IdrisKeyword{\`{}\{}False\IdrisKeyword{\}}} would give
you \texttt{\dt{NS} (\dt{UN} \dt{"False"}) \dt{["Bool", "Prelude"]}}.

\medskip

There are many other types used in the reflection of the core language, but we
will not give their definitions here since they are not as common as \TT,
\ty{Raw}, and \ty{TTName}. However, it is useful to at least list the most
important ones and describe what they represent.

\begin{itemize}
\item\ty{TyDecl} represents type declarations.
\item\ty{DataDefn} represents data type definitions.
\item\ty{FunDefn} represents function definitions.
\item\ty{FunClause} represents a single clause in a function definition.
\end{itemize}

\subsection{Quotations}\label{ssec:quotation}

Writing \TT\ and \ty{Raw} terms by hand can get tedious, hence there is a
quotation syntax that elaborates a given expression into its corresponding
\TT\ or \ty{Raw} term~\cite{idrisQuotation}.
The syntax \texttt{\`{}(\fn{e})}, where \fn{e} is an Idris expression, gives
us the typed or untyped core language syntax tree for \fn{e}. For example,
\texttt{\`{}(\fn{not} \dt{True})} gives us the following \TT\ term:

\begin{figure}[H]
  \caption{The \TT\ term we get when we quote \texttt{\fn{not} \dt{True}}.}
\begin{Verbatim}[framesep=2mm, label=\footnotesize{\normalfont{Idris}}, labelposition=topline]
\IdrisData{App} (\IdrisData{P} \IdrisData{Ref}
       (\IdrisData{NS} (\IdrisData{UN} \IdrisData{"not"}) \IdrisData{[}\IdrisData{"Bool"}\IdrisData{,} \IdrisData{"Prelude"}\IdrisData{]})
       (\IdrisData{Bind} (\IdrisData{UN} \IdrisData{"__pi_arg"})
             (\IdrisData{Pi} (\IdrisData{P} (\IdrisData{TCon} \IdrisData{8} \IdrisData{0}) (\IdrisData{NS} (\IdrisData{UN} \IdrisData{"Bool"}) \IdrisData{[}\IdrisData{"Bool"}\IdrisData{,} \IdrisData{"Prelude"}\IdrisData{]}) \IdrisData{Erased})
                 (\IdrisData{TType} (\IdrisData{UVar} \IdrisData{"./Prelude/Bool.idr"} \IdrisData{71})))
             (\IdrisData{P} (\IdrisData{TCon} \IdrisData{8} \IdrisData{0}) (\IdrisData{NS} (\IdrisData{UN} \IdrisData{"Bool"}) \IdrisData{[}\IdrisData{"Bool"}\IdrisData{,} \IdrisData{"Prelude"}\IdrisData{]}) \IdrisData{Erased})))
    (\IdrisData{P} (\IdrisData{DCon} \IdrisData{1} \IdrisData{0})
       (\IdrisData{NS} (\IdrisData{UN} \IdrisData{"True"}) \IdrisData{[}\IdrisData{"Bool"}\IdrisData{,} \IdrisData{"Prelude"}\IdrisData{]})
       (\IdrisData{P} (\IdrisData{TCon} \IdrisData{0} \IdrisData{0}) (\IdrisData{NS} (\IdrisData{UN} \IdrisData{"Bool"}) \IdrisData{[}\IdrisData{"Bool"}\IdrisData{,} \IdrisData{"Prelude"}\IdrisData{]}) \IdrisData{Erased}))
\end{Verbatim}
\end{figure}

The \ty{Raw} term for the same expression is a bit smaller:
\begin{figure}[H]
  \caption{The \Raw\ term we get when we quote \texttt{\fn{not} \dt{True}}.}
\begin{Verbatim}[framesep=2mm, label=\footnotesize{\normalfont{Idris}}, labelposition=topline]
(\IdrisData{RApp} (\IdrisData{Var} (\IdrisData{NS} (\IdrisData{UN} \IdrisData{"not"}) \IdrisData{[}\IdrisData{"Bool"}\IdrisData{,} \IdrisData{"Prelude"}\IdrisData{]}))
      (\IdrisData{Var} (\IdrisData{NS} (\IdrisData{UN} \IdrisData{"True"}) \IdrisData{[}\IdrisData{"Bool"}\IdrisData{,} \IdrisData{"Prelude"}\IdrisData{]})))
\end{Verbatim}
\end{figure}

Obviously, we would not want to write terms like these manually every time we want to return
the syntax tree for a simple function application.
At times like this, quotation saves us, for both expressions and patterns.

We can also give the type of the expression we want to elaborate, which becomes
necessary when Idris cannot infer the type. For \dt{True}, it is trivial to
infer that type is \ty{Bool}, but for \dt{5}, the type can be \ty{Int},
\ty{Integer}, \ty{Nat}, or anything that satisfies the \ty{Num} interface.
Therefore, we have to specify the type when we are quoting. The
syntax for that is \mbox{\texttt{\`{}(\fn{e} : \ty{t})}},
e.g. \mbox{\texttt{\`{}(\dt{5} : \ty{Nat})}}.

We also can do antiquotation. If we have some variable expression \fn{x} that has the
type \TT\ or \ty{Raw}, then we can construct a syntax tree using it within the
quotation, with the syntax \mbox{\texttt{\`{}(\fn{not} \textasciitilde\fn{x})}}.
Note that antiquotation works for expressions, not just variables. The type of
expression or variable we have in the antiquotation has to match the type of
the quotation. In other words, only \TT\ expressions can be used in an
antiquotation in a \TT\ quotation, and \emph{mutatis mutandis} for \ty{Raw}.

For further information on Idris' quotations, see \citet{idrisQuotation}.

\subsection{\protect\Elab\ monad}

The elaborator reflection~\cite{elabref} feature that has been added to the
Idris compiler recently provides a tool for metaprogramming with a monad called
\Elab.  This monad is implemented as a primitive and it can only be run during
compile time, or in the interactive proof shell.

Elaborator reflection adds a new declaration
\texttt{\IdrisKeyword{\%runElab}}\ \fn{e} to Idris,
where \fn{e} has the type \ty{Elab}\ \ty{()}.
This declaration runs the \Elab\ action and adds new type declarations,
function and data type definitions generated by the \Elab\ action generated by
\fn{e} to the context.

Elaborator reflection also adds a new \emph{expression}
\texttt{\IdrisKeyword{\%runElab}}\ \fn{e} to Idris,
where \fn{e} has the type \ty{Elab}\ \ty{()}.
The type \ty{t} of the entire expression is started as the goal of the
\Elab\ action, and the tactics in \fn{e} must solve the goal
that has the type \ty{t}.
Like the declaration above, this expression also adds the newly generated
declarations and definitions to the context.

The \Elab\ monad holds a proof state inside, which has a goal type, a proof
term that is incrementally built up, a hole queue, a collection of open
unification problems, and a supply of fresh names~\cite{elabref}.
This state is really held in the Haskell \Elab\ monad, though it can be
observed from Idris.

Tactics can change the proof state. Here are some examples that do that:
\begin{itemize}
\item\texttt{\IdrisFunction{claim} : \IdrisType{TTName} -> \IdrisType{Raw} -> \IdrisType{Elab} \IdrisType{()}}\\
  Creates a new hole with a given name and a type.
\item\texttt{\IdrisFunction{fill} : \IdrisType{Raw} -> \IdrisType{Elab} \IdrisType{()}}\\
Create a guess to fill the current hole with a term. Fail if the types do not unify~\cite{mcbridephd}.
\item\texttt{\IdrisFunction{solve} : \IdrisType{Elab} \IdrisType{()}}\\
Try to finalize the guess in the hole. Fail if there is no guess~\cite{mcbridephd}.
\end{itemize}

\noindent There are a lot more tactics, which we will not list here. A more thorough list
can be found in \citet{elabref} and Idris documentation.

We also have access to \Elab\ actions that do not change the proof state, but give us access to the context or other compiler primitives:
\begin{itemize}
\item\texttt{\IdrisFunction{check} : \IdrisType{List} \IdrisType{(}\IdrisType{TTName}\IdrisType{,} \IdrisType{Binder} \IdrisType{TT}\IdrisType{)} -> \IdrisType{Raw} -> \IdrisType{Elab} \IdrisType{(}\IdrisType{TT}\IdrisType{,} \IdrisType{TT}\IdrisType{)}}\\
Type-checks a term under a given environment and gives the typed core term version of the \ty{Raw} term and the type of it as a typed core term.
\item\texttt{\IdrisFunction{normalise} : \IdrisType{List} \IdrisType{(}\IdrisType{TTName}\IdrisType{,} \IdrisType{Binder} \IdrisType{TT}\IdrisType{)} -> \IdrisType{TT} -> \IdrisType{Elab} \IdrisType{TT}}\\
Normalizes\footnote{\fn{normalise} is spelled the British way, since most Idris development happens in the UK.} a typed term under a given environment.
\item\texttt{\IdrisFunction{lookupTy} : \IdrisType{TTName} -> \IdrisType{Elab} (\IdrisType{List} \IdrisType{(}\IdrisType{TTName}\IdrisType{,} \IdrisType{NameType}\IdrisType{,} \IdrisType{TT}\IdrisType{)})}\\
Looks up the type of the given name and returns the ones it finds in a list, in case the name is ambiguous.
\end{itemize}

Observe that in some of these functions, for inputs we use \ty{Raw}, the
untyped core language terms, and results are in \ty{TT}, the typed core
language terms.  This is because untyped core language terms are easier to
write for the tactic users, and type-checking them in the elaborator is easy.

Now let's define a function using elaborator reflection.
Take the polymorphic identity function, for example.

\begin{figure}[H]
  \caption{The identity function using elaborator reflection in Idris.}
\begin{Verbatim}[framesep=2mm, label=\footnotesize{\normalfont{Idris}}, labelposition=topline]
\fn{id} : (\bn{a} : \ty{Type}) -> \bn{a} -> \bn{a}
\fn{id} = \kw{%runElab} (\kw{do} \fn{intro} \kw{`\{\{}ty\kw{\}\}}
                  \fn{intro} \kw{`\{\{}a\kw{\}\}}
                  \fn{fill} (\dt{Var} \kw{`\{\{}a\kw{\}\}})
                  \fn{solve})
\end{Verbatim}
\end{figure}

For anyone familiar with Coq, this will look very similar to a normal Coq
proof. First we take the type as an argument, and then a value of that type,
and we return the same value. Elaborator reflection proofs look a bit more
unpolished compared to Coq proofs\footnote{This is because Coq has the Gallina
language for proof terms and Ltac~\cite{ltac} for tactics, therefore, while
Idris does not have such a distinction; it only has one language: Idris itself.
Therefore the syntaxes for core language terms, quotation and special names
look more cluttered.  Another reason that elaborator reflection looks more
unpolished is that Coq tactics are designed to just inhabit a type, while
\Elab\ is designed for \emph{programs} where we want to control the precise
computational behavior.}, but it is essentially very similar to Coq
tactics, hence the name ``tactics'' we use to refer to monadic \Elab\ actions.


Let's prove a lemma in Idris. This time we want to prove that
\mbox{$(\forall n \in \mathbb{N})\ n = n + 0$}, for the standard definition of
addition. Since that requires more complex tactics like induction, we will
import the Pruviloj\footnote{Pruviloj is the Idris library included in
the Idris distribution. It contains complex tactics written with elaborator
reflection.} library~\cite{davidphd}.

\begin{figure}[H]
  \caption{A proof that $(\forall n \in \mathbb{N})\ n = n + 0$ using elaborator reflection in Idris.}
\begin{Verbatim}[framesep=2mm, label=\footnotesize{\normalfont{Idris}}, labelposition=topline]
\fn{nPlusZero} : (\bn{n} : \ty{Nat}) -> \bn{n} = \fn{plus} \bn{n} \dt{0}
\fn{nPlusZero} = \kw{%runElab} (\kw{do} \fn{intro} \kw{`\{\{}n\kw{\}\}}
                         \fn{induction} (\dt{Var} \kw{`\{\{}n\kw{\}\}})
                         \fn{compute}
                         \fn{reflexivity}
                         \fn{compute}
                         \fn{attack}
                         \fn{intro} \kw{`\{\{}n1\kw{\}\}}
                         \fn{intro} \kw{`\{\{}indHyp\kw{\}\}}
                         \fn{rewriteWith} (\dt{Var} \kw{`\{\{}indHyp\kw{\}\}})
                         \fn{reflexivity}
                         \fn{solve})
\end{Verbatim}
\end{figure}

The proof proceeds as follows: we first take in the argument \bn{n}, and then do
an induction on \bn{n}. Because of the way induction works in Pruviloj, we have
to simplify the goal using \fn{compute}\footnote{Its Coq equivalent would be
\texttt{simpl}.}.
For the base case, the goal is just proving \dt{0} \ty{=} \dt{0}.
For the inductive step, we have to restructure the goal with \fn{attack} and
then reintroduce the input and then introduce the induction hypothesis. Then we
rewrite the goal with the induction hypothesis and then the goal becomes
trivial. Understanding this proof completely is not crucial for this thesis,
but if you want to fully comprehend \fn{attack} and \fn{solve}, you can refer to
\citet{elabref}.

Finally let's do an example for a declaration with elaborator reflection:

\begin{figure}[H]
  \caption{A type declaration and new definition for \fn{n}, using elaborator reflection in Idris.}
\begin{Verbatim}[framesep=2mm, label=\footnotesize{\normalfont{Idris}}, labelposition=topline]
\kw{%runElab} (\kw{do} \fn{declareType} (\dt{Declare} \kw{\`{}\{\{}n\kw{\}\}} \dt{[]} \kw{`{}(}\ty{Nat}\kw{)})
             \fn{defineFunction} (\dt{DefineFun} \kw{\`{}\{\{}n\kw{\}\}}
                              \dt{[}\dt{MkFunClause} (\dt{Var} \kw{\`{}\{\{}n\kw{\}\}}) \kw{\`{}(}\dt{Z}\kw{)}\dt{]}))
\end{Verbatim}
\end{figure}

\noindent The example above first declares that \fn{n} will have the type \ty{Nat}.
Then it defines it as \mbox{\texttt{\IdrisFunction{n} = \IdrisData{Z}}}.

Now that we have seen different use cases for elaborator reflection, we can move on to the
design of the edit-time tactics feature.


% Previous versions of Idris had a tactic based prover\footnote{As of Idris
% 1.1.1 it is still available, with the warning that it will be deprecated in the
% future versions.}, which embedded the proof tactics in a Haskell monad in the
% implementation~\cite{elabref}

% what sorts of things stay, what sorts of things go away with elaboration
% mcbride dependent pattern matching? (Epigram paper?)
% standard eliminators?
% (a few constructions on constructors: injectivity and disjointness)
% icfp 2016 - Jesper Cockx, doing that without axiom K
% how to get rid of dep pattern matching and turning it into induction principle

\section{Design} \label{sec:design}

We described in \autoref{sec:introduction} what a simple editor interaction
action like case splitting does.  What we want to do is to add another command
that would be recognized by our compiler, and specify how such a command would
run, and what arguments it should take to communicate with the editor
effectively.

The current Idris\footnote{This thesis is using Idris 1.1.1.} implementation of
the editor interaction mode is a part of the Idris compiler, and is written in
Haskell. When the editor is running,
it runs the \path{idris} executable with the \path{--ide-mode} flag, which
allows socket communication with the program through a machine-readable
syntax.\footnote{The Idris mode of Vim works differently, since Vim did not
support asynchronous jobs until 8.0. This is expected to change in the near
future.}
To be more precise, this program receives S-expressions\cite{mccarthy} as input
over the socket and sends back S-expressions as output. Let's see an example of
this. Suppose we are writing a predecessor function on natural numbers and we
want to case split on the argument.

\begin{Verbatim}
\IdrisFunction{pred} : \IdrisType{Nat} -> \IdrisType{Nat}
\IdrisFunction{pred} \IdrisBound{x} = \IdrisMetavar{?\IdrisMetavar{p}}
\end{Verbatim}

When we put the cursor on \path{x} and run case splitting, the following
communication happens in the background.

\begin{Verbatim}
-> ((:case-split 5 "x") 13)
<- (:return (:ok "pred Z = ?p_1{\textbackslash}npred (S k) = ?p_2{\textbackslash}n") 13)
\end{Verbatim}

Case split command requires two arguments, the line number and the variable
name to case split on. The line number is included in most editor actions that
change the code. The variable name is a string that hold the necessary data.
The number \path{13} that comes afterwards is how many rounds of communications
have been done so far. Notice that the response also carries the same number.
The response contains \path{:ok}, which means the case splitting succeeded,
and then a string that contains two lines split by the new line character.
The editor receives twose two lines and replaces that with the line we ran the
case split command on.

The most important thing to notice here is that the way the editor and the
compiler communicate is very ``stringly typed''\footnote{A term coined by Mark
Simpson, mentioned in a deleted Stack Overflow thread.}.  Since all of these
editor actions are primitives, this is initially fine. However, we now want to
create a single command that handles different kinds of actions. These actions
possibly take multiple arguments that hold different kinds of values, such as
names, type declarations, function definitions, or expressions.
The data itself can still be sent as a string by the editor, but having type
annotations would give a hint to the compiler about how to parse the given
string into parts of a syntax tree. In other words, the command should take a
list of strings, each tagged with a type annotation.

Then the question is, how will the editor know what to tag each string with?
It will not. For each non-trivial editor action, the user will have to extend
the Idris mode of the editor with a function that tags each string with a type
annotation. The resulting communication should look like this for case splitting.

\begin{Verbatim}
-> ((:elab-action :case-split 5 '((:name . "x"))) 13)
<- (:return (:ok '((:fun-defn . "pred Z = ?p_1")
                   (:fun-defn . "pred (S k) = ?p_2"))) 13)
\end{Verbatim}

Notice that the call to the editor states that we are calling the \path{Elab}
action that does case splitting, and it takes a singleton argument list, which
is a pair of a tag and a string. The tag states that the string should be
handled as a variable name by the compiler.
The result from the editor returns that the action succeeded and then a list of
results.  In the result list we have two entries, each of which is a pair of a
tag and a string.  The tags in here state that the string represents a function
definition.


\chapter{Implementation}\label{sec:implementation}

\section{Primitive \ty{Editorable} implementations}\label{sec:primitiveEditorableImpl}




\chapter{Applications} \label{chap:applications}

\section{A tactic to replace the built-in ``add clause'' action}\label{sec:addClause}

We have seen in \autoref{chap:introduction} how the ``Add initial match clause
to type declaration'' editor action works. When the cursor is on the type
signature of a function that does not have any clauses, we can run this editor
action and get an initial clause for the function.

In this section we implement this editor action for top-level type declarations
without implicit arguments or interface constraints, using edit-time tactics.
The Idris code we need to write is in \autoref{code:addClause}.

\begin{figure}[ht]
\caption{Implementation of the edit-time tactic for ``add clause''.}
\label{code:addClause}
\begin{Verbatim}[framesep=2mm, label=\footnotesize{\normalfont{Idris}}, labelposition=topline]
\fn{collectTypes} : \ty{TT} -> (\bn{xs} : \ty{List} \ty{TT} \dt{**} \ty{NonEmpty} \bn{xs})
\fn{collectTypes} (\dt{Bind} _ (\dt{Pi} \bn{ty} _) \bn{t}) =
  (\bn{ty} \fn{::} \bn{fst} (\fn{collectTypes} \bn{t}) ** \dt{IsNonEmpty})
\fn{collectTypes} \bn{t} = (\dt{[}\bn{t}\dt{]} ** \dt{IsNonEmpty})

\kw{%editor}
\fn{addClause} : \ty{TTName} -> \ty{Elab} (\ty{FunClause} \ty{TT})
\fn{addClause} \bn{n} =
  \kw{do} (_, _, \bn{ty}) <- \fn{lookupTyExact} \bn{n}
     \kw{let} (\bn{collected} \dt{**} \bn{nonEmpty}) = \fn{collectTypes} \bn{ty}
     \kw{let} (\bn{argTys}, \bn{retTy}) = (\fn{init} \bn{collected}, \fn{last} \bn{collected})
     \bn{argNames} <- \fn{traverse} (\fn{const} \fn{fresh}) \bn{argTys}
     \kw{let} \bn{lhsUntyped} = \fn{foldl} \dt{RApp} (\dt{Var} \bn{n}) (\fn{map} \dt{Var} \bn{argNames})
     \bn{env} <- \fn{getEnv}
     (\bn{lhsTyped}, _) <- \fn{check} \bn{env} \bn{lhsUntyped}
     \bn{holeName} <- \fn{fresh}
     \kw{let} \bn{rhs} = \dt{Bind} \bn{holeName} (\dt{GHole} \bn{retTy}) (\dt{V} \dt{0})
     \fn{pure} (\bn{MkFunClause} \bn{lhsTyped} \bn{rhs})
\end{Verbatim}
\end{figure}

The \fn{collectTypes} function we write in Idris is very similar to the one we
defined in \autoref{code:collectTypes} for the compiler implementation.
This one, however, also returns a proof that the resulting list is not empty.
The function takes a list and you the same list with the last element removed,
namely \fn{init}, requires a proof that the list is non-empty.

The \fn{addClause} tactic only takes one input, which is the name of the function
we will add an initial clause for.
Using this name, we look up the type of that function, and get its components
using \fn{collectTypes}. We know that there is at least one member in that
list, so we name the last element \bn{retTy}, for the return type, and we name
the rest \bn{argTys}, since they represent types of the arguments.
For each of member of \bn{argTys}, we generate a new name using
\fn{fresh}.\footnote{\fn{fresh} is defined in Hezarfen, not in a standard
library. The standard way of creating fresh names is \fn{gensym}, but we wrote
wrapper function \fn{fresh} that does not generate \dt{MN}, and gives variables
readable names, usually one-letter.} We later use these names and make them
into variable terms, using \dt{Var}, and create a function application using
all of these names. This application is supposed to represent the left-hand
side in our final definition. The right-hand side is a hole term that has the
type \bn{retTy}.  This concludes our type declaration term.

The Emacs Lisp code we write for this exactly the same as the existing
add-clause editor action, so we only have to change the part that Emacs sends a
message to the compiler. Our message now should refer to \fn{addClause} instead
of the built-in add-clause editor action.

Let's see this \Elab\ action at work in \autoref{code:exampleAddClause1}.

\begin{figure}[ht]
\caption{Example function to run \fn{addClause} on.}
\label{code:exampleAddClause1}
\begin{Verbatim}[framesep=2mm, label=\footnotesize{\normalfont{Idris}}, labelposition=topline]
\fn{example} : (\bn{name} : \ty{String}) -> (\bn{age} : \ty{Nat}) -> \ty{IO} \ty{()}
\end{Verbatim}
\end{figure}

If we put the cursor on \fn{example} and execute the Emacs Lisp code somehow, which is often done via a shortcut, we will get the result in \autoref{code:exampleAddClause2}.

\begin{figure}[ht]
\caption{Result of running \fn{addClause} on the \fn{example} function.}
\label{code:exampleAddClause2}
\begin{Verbatim}[framesep=2mm, label=\footnotesize{\normalfont{Idris}}, labelposition=topline]
\fn{example} : (\bn{name} : \ty{String}) -> (\bn{age} : \ty{Nat}) -> \ty{IO} \ty{()}
\fn{example} \bn{a} \bn{b} = \hole{c}
\end{Verbatim}
\end{figure}

This concludes the rudimentary replacement the the existing add-clause section.
For simplicity purposes, we have not covered the type signatures with implicit
arguments or interface constraints. One can write a tactic to handle
those cases as well.

\section{Theorem prover for intuitionistic propositional logic}\label{sec:hezarfen}

In this section, we describe the tactic Hezarfen\footnote{ The name is
  pronounced {[\textipa{hezaRf\ae n}]}, like ``has are fan'', and it means
    polymath in Turkish.  Tactic source code is available at
    \url{http://github.com/joom/hezarfen}.}, which can decide intuitionistic
propositional logic theorems.\footnote{Similar to Coq's \texttt{tauto} tactic.}
This tactic will be based on Dyckhoff's LJT~\cite{ljt} and its Haskell
implementation Djinn~\cite{djinn}, which generates Haskell expressions
that have a given type.

Djinn is a standalone program that takes commands
interactively, and when it generates an expression it prints it on the screen.
Instead, we want to design Hezarfen as a library that provides an \Elab\ action
that can be used as a tactic in proofs, and also as a custom editor action that
helps us when the built-in proof search mechanism does not suffice.

As a part of this library, we want to define the \Elab\ action we used in
\autoref{sec:communication}, with the type
\texttt{\fn{prover} : \ty{TTName} -> \ty{Elab} \ty{TT}}.
This tactic takes the name of the hole it is supposed to fill, and gives back a
\TT\ term in the \Elab\ monad.

Since Hezarfen's proof terms are sent back to the editor and put back into the
source code, we should aim to make our proof terms as simple as possible.
Hence, we should implement both proof term generation and simplification.

\subsection{Proof term generation}

In Hezarfen, define two types \ty{Context} and \ty{Sequent} to help us
represent the proof rules as manipulations of the goal type.

\begin{figure}[ht]
\caption{Definitions of \ty{Context} and \ty{Sequent} for Hezarfen.}
\label{code:hezarfenTypes}
\begin{Verbatim}[framesep=2mm, label=\footnotesize{\normalfont{Idris}}, labelposition=topline]
\kw{data} \ty{Context} = \dt{Ctx} (\ty{List} (\ty{TTName}, \ty{Raw})) (\ty{List} (\ty{TTName}, \ty{Raw}))
\kw{data} \ty{Sequent} = \dt{Seq} \ty{Context} \ty{Raw}
\end{Verbatim}
\end{figure}

A context is two lists of pairs that consist of names mapped to \Raw\ terms
that represent the type of the term that the name refers to.
The reason we want to have two lists is that we want to distinguish between the
\emph{consumed} and \emph{unconsumed} bindings. Once we use up an entry in the second list,
we delete it from the second list. But we may want to add new bindings, which
we do on the first list. Suppose we have $(C \vee D) \supset B$ in our context.
When we are checking the premises we want to remove it from the second list
and add $C \supset B$ and $D \supset B$ to the first list. In Dyckhoff's
presentation, this rule looks like:

\begin{prooftree}
  \AxiomC{$C \supset B,\ D \supset B,\ \Gamma \Longrightarrow G$}
  \UnaryInfC{$(C \vee D) \supset B,\ \Gamma \Longrightarrow G$}
\end{prooftree}
\vspace{\baselineskip}

But in Hezarfen's source code, this rule is written as in \autoref{code:hezarfenTypes}.

\begin{figure}[ht]
\caption{The ``\ty{Either} implies'' case in Hezarfen}
\label{code:hezarfenTypes}
\begin{Verbatim}[framesep=2mm, label=\footnotesize{\normalfont{Idris}}, labelposition=topline]
\fn{breakdown'} : \ty{Sequent} -> \ty{Elab} \ty{Tm}
\fn{breakdown'} \bn{goal} = \kw{case} \bn{goal} \kw{of}
  \cm{-- numerous previous cases}
  \dt{Seq} (\dt{Ctx} \bn{g} ((\bn{n}, \kw{\`{}(}(\dt{Either} ~\bn{d} ~\bn{e}) -> ~\bn{b}\kw{)}) :: \bn{o})) \bn{c} =>
    \kw{let} (\bn{n1}, \bn{n2}, \bn{newgoal}) = !(\fn{appDisjImplL} (\dt{Ctx} \bn{g} \bn{o}) (\bn{d}, \bn{e}, \bn{b}, \bn{c})) \kw{in}
    \kw{let} (\bn{l1}, \bn{l2}) = (!\fn{fresh}, !\fn{fresh}) \kw{in}
    \fn{pure} \fn{$} \dt{RBind} \bn{n1} (\dt{Let} \kw{\`{}(}~\bn{d} -> ~\bn{b}\kw{)}
                      (\dt{RBind} \bn{l1} (\dt{Lam} \bn{d}) (\dt{RApp} (\dt{Var} \bn{n})
                        \kw{\`{}(}\dt{Left} {\bn{a}=~\bn{d}} {\bn{b}=~\bn{e}} ~(\dt{Var} \bn{l1})\kw{)})))
          \fn{$} \dt{RBind} \bn{n2} (\dt{Let} \kw{\`{}(}~\bn{e} -> ~\bn{b}\kw{)}
                      (\dt{RBind} \bn{l2} (\dt{Lam} \bn{e}) (\dt{RApp} (\dt{Var} \bn{n})
                        \kw{\`{}(}\dt{Right} {\bn{a}=~\bn{d}} {\bn{b}=~\bn{e}} ~(\dt{Var} \bn{l2})\kw{)})))
            !(\fn{breakdown} \dt{False} \bn{newgoal})
\end{Verbatim}
\end{figure}

Understanding this code fully is not necessary; our goal is to give the greater
picture. This piece of code checks our second list in the context to see if
there is a name with the type \texttt{(\ty{Either} \bn{d} \bn{e}) -> \bn{b}},
for some \bn{d}, \bn{e} and \bn{b}. If there is, using this name \bn{n}, we can
create two functions, one with the type \texttt{\bn{d} -> \bn{b}} and the other
with the type \texttt{\bn{e} -> \bn{b}}. We generate two fresh names so we can
name these functions, and then we create a proof term, in which we generate
lambda bindings for these two functions we define in terms of \bn{n}. The rest
of the proof proceeds recursively.

The remaining rules of proof term generation proceed similarly. For more detail
on what the generated proof terms are, see Dyckhoff's paper or Hezarfen's
source code.

% explain the nested impl rule in depth ???
% and possibly one or two more, to show what they look like in general

\subsection{Simplification}

In \texttt{Hezarfen.Simplify}, we define a function \fn{reduce} that simplifies
a given \Raw\ term into another \Raw\ term in the \Elab\ monad.\footnote{We
depend on the \Elab\ monad for fresh name generation.}
A rudimentary implementation of this function that does not include all the
simplification steps is given in \autoref{code:reduce}.

\begin{figure}[ht]
\caption{Rudimentary implementation of \fn{reduce} in Hezarfen.}
\begin{Verbatim}[framesep=2mm, label=\footnotesize{\normalfont{Idris}}, labelposition=topline]
\fn{reduce} : \ty{Raw} -> \ty{Elab} \ty{Raw}
\fn{reduce} \fn{t} = \kw{case} \bn{t} \kw{of}
  \cm{-- Eta reduction:  (\textbackslash{}x => f x) becomes f}
  \dt{RBind} \bn{n} (\dt{Lam} \bn{b}) (\dt{RApp} \bn{t'} (\dt{Var} \bn{n'})) =>
    \kw{if} \bn{n} \fn{==} \bn{n'}
      \kw{then} \fn{reduce} \bn{t'}
      \kw{else} \fn{pure} \fn{$} \dt{RBind} \bn{n} (\dt{Lam} \bn{b}) !(\fn{reduce} (\dt{RApp} !(\fn{reduce} \bn{t'}) (\dt{Var} \bn{n'})))

  \cm{-- (id x) becomes x}
  \dt{RApp} (\dt{RApp} (\dt{Var} \kw{\`{}\{}\fn{id}\kw{\}}) \bn{c}) \bn{x} => \fn{reduce} \bn{x}

  \dt{RBind} \bn{n} \bn{b} \bn{t'} => \fn{pure} \fn{$} \dt{RBind} \bn{n} \bn{b} !(\fn{reduce} \bn{t'})
  \dt{RApp} \bn{t1} \bn{t2} => \fn{pure} \fn{$} \dt{RApp} !(\fn{reduce} \bn{t1}) !(\fn{reduce} \bn{t2})
  _ => \fn{pure} \bn{t}
\end{Verbatim}
\end{figure}

In the full implementation, we do more complex simplifications, such as
simplifying \texttt{(\textbackslash{}\bn{x} => \fn{g} (\fn{f} \bn{x}))} into
\texttt{(\fn{g} \fn{.} \fn{f})}, removing unused \kw{let} bindings,
substituting a \kw{let} binding in the body if the binding is only used once,
etc.

To fully simplify a \Raw\ term, we repeatedly apply it to \fn{reduce} until
fixpoint, which should be improved in future work.

There is also recent work on writing code generating theorem provers that
are more modular and efficient since they depend on an intermediate proof
representation that is later reconstructed.\cite{theoremProverCodeGeneration}
Hezarfen directly deals with the untyped core language syntax tree, both for
the input terms that represent types, and for the proof term it returns, and
this causes some overhead in our tactic.

\chapter{Related work} \label{chap:relatedwork}

\section{In Haskell}

Template Haskell~\cite{th} is the main metaprogramming mechanism in Haskell.
It is similar to elaborator reflection in the sense that metaprograms are
defined in a monad called \ty{Q}, which allows metaprograms to create fresh
names and look up definitions.
Template Haskell metaprograms generate expressions and definitions, which are
among the capabilities of the \Elab\ monad in Idris.
However, there are significant differences;
quotations in Template Haskell return values in the \ty{Q} monad, and Template
Haskell does not try to reflect the elaboration infrastructure of
Haskell.\footnote{However, Haskell metaprogramming using the GHC core language
has been discussed in the GHC developers mail list, with credit to Idris:
\url{http://mail.haskell.org/pipermail/ghc-devs/2015-November/010402.html}}
Neither does it hold an internal proof state that can be changed by monadic
actions, nor it does try to provide an alternative way to implement tactics in
Haskell.\footnote{That being said, Siva Somayyajula has a rudimentary
implementation a tactic monad in Haskell based on the \ty{Q} monad:
\url{http://github.com/ssomayyajula/elab}}

Brian McKenna worked on expanding the definitions generated by Template Haskell
to source code, which then is pretty printed and put back into the source code
in Emacs using YASnippet.\footnote{His tweet with screenshots can be found at
\url{http://twitter.com/puffnfresh/status/935274097642057728} and the project
that enables this feature can be found at
\url{http://hackage.haskell.org/package/th-pprint}.}

On the IDE feature side of things, Alan Zimmerman and Matthew Pickering
developed \texttt{ghc-exactprint}\footnote{It can be found here:
\url{http://hackage.haskell.org/package/ghc-exactprint}}, which is a library
that helps IDE and tooling development by providing a way to automatically
refactor Haskell programs without changing a part of the program
unintentionally. As they put it, their library respects ``the identity
refactoring'', which is non-trivial if your system allows many different kinds
of transformations~\cite{ghc-exactprint-blog}.
There is also the Haskell IDE Engine project that aims integrate many Haskell
tools based on the GHC API to the editor workflow, by providing a backend for
editor modes.\footnote{The project can be found here:
\url{http://github.com/haskell/haskell-ide-engine}, and Alan Zimmerman's talk
at the Haskell Implementors' Workshop 2017 can be found here:
\url{http://youtu.be/-pjQcG94CxM}}

\section{In Agda}

There is a line of work on bringing more automated theorem proving,
proof automation and tactics, or metaprogramming to Agda.
Lindblad and Benke (2006) introduced a term search algorithm called
Agsy, a proof search mechanism that aims to save users' time by automating
parts of the proof that are straightforward but tedious to write~\cite{agsy}.
Agda has a derivative of this mechanism implemented as a part of its compiler.
Kokke and Swierstra (2015) used the Agda's prior reflection system to define a
new proof search mechanism in Agda itself~\cite{autoinagda}.
The Hezarfen tactic we discussed in \autoref{sec:hezarfen} is not as advanced
as their \fn{auto} function, yet in their paper, they discussed a feature
similar to edit-time tactics as future work:

\begin{quote}
  ``In the future, it may be interesting to explore how to integrate proof
  automation using the reflection mechanism better with Agda's IDE. For
  instance, we could create an IDE feature which replaces a call to
  \fn{auto} with the proof terms that it generates. As a result, reloading
  the file would no longer need to recompute the proof terms.''~\cite{autoinagda}
\end{quote}

In this thesis, we generalized their suggestion to all tactics, and specified
how the editor/IDE and the compiler should communicate with each other
in order to successfully call a tactic with inputs of the correct types.

There is also work on ``proof by reflection'' in Agda, which is different from our
usage of the word ``reflection'' so far.

\begin{quote}\label{quote:reflection}
``Reflection is an overloaded word in this context, since in programming language
  technology reflection is the capability of converting some piece of concrete
  code into an abstract syntax tree object that can be manipulated in the same
  system. Reflection in the proof technical sense is the method of mechanically
  constructing a proof of a theorem by inspecting its shape.''~\cite{reflInAgda}
\end{quote}

We have been concerned with the first meaning of ``reflection'' in this thesis,
however the work on the second meaning of this word is still relevant to proof
automation, and their ideas can be reused in our edit-time tactics. Work by van
der Walt and Swierstra showed compelling examples of proof by reflection in
Agda, such as a proof mechanism for boolean tautologies~\cite{pfByReflAgda}.

\section{In Coq}

Coq has a metaprogramming mechanism called
\texttt{template-coq}\footnote{It can be found here:
\url{https://github.com/Template-Coq/template-coq}} that is based on
Malecha's term reification~\cite{malecha-phd}. Recently a typed
version of this system is also introduced~\cite{typed-template-coq}.
However, we are not aware of any work on using template metaprograms in Coq to
write new features for the editor.

Aside from this, there is a large body of work on proof automation, proof
engineering and tactic languages in Coq.  Coq's original tactic language is
Ltac~\cite{ltac}, which is separate from its Coq's term language Gallina.
However, alternatives to Ltac have been developed, such as Mtac~\cite{mtac} and
MetaCoq~\cite{metacoq}.  Especially Mtac, which is a tactic language
for Coq that facilitates custom proof search by providing a monadic interface,
has inspired further research in the area, including Idris' elaborator
reflection~\cite{elabref}.

Chlipala's \emph{Certified Programming with Dependent Types}~\cite{cpdt} has
emerged as the canonical introductory textbook for proof engineering; it
explains the basics of tactic programming and even delves into proof search and
proof by reflection. Note that we use the word reflection in the proof
technical sense, \hyperref[quote:reflection]{as mentioned in the quote above}.

\section{In Lean}

Lean~\cite{lean}, which has a tactic metaprogramming system~\cite{leanmeta}
similar to Idris' elaborator reflection, also allows running tactics in
edit-time, and it does not require writing any code for the editor mode
frontend.\footnote{No Emacs Lisp if you are using Emacs.} The type of these
editor actions can be seen in \autoref{code:holecommand}.

\begin{figure}[H]
\caption{Definition of \ty{hole\_command} in Lean.}
\label{code:holecommand}
\begin{Verbatim}
\kw{meta} \kw{structure} \ty{hole_command} :=
  (\bn{name}   : \ty{string})
  (\bn{descr}  : \ty{string})
  (\bn{action} : \ty{list} \ty{pexpr} \ensuremath{\to} \ty{tactic} (\ty{list} (\ty{string} \ty{\ensuremath{\times}} \ty{string})))
\end{Verbatim}
\end{figure}

They provide the following documentation for \ty{hole\_command}:\footnote{From
the \href{https://github.com/leanprover/lean/blob/17fe3decaf8ae236f0d0ff51ac8fd7f6940acdee/library/init/meta/hole\_command.lean}{source
code}
of Lean 3.4.1.}

\begin{quote}
  ``The front-end (e.g., Emacs, VS Code) can invoke commands for holes
  \mt{\{! ... !\}} in a declaration. A command is a tactic that takes zero or
  more pre-terms in the hole, and returns a list of pair \mt{(\bn{s},
  \bn{descr})} where \bn{s} is a substitution and 'descr' is a short explanation
  for the substitution.  Each string \bn{s} represents a different way to fill
  the hole.  The frontend is responsible for replacing the hole with the
  string/alternative selected by the user.  This infrastructure can be used to
  implement auto-fill and/or refine commands. An action may return an empty
  list. This is useful for actions that just return information such as: the
  type of an expression, its normal form, etc.''
\end{quote}

In comparison to the edit-time tactics mechanism presented in our work, Lean's
system is very restrictive. It only allows editor action that run on holes, but
our system allows any kind of editor action as long as the user writes the
necessary glue code in the editor mode language. We already showed in
\autoref{code:elispToy}, what the glue code to fill a hole would look like in
Emacs Lisp. Another downside of Lean's system is that editor actions can only
have a single type, as opposed to our system, which allows any kind of
\Elab\ action as long as the components of the a type all have an
\Editorable\ implementation. Our system lets users write more expressive custom
editor actions.

\section{Others}

A question we received from Edward Morehouse was ``What are the prospects for
building editor interactions into a compiler from the start?  Is it feasible to
implement delaboration in such a way that the compiler could respond directly
with surface-syntax terms that fit in the current binding context?'' We believe
we should discuss the existing work on languages that are designed
with editor interaction in mind.

Building editor interactions in a compiler from the start is not a new idea,
both Idris and Agda have done this already. They did not, however, take
metaprogrammable editor interactions into account, and that is what our work
brings to Idris. We believe a path through Racket, a language-oriented
programming~\cite{racketManifesto, feltey2016languages} language would be an
interesting take on building a language around its editor interactions.
DrRacket~\cite{drracket}, Racket's IDE, makes writing editor interaction easy
for the languages defined in Racket. This not only eliminates a lot of
boilerplate code, but it also allows using Racket itself to define new editor
actions. There are already dependently-typed languages defined in Racket: one
example is Cur\footnote{It can be found here:
\url{http://github.com/wilbowma/cur}}~\cite{cur}, a proof assistant with
powerful metaprogramming tools. There is also Pudding\footnote{It can be found
here: \url{http://github.com/david-christiansen/pudding}}, a proof assistant in
development that uses Racket for specifications, proof automation, code
extraction and also extensions to the proof assistant itself.
Another one is Pie\footnote{It can be found
here: \url{http://github.com/the-little-typer/pie}}~\cite{theLittleTyper}, a
minimal language used for educational purposes.
We believe there is potential
for stronger editor interaction for these languages through metaprogramming.

Another path that is worth exploring more is structure editors. In the proof
assistant world, The Alfa proof editor~\cite{alfa} has established a proof
interface based on structure editor manipulating proof trees. More recently and
for a simpler type theory, the Hazel project~\cite{hazelnut,hazelEditor}
explored what a language designed around its editor would look like.
Specifically, they designed a structure editor and a type theory to deal with
incomplete programs in this setting. They also coined the term ``edit-time'' to
mean when the user is writing a program in the editor, and suggested
``edit-time tactics'' as future work\footnote{We learned this from their slides
and also personal communication with Cyrus Omar and Ravi Chugh.}, by which they
meant a separate language in which users can define editor actions, and a
library of predefined editor actions that the users can compose.

The second part of Morehouse's question was about implementing delaboration in
a way that the compiler could respond directly with surface-syntax terms that
fit in the current binding context.
Currently there is no way in Idris to write an editor action that returns a
surface-syntax term.\footnote{ The only way around returning surface-syntax
directly from an editor action is to return a \ty{String} that consists of the
code, but that is inelegant and we would like to avoid that.} The way
elaborator reflection is defined in Idris forces us to deal with core language
terms only, and for the rest we depend on the built-in delaboration.  There is
also no reflected Idris type that represents the surface syntax, since the
surface syntax can change quite often, maintaining its Idris representation
would be difficult, not to mention with every change it would likely break
users' code that depends on it.  Therefore, adding an Idris representation of
the surface-syntax is not planned.

Apart from Idris, it is possible to design a language that lets the users
define editor actions that return surface-syntax terms. We see two possible ways to do this:
\begin{enumerate}
  \item Not having a core language and surface-syntax distinction. This is not
    ideal if you have a large programming language, then the type-checking,
    evaluation, etc. have to be extended every time we want to add a new
    syntax. Not to mention that lacking features like implicit arguments is bad
    language ergonomics; elaboration is needed to resolve the implicit
    arguments~\cite{pollack}.
  \item Having a reflected type in your language that represents the
    surface-syntax terms, exposing the delaboration mechanism in your
    metaprogramming mechanism, and allowing splicing surface-syntax terms into
    programs. We are not aware of any work that does this.
\end{enumerate}

\chapter{Conclusion}\label{chap:conclusion}

\section{Future work}

\subsection{Proof simplification}

The ability to run tactics as editor actions has a consequence
that we have not explored much in this thesis.
% Idris programs are usually meant to be executed, unlike Coq or
% Agda programs, which are usually only meant to be type checked.\footnote{Of
%   course there are backends for these languages, such as the OCaml
%   backend for Coq and the Haskell backend for Agda.}
% This means Idris programs will be compiled more often than Agda or Coq in the
% long run, since not much code is added to proofs after their completion, but
% practical programs tend to change in time. Therefore, minimizing compile time
% is more of a priority for Idris.
Idris tactics generate proof terms at compile time, but their
compilation can take a long time for complex tactics\footnote{Similar problems
arise in Coq as well. For example, theorems that use the famous \texttt{omega}
tactic that decides Presburger arithmetic~\cite{omega} take a long time to compile, and it
usually generates a huge proof term.},
not to mention that the implementation of elaborator
reflection in Idris has significant performance issues, as shown by \citet{leanmeta}.
Yet we still want to utilize complex tactics to generate proofs or terms.
Using edit-time tactics, one would run a tactic once from the editor, generate
the proof term and serialize and send that to the editor and put it back in the
file.
If we think of the differences between the traditions of writing the proof
terms directly and writing tactics, the former more common in Agda and Idris
and the latter in Coq, this work will constitute a one way bridge
between the two, by making use of the elaborator reflection to create proof
terms in the editor in a smarter and quicker way.

The problem with that approach is that the generated proof terms can be (and
often are) gigantic and hideous, especially if generating a minimal proof term
is not a priority for the tactic we are using.
If there was a generic mechanism to simplify and minimize the generated proof
terms, and even write them in a way that makes use of dependent pattern
matching, then this could be a more usable consequence of this work.
Ideally, we would want the artifact we are handing in to the reader of our
proofs to look just like what it would be if we had not used this system.
We leave that for future work.  However, even without proof simplification,
edit-time tactics still could be a last resort solution to long compile times
for tactics.

\subsection{Writing an editor action frontend in Idris}

We explained in \autoref{chap:introduction} that this thesis focuses on writing
the backend of an editor action in Idris, and that we still had to write some
Emacs Lisp (if we are using Emacs). However, Idris supports many different
code generation targets~\cite{idriscodegen} seamlessly.

For example, since compiling to JavaScript is built-in, we can use JavaScript
code generation to write the editor interaction frontend for Visual Studio Code
and Atom.

There are also experimental projects on compiling Idris to Emacs
Lisp\footnote{Steven Shaw's work on compiling Idris to Emacs Lisp:
\url{http://github.com/steshaw/idris-elisp}} and VimL
(Vimscript)\footnote{Oskar Wickstr\"om and Soham Chowdhury's work on
compiling Idris to VimL:
\url{https://github.com/owickstrom/idris-vimscript}}. These projects are not
mature enough yet, but we believe they have the potential to inspire different
applications of metaprogramming, especially if the Idris modes of these editors
are written in Idris via their respective code generation targets.

\section{Final words}

In this thesis, we extended the capabilities of the editor interaction mode of
Idris by allowing users to define new editor actions in Idris itself. We did
so through a metaprogramming technique that was introduced to Idris recently by
Christiansen and Brady~\cite{elabref}.

Editors communicate with the compiler via S-expressions, so we gave users the
power to dictate how a value of a given Idris type should exactly be
communicated; through the \ty{Editorable} interface users are now able to
define how a received S-expression should be parsed by the compiler, and how
the compiler should send the result as an S-expression. To achieve this, we
reflected the \ty{SExp} type to Idris, and extended elaborator reflection
by adding new \Elab\ primitives, with which we defined the \ty{Editorable}
implementations for Idris types representing the Haskell representation of
Idris core language terms.

Using this feature, we showed a simple \fn{toy} example, and then the
\fn{addClause} example that can replace an existing built-in editor action,
and Hezarfen, which is meant to be a better proof search mechanism than the
built-in one. We believe there is potential to replace even more of the
built-in editor actions with edit-time tactics, such as case-splitting and
lifting a hole into a lemma. We can also add new general edit-time tactics,
such as renaming a binder, renaming a function within a file, pruning unused
arguments in a function, etc.

We also believe that as more decision procedures are coded up in Idris,
edit-time tactics can become a more popular feature. Especially library and DSL
authors can ship custom editor actions for their package, which would allow
library users to write code more easily with that library or DSL.

Hopefully our work will bring dependently-typed languages one step closer to the
state-of-the-art IDEs, and even give them an edge by allowing the reuse of the
existing metaprogramming mechanisms and tactic engineering efforts to write
editor actions.

\bibliography{paper}
\bibliographystyle{plain}
\end{document}

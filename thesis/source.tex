\documentclass[11pt, ma]{westhesis}
\usepackage[utf8]{inputenc}
\usepackage{natbib, amsmath, amsfonts, amssymb, bussproofs, listings, float, upgreek, stmaryrd, epigraph, hyperref}
\usepackage[font=small]{caption}
\hypersetup{colorlinks = true, allcolors = {blue}}

% Idris
% in REPL run `:pp latex 80 <expr>` to get the colored verbatim
\usepackage{inconsolata}
\usepackage{fancyvrb}
\usepackage[usenames,dvipsnames]{xcolor}
\newcommand{\IdrisData}[1]{\textcolor{Maroon}{#1}}
\newcommand{\IdrisType}[1]{\textcolor{NavyBlue}{#1}}
\newcommand{\IdrisBound}[1]{\textcolor{Plum}{#1}}
\newcommand{\IdrisFunction}[1]{\textcolor{Green}{#1}}
\newcommand{\IdrisKeyword}[1]{{\textbf{#1}}}
\newcommand{\IdrisImplicit}[1]{{\itshape \IdrisBound{#1}}}
\usepackage{setspace, microtype}
\fvset{commandchars=\\\{\},formatcom=\singlespacing}

\newcommand{\quadthree}{\qquad\quad}
\newcommand{\quadfour}{\quadthree\quad}
\newcommand{\quadfive}{\quadfour\quad}
\newcommand{\quadsix}{\quadfive\quad}
\newcommand{\quadseven}{\quadsix\quad}
\newcommand{\quadeight}{\quadseven\quad}
\newcommand{\quadten}{\quadfive\quadfive}

% \allowdisplaybreaks

\department{Mathematics and Computer Science}
\submitdate{April 2018}
\advisor{Daniel R. Licata}
\title{Editor Interaction Through Metaprogramming in Idris}
\author{Joomy Korkut}

\theoremstyle{plain}
\newtheorem{theorem}{Theorem}[chapter]
\newtheorem{corollary}{Corollary}[theorem]
\newtheorem{lemma}[theorem]{Lemma}

\theoremstyle{definition}
\newtheorem{defn}{Definition}[chapter]

\newcommand{\T}[1]{\texttt{#1}}
\lstset{mathescape=true,basicstyle=\ttfamily}

\newcommand\numberthis{\addtocounter{equation}{1}\tag{\theequation}}

\begin{document}

\begin{abstract}
Metaprogramming is a technique that allows users to write programs that writes
programs. In dependently-typed languages such as Idris, recent work on the
elaborator reflection system paved the way for new applications of
metaprogramming, such as tactics. The goal of this work is to use the
elaborator reflection system to rewrite the editor interaction mode (or IDE
mode) of Idris. The current implementation is a part of the compiler and it
is written in Haskell. Rewriting it in Idris would allow us to make it a mere
library, and would be a decent first step towards a self-hosting Idris compiler.
Furthermore, it would allow users to write their own tactics through the
elaborator reflection, and run them in the editor. This would extend the
abilities of the editor interaction mode from the current limitations of the
built-in type checking holes, case splitting etc.\ to anything that a tactic
can do.
\end{abstract}

\begin{dedication}
\epigraph{``With metaprogramming, abstraction is no longer a guilty pleasure,
  it is simply a pleasure.''}{\textit{Jeremy Yallop, ICFP 2017}}
\end{dedication}

\begin{acknowledgements}
  % First and foremost, I would like to thank Dan Licata, my research advisor.
\end{acknowledgements}

\frontmatter
\maketitle
% \makededication
% \makeack
\makeabstract

\tableofcontents

\mainmatter
\section{Introduction} \label{sec:introduction}

Rich type systems give programmers a way to express their intentions
as types, statically ruling out many incorrect programs. But rich
types are useful for much more than preventing mistakes: the
information provided by informative types can be used by programming
tools to guide program construction, automating away tedious details
and freeing programmers to concentrate on the parts of their
problem that require human creativity.

Type-driven programming environments are necessarily built according
to language developers' assumptions about how programmers will use
them. These assumptions, however, can never hold for all members of a
diverse community working on a variety of problems. Unfortunately, the
interactive features of Idris and Agda are presently built in to their
respective compilers, and skill in dependently typed programming does
not imply the ability to extend the implementation of dependently
typed languages and maintain those extensions so that they continue to
work as compilers are improved.

The Idris elaborator~\citep{idris} translates programs written in
Idris into a smaller core type theory, called \textsf{TT}.
The elaborator is written
in Haskell, making use of an elaboration monad to track the
complicated state that is involved. The high-level Idris language is
extensible using \emph{elaborator reflection}~\citep{davidphd,
  elabref}, which directly exposes the elaboration monad to Idris
programs, so that Idris can be extended in itself. Concretely,
elaborator reflection extends Idris with a primitive monad
\Elab{}. Just as values in \IO{} describe effectful programs to be
executed by the run-time system, \Elab{} actions describe effectful
programs to be run during elaboration.

We have extended Idris's implementation of elaborator reflection with
new primitives that enable it to be used to construct \emph{editor
  actions}. These editor actions have access to the full power of
\Elab{}, but instead of running in the course of elaboration, they are
manually invoked by programmers to modify already-elaborated programs.
With these new primitives, it becomes possible to write
domain-specific editor actions for embedded domain-specific
languages~\citep{dsel} and to replace parts of the compiler with
customizable library code written in Idris. Even more importantly,
users who were previously stuck with whatever the developers provided
are now empowered to make not only their language, but also their
environment, their own.

% \subsection*{Contributions}

We make the following contributions in this paper:

\begin{itemize}[topsep=0pt] % , leftmargin=10pt]
\item We explore the features that are necessary to use
  \mbox{elaborator reflection} to implement editor actions.
\item We describe a concrete realization of this design, and the
  communication protocol that allows it to work in multiple
  interactive environments.
\item We describe a non-trivial editor action that invokes a
  theorem prover for intuitionistic propositional logic to
  interactively fill a hole in an incomplete
  program.
\item We demonstrate that editor actions written in Idris are
  sufficently powerful to replace parts of implementation by
  reimplementing a feature that constructs initial implementations of
  functions, based on their type signatures.
\end{itemize}
%there's space for two more lines on the first page


\subsection{Extending An Editor in Idris}

Dependently typed languages typically both allow programs to be
incomplete and provide support for making them more complete. A
limited version of this support could be a facility that substitutes
the unit constructor (written \dt{()}, as in Haskell) for a hole of
the unit type (also written \ty{()}, as in Haskell), and the
reflexivity constructor \dt{Refl} when the goal is a reflexive case of
the equality type.

\begin{figure}[b]
\begin{Verbatim}
\kw{%editor}
\fn{easy} : \ty{TTName} -> \ty{Elab} \ty{TT}
\fn{easy} \bn{n} =
  \kw{do} \bn{ty} <- \fn{getType} \bn{n}
     \kw{case} \bn{ty} \kw{of}
       \kw{`(}\ty{()} \kw{:} \ty{Type}\kw{)} \kw{=>}
         \fn{pure} \kw{`(}\dt{()} \kw{:} \ty{()}\kw{)}
       \kw{`(}\ty{(=)} \{A=\textasciitilde{}\bn{a}\} \{B=\textasciitilde{}\bn{b}\} \textasciitilde{}\bn{x} \textasciitilde{}\bn{y}\kw{)} \kw{=>}
         \kw{do} \fn{converts} \bn{a} \bn{b}
            \fn{converts} \bn{x} \bn{y}
            \fn{pure} \kw{`(}\dt{Refl} \{A=\textasciitilde{}\bn{a}\} \{x=\textasciitilde{}\bn{x}\}\kw{)}
       _ \kw{=>}
         \fn{fail} \dt{[}\dt{TextPart} \dt{"Cannot solve"}\dt{]}
\end{Verbatim}
\hrulefill
\begin{Verbatim}
(\kw{defun} \fn{idris-easy} ()
  \dt{"Invoke the first example action."}
  (\fn{interactive})
  (\fn{idris-elab-hole-arg}
   \dt{"easy"} (\fn{list} (\fn{idris-name-at-point}))))
\end{Verbatim}
  \caption{A simple editor action in Idris (top) and its \mbox{Emacs Lisp} support code (bottom)}
  \label{fig:motivating-example}
\end{figure}

Figure~\ref{fig:motivating-example} presents an implementation of this
editor action. The \kw{\%editor} keyword registers the declaration as
an editor action. Its type states that, when passed a representation
of a name from Idris's core language, it will produce a representation
of a term in Idris's core language, potentially having
elaboration-time side effects. It is passed a name because Idris holes
are identified by name.

The first step is to look up the type of the hole to be replaced,
using \fn{getType}, which takes a name and returns the type of the definition
associated with that name.
If the name is ambiguous, \fn{getType} fails.  Having discovered the name's
type, it then pattern-matches on said type, using Idris's quasiquotation
syntax~\citep{idrisQuotation}.

The first case to be considered is the unit type. In this pattern, a
type annotation is needed due to the Haskell-style overloading of the
double-parentheses. If the case is the unit type, the quoted form of
the unit constructor is returned with \fn{pure}, which is analogous to
Haskell's \fn{return}.

The second case to be considered is the equality type, which is
heterogeneous~\citep{mcbridephd} in the Idris standard library.  The
equality type requires two implicit arguments~\citep{pollack}, called
\bn{A} and \bn{B}, as well as explicit arguments \bn{x} and \bn{y}.
When \bn{A} and \bn{B} are the same type, and \bn{x} and \bn{y}
can be judged to be equal according to that type, \dt{Refl} proves the
equality. The \fn{converts} action checks whether two quoted terms are
judgmentally equal, and fails if they are not. Having checked that the
types and their inhabitants are equal, the second case returns \dt{Refl}.

The third and final case matches any other goal, and it
fails. Additional cases could be added on an \emph{ad hoc} basis, or a
more automatic approach could be taken. See \citet{davidphd} or
\citet{elabref} for a description of how to increase the level of
automation; this example is chosen to be easier to understand.

Each of Idris's editor actions requires a small amount of
editor-specific code to provide a user interface, and editor actions
written in Idris are no exception. With a suitable library, most
editing actions can be accommodated with fewer than five lines of
Emacs Lisp, and we expect the burden to be similar for other
extensible editors. Including in-editor documentation, this example
requires five lines of Emacs Lisp.

\begin{figure}[h]
\begin{BVerbatim}
\fn{ex1} : \ty{()}
\fn{ex1} = \hole{ex1_impl}

\fn{ex2} : \fn{not} \dt{False} = \dt{True}
\fn{ex2} = \hole{ex2_impl}

\fn{ex3} : \dt{False} = \dt{True}
\fn{ex3} = \hole{ex3_impl}
\end{BVerbatim}
\hspace{1em}
\begin{BVerbatim}
\fn{ex1} : \ty{()}
\fn{ex1} = \dt{()}

\fn{ex2} : \fn{not} \dt{False} = \dt{True}
\fn{ex2} = \dt{Refl}

\fn{ex3} : \dt{False} = \dt{True}
\fn{ex3} = \IdrisError{\hole{ex3_impl}}
\end{BVerbatim}

  \caption{Before and after invoking \fn{easy}}
  \label{fig:motivating-example-exec}
\end{figure}

Figure~\ref{fig:motivating-example-exec} displays the results of
executing this editor action on three holes. In the first two
examples, the program is completed automatically. In the third
example, however, an error is indicated because the underlying
\ty{Elab} action fails.

%%% Local Variables:
%%% mode: latex
%%% TeX-master: "source"
%%% End:

% \section{Future Work}

The story of dependently typed languages that can be reprogrammed in
themselves is only just beginning. Further developments can increase
the convenience and reliability of Idris's editor actions.

\subsection{Proof Simplification}
\label{sssec:simplification}

\citet{elabref} showed that elaborator reflection can be used as a
tactic language for interactive theorem proving. It is possible to use
\Elab{} tactics to define custom editor actions and reuse existing
proof automation efforts directly from the editor.

\Elab{} tactics generate a proof term during elaboration, but the
artifact is only a call to the tactic, which allows users to ignore
the proof terms generated by the tactics. However gigantic or hideous
the proof terms are, readers of the code will only see that the
tactics satisfy the goal, while the proof term
itself remains hidden. Many well-known proof automation procedures, such as Coq's
\mt{omega}~\cite{omega}, make use of this fact to hide large, complicated proof terms.  However, when using
\Elab{} tactics to define custom editor actions, the result of the
action is an expression that is visible to the user. Thus, brevity and
readability are desirable qualities in the proof terms generated by
those tactics.  Requiring all tactic authors to simplify their own
expressions qualities is burdensome, and it hampers the reuse of
existing tactics.  If there were a generic mechanism to simplify and
minimize generated proof terms, and even write them in a way that
makes use of dependent pattern matching, then existing tactics would become
much more useful for implementing editor actions. Ideally, a finished program that was written with
custom editor actions based on proof automation should be indistinguishable from one written without.

\subsection{A Universe of Actions}
\label{sssec:universeEncoding}

Section~\ref{ssec:typechecking} described how the Idris compiler
checks whether all components of an editor action type have an
instance of the \Editorable{} type class. However, it is not necessary
to implement this as an additional step during elaboration: it would suffice to
encode the allowed types of editor actions as a universe à la
Tarski~\cite{genericDep}.  The universe would include only those functions whose domains
have \Editorable{} instances and whose ranges are in the universe, as well as
other types that have \Editorable{} instances.
Figure~\ref{code:universe} demonstrates an implementation of this universe.


\begin{figure}[H]
\begin{Verbatim}
\kw{data} \ty{Act} : \ty{Type} \kw{where}
  \dt{Done} : (\bn{a} : \ty{Type}) -> \ty{Editorable} \bn{a} => \ty{Act}
  \dt{Arg} : (\bn{a} : \ty{Type}) -> \ty{Editorable} \bn{a} =>
     (\bn{ty} -> \ty{Act}) -> \ty{Act}

\fn{actTy} : \ty{Act} -> \ty{Type}
\fn{actTy} (\dt{Done} \bn{ty}) = \ty{Elab} \bn{ty}
\fn{actTy} (\dt{Arg} \bn{ty} \bn{f}) = (\bn{v} : \bn{ty}) -> \fn{actTy} (\bn{f} \bn{v})
\end{Verbatim}
\caption{Universe encoding of types feasible to be treated as editor actions.}
\label{code:universe}
\end{figure}

Figure~\ref{code:universe-example} shows how the type of the \fn{easy} editor
action from figure~\ref{fig:motivating-example} would change with this encoding.

\begin{figure}[H]
\begin{Verbatim}
\fn{easy} : \fn{actTy} (\dt{Arg} \ty{TTName} (\textbackslash\bn{n} => \dt{Done} \TT))
\fn{easy} \bn{n} = \cm{\{- elided, same as before -\}}
\end{Verbatim}
\caption{\fn{easy} rewritten as a universe encoded editor action.}
\label{code:universe-example}
\end{figure}

Observe that \mt{\fn{actTy} (\dt{Arg} \ty{TTName} (\dt{Done} \TT))} evaluates
to \mt{\ty{TTName} -> \Elab{} \TT}, therefore the definition of \fn{easy} does
not have to change.

The most important outcome of this change would be the increase in the
expressiveness of editor action types. The current implementation rules out
dependently typed editor actions, while this universe encoding would allow
them. Figure ~\ref{code:universe-dependent} shows a
hypothetical editor action that takes a vector of some length that contains
function names and returns a vector of the \emph{same} length that contains the types
found for the function names.

\begin{figure}[H]
\begin{Verbatim}
\fn{getTypes} : \fn{actTy} (\dt{Arg} \ty{Nat} (\textbackslash\bn{n} =>
                    \dt{Arg} (\ty{Vect} \bn{n} \ty{TTName}) (\textbackslash{}_ =>
                      \dt{Done} (\ty{Vect} \bn{n} (\ty{Maybe} \ty{TT})))))
\fn{getTypes} \bn{n} \bn{v} =
  \fn{for} \bn{v} (\textbackslash\bn{n} =>
    \kw{do} \bn{l} <- \fn{lookupTy} \bn{n}
       \kw{case} \bn{l} \kw{of}
         \dt{[(}_\dt{,} _\dt{,} \bn{ty}\dt{)]} => \fn{pure} (\dt{Just} \bn{ty})
         _ => \fn{pure} \dt{Nothing})
\end{Verbatim}
\caption{A dependently typed editor action that would be possible with the universe encoding.}
\label{code:universe-dependent}
\end{figure}

% I'm not sure if we need this paragraph... we can remove it later
However, writing editor actions with dependent data types would require writing
more complex \Editorable{} instances.  Figure ~\ref{code:editorable-vect} shows the
\Editorable{} instance for length-indexed vectors, which uses lists to denote
vectors and hence has to check if the lengths match in every deserialization.
\citet{foundations-dep-interop} provide a solution to this problem that could be
adopted in Idris.

\begin{figure}[H]
\begin{Verbatim}
\kw{implementation} \ty{Editorable} \bn{a}
            => \ty{Editorable} (\ty{Vect} \bn{n} \bn{a}) \kw{where}
  \fn{fromEditor} \{\bn{a}\} \{\bn{n}\} (\dt{SExpList} \bn{l}) =
    \kw{do} \bn{l'} <- \fn{traverse} (\fn{fromEditor} \{a = \bn{a}\}) \bn{l}
             \fn{<|>} \fn{fail} \dt{[}\cm{\{- elided -\}}\dt{]}
       \kw{case} \fn{decEq} (\fn{length} \bn{l'}) \bn{n} \kw{of}
         \dt{Yes} \bn{pf} =>
           \fn{pure} (\fn{replace} \{P = \textbackslash{}\bn{k} => \ty{Vect} \bn{k} \bn{a}\}
                         \bn{pf} (\fn{fromList} \bn{l'}))
         \dt{No} _ => \fn{fail} \dt{[}\cm{\{- elided -\}}\dt{]}
  \fn{fromEditor} _ = \fn{fail} \dt{[}\cm{\{- elided -\}}\dt{]}
  \fn{toEditor} \bn{v} = \fn{toEditor} (\fn{toList} \bn{v})
\end{Verbatim}
\caption{\Editorable{} instance for length-indexed vectors.}
\label{code:editorable-vect}
\end{figure}

\subsection{Surface-Language Syntax}

Editor actions presently accept and produce representations of \TT{},
rather than high-level Idris, which greatly simplifies the
implementation and maintenance of editor actions. For many
applications, this does not matter, because the \emph{meaning} of an
expression is more important than how it is written. In some cases,
however, this lack of expressive power might be a problem. For
instance, it is presently impossible to define an editor action that
converts a use of idiom brackets~\citep{Applicative} into the
equivalent \kw{do}-notation, as both expressions have the same
representation in the core language. In the future, it would be
interesting to explore representations of the syntax of high-level
Idris that are robust in the face of change and extension.



\section{Conclusion}\label{sec:conclusion}

In this paper, we extended the capabilities of the editor interaction mode of
Idris by allowing users to define new editor actions in Idris itself. We did
so through a metaprogramming technique that was introduced to Idris recently by
Christiansen and Brady~\cite{elabref}.

Editors communicate with the compiler via S-expressions, so we gave
users the power to dictate how a value of a given Idris type should
exactly be communicated; through the \ty{Editorable} interface users
are now able to define how a received S-expression should be parsed by
the compiler, and how the compiler should send the result as an
S-expression. To achieve this, we reflected the \ty{SExp} type to
Idris, and extended elaborator reflection by adding new \Elab\
primitives, with which we defined the \ty{Editorable} implementations
for Idris types representing the Haskell representation of Idris core
language terms. This demonstrates the value of directly reusing the
compiler's implementations.

We have demonstrated editor actions such as simple proof searches and
a DSL-specific action, as well as a demonstration of rewriting part of
Idris in itself. We hope that Hezarfen will eventually be a better
proof search than the built in action. We believe there is potential
to replace even more of the built-in editor actions with custom editor
actions written in Idris, such as case-splitting and lifting a hole
into a lemma. We can also add new general editor actions such as
renaming a binder, renaming a function within a file, pruning unused
arguments in a function, and so forth.

As the library of elaborator actions grows, more building blocks will
be available for custom editor actions. Even today, however, authors
of libraries and DSLs can include custom editor actions with their
packages, giving library and DSL authors access to power that was
previously reserved for compiler implementors.

If we are serious about type-driven interactive programming, we need
to give users the power to control not only their programming
language, but also their programming environment. Idris's editor
actions are one small step towards that goal.

%%% Local Variables:
%%% mode: latex
%%% TeX-master: "source"
%%% End:

\bibliography{paper}
\bibliographystyle{plain}
\end{document}

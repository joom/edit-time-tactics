\section{Applications} \label{sec:applications}

In this section, we present a custom domain-specific editor action, an editor
action that is meant to replace a built-in Idris IDE mode feature, and an
editor action that improves on Idris's proof search mechanism in specific
logic.

\subsection{Regular Expression Simplification}

One of the most promising benefits of our work is that it allows authors of
embedded domain-specific languages~\cite{dsel} (or eDSLs) to write domain-specific
custom editor actions that assist eDSL users.
One such language that most programmers are already familiar with is regular
expressions.

\begin{figure}[H]
\begin{Verbatim}
\kw{data} \ty{Regex} = \dt{Empty}
           | \dt{Epsilon}
           | \dt{Lit} \ty{Char}
           | \dt{Concat} \ty{Regex} \ty{Regex}
           | \dt{Or} \ty{Regex} \ty{Regex}
           | \dt{Star} \ty{Regex}
\end{Verbatim}
\caption{Definition of regular expressions.}
\label{code:regex}
\end{figure}

Mirroring the formal definition of regular expressions, the Idris definition of
regexes has constructors for $\varnothing$, $\varepsilon$, literal
characters, concatenation ($\cdot$), alternation ($|$) and the Kleene star ($*$),
as seen in \autoref{code:regex}.

The most common usage of regular expressions is to determine whether a string is in its language.
For that, users would have to write regular expression literals using the
\ty{Regex} data type. If the user wants to check whether the string \textit{``aaa''} is in
the language of the regular expression $a*$, then they would call
\mt{\fn{accepts} \dt{"aaa"} (\dt{Star} (\dt{Lit 'a'}))}.

However, there is no guarantee that the user would write the regular expression
in its simplest form. Especially for more complex regular expressions, it is
easy to overlook simpler versions. For instance, a user might write the regex
term \mt{\dt{Or} \dt{Epsilon} (\dt{Star} (\dt{Lit 'a'}))}, representing
$\varepsilon | a*$, instead \mt{\dt{Star} (\dt{Lit 'a'})}, representing $a*$.
A custom editor action can perform this simplification automatically.

For reasons of space, we will not explain the actual simplification algorithm, since there are
many external sources such as \citet{ortizRegex} and \citet{harperRegex}
 that do, and it is not essential to the understanding how the custom
editor action works.
We will take the function \mt{\fn{simplify} : \ty{Regex} -> \ty{Regex}} as a
given, and proceed to describe how it can be used to construct an editor
action.\footnote{Full source code is available at
\url{http://github.com/joom/edit-time-tactics/tree/master/code/regex}.}

The custom editor action to simplify regexes should consume a regex, returning a
potentially simpler regex. However, when the editor sends expressions to the compiler,
it sends them as strings containing snippets of code, which are then parsed and
elaborated in the compiler.  The editor action should therefore have the type
\mt{\TT{} -> \Elab{} \TT{}}. The input and output are not regular expressions;
they are ASTs of Idris code that represents regular expressions.

Simplification, however, is a function from \ty{Regex} to \ty{Regex}, not from
\ty{TT} to \ty{TT}.
At the same time, regular expression simplification is vastly easier to implement
on actual \ty{Regex}es, rather than on their quotations.
To implement the editor action, \autoref{code:regexQuote} defines two functions \fn{unquote} and \fn{quote}
that converts between \TT{} and the constructors of \ty{Regex}. The
conversion from \ty{Regex} to \ty{TT} in the \fn{quote} function cannot fail,
since all regexes must have a core language representation. However, the
conversion from \TT{} to \ty{Regex} can fail in the \fn{unquote} function,
since if \fn{unquote} can be given any core language term, including ones that
represent ill-typed Idris terms, ones that have types other than \ty{Regex}, or \ty{Regex}es
that contain free variables.

\begin{figure}
\begin{Verbatim}
\fn{unquote} : \ty{TT} -> \ty{Elab} \ty{Regex}
\fn{unquote} \qt{\dt{Empty}} = \fn{pure} \dt{Empty}
\fn{unquote} \qt{\dt{Epsilon}} = \fn{pure} \dt{Epsilon}
\fn{unquote} \qt{\dt{Lit} \antiqt\bn{c}} = \kw{do} \bn{c'} <- \fn{unquote} \bn{c}
                       \fn{pure} (\dt{Lit} \bn{c'})
\fn{unquote} \qt{\dt{Concat} \antiqt\bn{x} \antiqt\bn{y}} = \kw{do} \bn{x'} <- \fn{unquote} \bn{x}
                             \bn{y'} <- \fn{unquote} \bn{y}
                             \fn{pure} (\dt{Concat} \bn{x'} \bn{y'})
\fn{unquote} \qt{\dt{Or} \antiqt\bn{x} \antiqt\bn{y}} = \kw{do} \bn{x'} <- \fn{unquote} \bn{x}
                         \bn{y'} <- \fn{unquote} \bn{y}
                         \fn{pure} (\dt{Or} \bn{x'} \bn{y'})
\fn{unquote} \qt{\dt{Star} \antiqt\bn{x}} = \kw{do} \bn{x'} <- \fn{unquote} \bn{x}
                        \fn{pure} (\dt{Star} \bn{x'})
\fn{unquote} \bn{t} = \fn{fail} \dt{[}\cm{\{- elided -\}}\dt{]}

\fn{quote} : \ty{Regex} -> \ty{TT}
\fn{quote} \dt{Empty} = \qt{\dt{Empty}}
\fn{quote} \dt{Epsilon} = \qt{\dt{Epsilon}}
\fn{quote} (\dt{Lit} \bn{c}) = \qt{\dt{Lit} \antiqt(\fn{quote} \bn{c})}
\fn{quote} (\dt{Concat} \bn{x} \bn{y}) =
  \qt{\dt{Concat} \antiqt(\dt{quote} \bn{x}) \antiqt(\fn{quote} \bn{y})}
\fn{quote} (\dt{Or} \bn{x} \bn{y}) = \qt{\dt{Or} \antiqt(\fn{quote} \bn{x}) \antiqt(\fn{quote} \bn{y})}
\fn{quote} (\dt{Star} \bn{x}) = \qt{\dt{Star} \antiqt(\fn{quote} \bn{x})}
\end{Verbatim}
\caption{Functions to convert between values of the \ty{Regex} type and their representation in the core language.}
\label{code:regexQuote}
\end{figure}

The functions \fn{unquote}, \fn{quote}, and \fn{simplify} provide all the
building blocks needed to define the custom editor action, which are combined in
\autoref{code:regexElab} to make up the necessary \Elab{} action.

\begin{figure}[ht]
\begin{Verbatim}
\kw{\%editor}
\fn{simplifyInEditor} : \ty{TT} -> \ty{Elab} \ty{TT}
\fn{simplifyInEditor} \bn{t} =
  \kw{do} \bn{r} <- \fn{unquote} \bn{t}
     \fn{pure} (\fn{quote} (\fn{simplify} \bn{r}))
\end{Verbatim}
\hrulefill
\begin{Verbatim}
(\kw{defun} \fn{idris-simplify-regex} ()
  \dt{"Replace selection with simplified regex."}
  (\fn{interactive})
  (\kw{let*} ((\bn{regexp} (\fn{buffer-substring-no-properties}
                  (\fn{region-beginning})
                  (\fn{region-end})))
         (\bn{result} (\fn{idris-elab-edit}
                  \dt{"simplifyInEditor"} \bn{regexp})))
    (\fn{replace-region}
     (\fn{region-beginning}) (\fn{region-end}) \bn{result})))
\end{Verbatim}
\caption{Definition of the \Elab{} action for regex simplification, and the
  necessary Emacs Lisp support code to run it.}
\label{code:regexElab}
\end{figure}

The Emacs Lisp support code required to run the custom editor action gets the
region selected in the editor by the user, sends it as the first argument for
the \fn{simplifyInEditor} editor action, receives a response from the compiler,
and replaces the code in the response with the selected region.


\begin{figure}[ht]
\begin{Verbatim}
\kw{if} \fn{accepts} \dt{"aaa"} (\select{\dt{Or} \dt{Epsilon} (\dt{Star} (\dt{Lit 'a'}))})
  \kw{then} \cm{\{- elided -\}} \kw{else} \cm{\{- elided -\}}
\end{Verbatim}
  \vspace{1em}
\begin{Verbatim}
\kw{if} \fn{accepts} \dt{"aaa"} (\dt{Star} (\dt{Lit 'a'}))
  \kw{then} \cm{\{- elided -\}} \kw{else} \cm{\{- elided -\}}
\end{Verbatim}
\caption{Before and after invoking regular expression simplification}
  \label{fig:regex-example}
\end{figure}

\autoref{fig:regex-example} shows an example editor session using the regex
simplification editor action. The user selects a region containing an
expression and executes the Emacs Lisp function, which replaces the selected
expression with the simplified version of the same regex.

\subsection{Reimplementing the Built-In ``Add Clause'' Action}\label{sec:addClause}

Idris's editor modes support a built-in editor action called ``Add initial match
clause to type declaration.'' When the cursor is on the type signature of a
function that does not have any clauses, the user can run this editor action
and get an initial pattern clause for the function.
For instance, invoking the command on the declaration
\begin{Verbatim}
\fn{copy} : (\bn{n} : \ty{Nat}) -> \bn{a} -> \ty{Vect} \bn{n} \bn{a}
\end{Verbatim}
results in the clause
\begin{Verbatim}
\fn{copy} \bn{n} \bn{x} = \hole{copy_rhs}
\end{Verbatim}
which has a bound variable for each explicit argument in \fn{copy}'s type.

There is no longer any need to implement this feature in Haskell as
part of the compiler. This section describes the implementation of an
editor action in Idris itself that generates initial clauses for top-level
type declarations without implicit arguments or interface
constraints. A version that handles these additional features is
longer, but involves no additional concepts. The complete Idris code that
implements this editor action can be seen in \autoref{code:addClause}.

\begin{figure}[ht]
\begin{Verbatim}
\fn{collectTypes} : \ty{TT} -> (\ty{List} \ty{TT}, \ty{TT})
\fn{collectTypes} (\dt{Bind} _ (\dt{Pi} \bn{ty} _) \bn{t}) =
  \kw{let} \dt{(}\bn{xs}\dt{,} \bn{t'}\dt{)} = \fn{collectTypes} \bn{t} \kw{in}
  (\bn{ty} \fn{::} \bn{xs}, \bn{t'})
\fn{collectTypes} \bn{t} = \dt{([]}, \bn{t})

\kw{%editor}
\fn{addClause} : \ty{TTName} -> \ty{Elab} (\ty{FunClause} \ty{TT})
\fn{addClause} \bn{n} =
  \kw{do} \bn{ty} <- \fn{getType} \bn{n}
     \bn{env} <- \fn{getEnv}
     \bn{ty'} <- \fn{normalise} \bn{env} \bn{ty}
     \kw{let} \dt{(}\bn{argTys}\dt{,} \bn{retTy}\dt{)} = \fn{collectTypes} \bn{ty'}
     \bn{argNames} <- \fn{traverse} (\fn{const} \fn{fresh}) \bn{argTys}
     \kw{let} \bn{lhsUntyped} =
       \fn{foldl} \dt{RApp} (\dt{Var} \bn{n}) (\fn{map} \dt{Var} \bn{argNames})
     \dt{(}\bn{lhsTyped}\dt{,} _\dt{)} <- \fn{check} \bn{env} \bn{lhsUntyped}
     \bn{holeName} <- \fn{fresh}
     \kw{let} \bn{rhs} = \dt{Bind} \bn{holeName} (\dt{GHole} \bn{retTy}) (\dt{V} \dt{0})
     \fn{pure} (\fn{MkFunClause} \bn{lhsTyped} \bn{rhs})
\end{Verbatim}
\caption{Implementation of the editor action for ``add clause''.}
\label{code:addClause}
\end{figure}

The \fn{collectTypes} function takes a type and dissects it into components, and returns
a pair of the list of inputs and the output type. For instance,
\mt{\fn{collectTypes} \kw{\`{}(}\ty{Nat} -> \ty{Bool} -> \ty{String}\kw{)}}
returns \mt{\dt{(}\dt{[}\kw{\`{}(}\ty{Nat}\kw{)}\dt{,}
\kw{\`{}(}\ty{Bool}\kw{)}\dt{]}\dt{,}\kw{\`{}(}\ty{String}\kw{)}\dt{)}}.

The \fn{addClause} action only takes one input, which is the name of
the function declaration for which an initial clause has been
requested.  Using this name, it looks up the type of that function,
normalizes the type, and gets its components using \fn{collectTypes}.
The list of input types is named \bn{argTys}, and the output type is
named \bn{retTy}. For each of member of \bn{argTys}, it generates a
new user-accessible name using \fn{fresh}. A more featureful
implementation would attempt to preserve names from the type
signature, only generating fresh names when the user had not provided
a name or in the presence of shadowing.

A pattern match clause consists of a left-hand side, which is an
application of the function being defined to either constructors or
pattern variables, and a right-hand side, which is the expression that
results when the pattern on the left-hand side matches.  Having found
names for each pattern variable, the left hand side of the initial
clause is constructed by applying the function being defined, using
\dt{RApp}.  The \dt{Var} constructor injects names into terms.  The
right hand side of the initial clause should consist only of a hole
for the user to fill in, indicated by the \dt{GHole} term constructor.
Because Idris holes are binding forms, the de~Bruijn index \dt{0}
refers back to this new hole.


The Emacs Lisp code necessary to run \fn{addClause} as a custom editor
action is the almost identical to the existing add-clause editor
action.  The only difference is that the call to the primitive
add-clause editor action in the IDE protocol is replaced by a call to
\fn{addClause}.

Using this editor action on the declaration
\begin{Verbatim}
\fn{copy} : (\bn{n} : \ty{Nat}) -> \bn{a} -> \ty{Vect} \bn{n} \bn{a}
\end{Verbatim}
results in an initial match clause
\begin{Verbatim}
\fn{copy} \bn{a} \bn{b} = \hole{c}
\end{Verbatim}
which was just as expected. However, this new version is much more
readily extensible by users.

\subsection{A Theorem Prover for Intuitionistic Propositional Logic}\label{sec:hezarfen}

In this section, we describe the procedure Hezarfen,\footnote{The name is
  pronounced ``has are fan'', and it means
    polymath in Turkish. Source code is available at
    \url{http://github.com/joom/hezarfen}.} which can decide intuitionistic
propositional logic theorems, similar to Coq's \texttt{tauto} tactic.
This procedure will be based on Dyckhoff's LJT~\cite{ljt} and its Haskell
implementation Djinn~\cite{djinn}, which generates Haskell expressions
for a given type.
Djinn is a standalone program that takes commands
interactively, and when it generates an expression it prints it on the screen.
Instead, Hezarfen is a library that provides an \Elab\ action
that can be used as a tactic in proofs, and a custom editor action to be run
when the built-in proof search mechanism does not suffice.

The prover consists of mutually recursive functions that try to break
the goal type down into components, recursively finds terms that satisfy the
components, and then glues them together based on the initial matched goal
type.

Later in the prover there is also a term simplifier, similar to Haskell's
pointfree style
converter.\footnote{\url{http://hackage.haskell.org/package/pointfree}} It performs
\mbox{$\eta$-reduction}, removes unused \kw{let} bindings, and similar
simplification steps repeatedly until it reaches a fixed point.
However, this simplifier is tailored for Hezarfen's proof terms; it is not
general-purpose.  The necessity of further work on a general purpose one is
discussed in section~\ref{sssec:simplification}.

\begin{figure}[h]
\begin{Verbatim}
\fn{comm} : (\bn{a}, \bn{b} : \ty{Type}) -> \ty{Either} \bn{a} \bn{b} -> \ty{Either} \bn{b} \bn{a}
\fn{comm} = \hole{comm_impl}
\end{Verbatim}
  \vspace{1em}
\begin{Verbatim}
\fn{comm} : (\bn{a}, \bn{b} : \ty{Type}) -> \ty{Either} \bn{a} \bn{b} -> \ty{Either} \bn{b} \bn{a}
\fn{comm} = \textbackslash\bn{x}, \bn{y} => \fn{either} \dt{Right} \dt{Left}
\end{Verbatim}
\caption{Before and after invoking Hezarfen}
  \label{fig:hezarfen-example}
\end{figure}

Figure~\ref{fig:hezarfen-example} displays the results of
executing this editor action on a hole:
Hezarfen finds a term with the desired type in which \fn{either} is a non-dependent eliminator for \ty{Either}.
Observe that the type of \fn{comm} corresponds to the proposition
$(a \lor b) \to (b \lor a)$, and by finding the term, Hezarfen proves
the proposition.

%%% Local Variables:
%%% mode: latex
%%% TeX-master: "source"
%%% End:

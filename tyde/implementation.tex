\section{Implementation Concerns}\label{sec:implementation}

% \TODO{The main goal of this section will be to provide enough information about the implementation so that someone else can add this feature to their dependently typed language. What were the big challenges? Think of things like the need for the local context to be preserved and how that gets associated with commands, as well as document the interesting challenges that arose.}

The overall design described in section~\ref{sec:design} is not
completely sufficient to implement extensible type-directed
editing. Some additional machinery proves to be necessary in practice.

\subsection{Local Contexts}
\label{ssec:localContext}

Expressions can be understood only in their local context that
explains the types, and sometimes the values, of their free
variables. Editor actions should have access to the local context of
bound variables in addition to the global definition context. The
design in section~\ref{sec:design}, however, has no means of providing
them with this local context.

For example, the editor might send an expression like \mt{\fn{not}
  \bn{a}} to the compiler, where \bn{a} is bound in the local context.
The compiler can parse this expression, but it cannot elaborate it,
because without the local context, \bn{a} is meaningless. In a call to
a custom editor action, expressions stand alone; they do not come with
a context. It can deal with \fn{not}, which is defined in the global
context, but when it comes to the local context, elaboration is doomed
to fail.

Editor actions have, thus far, been provided with their arguments
explicitly. However, local contexts do not have a concrete
representation in Idris's syntax, so they cannot be selected directly.

To solve this problem, we take advantage of the fact that lexical
contexts correspond closely to source spans. Each local binding form
has a defined scope; this scope corresponds to a production of the
abstract syntax tree that originates from a specific range of
positions in the editor buffer. We extended the protocol for custom
editor actions so that the editor sends a source position along with
the action name and its arguments.

The next step was to extend the internal compiler state with an
interval map that connects ranges of source positions to the local
context that corresponds to the process of elaborating the expression
found in that range. We used the standard Haskell finger tree
implementation of interval maps~\cite{fingertrees}.

When initializing the reflected elaboration monad prior to executing
an editor action, the local context is initialized with the one that
corresponds to the cursor location sent by the editor. The compiler
can use that information in the elaboration of terms depending on the
local context, such as \mt{\fn{not} \bn{a}}. Note that this constrains
editor actions to have a single privileged source position. This
constraint has not proven difficult in practice, but it could be
lifted by making source positions \ty{Editorable} and providing any
number of them as ordinary arguments. Editor actions could then have a
primitive to enter the lexical scope corresponding to a source
position, and to check whether a source span is contained within a
particular scope.

We expect that the remembered association between source spans and
contexts will enable additional editor features, such as displaying
the local context as the user navigates a source file, but we have not
yet implemented these features.

\subsection{Hard-Coding \Editorable{} Instances}

\begin{figure*}
\newcommand\mlnode[1]{
  \begin{minipage}{3.3cm}
    \linespread{1}\selectfont \begin{center}\small #1 \end{center}
  \end{minipage}}
\begin{tikzcd}
  \mlnode{
    Haskell terms\\ \medskip \footnotesize such as the \ty{()} term\\
    \dt{()}
  }
  \arrow[rrr, "\text{\small reflection}" description, bend right=10]
  &  &  &
  \mlnode{
    Haskell representation of Idris core language terms\\ \medskip \scriptsize
    such as the \ty{Term} term\\
    \texttt{\dt{P} (\dt{DCon} \dt{0} \dt{0} \dt{False}) (\dt{UN} \dt{"MkUnit"}) (\dt{P} (\dt{TCon} \dt{0} \dt{0 \dt{False}}) (\dt{UN} \dt{"Unit"}) \dt{Erased})}
  }
  \arrow[lll, "\text{\small reification}" description, bend right=10] \\
  &  &  &\\
  \mlnode{
    Idris terms\\ \medskip \footnotesize such as the \ty{()} term\\
    \dt{()}
  }
  \arrow[r, "\text{\small elaboration}" description, bend right=50]
  &
  \mlnode{
    Idris core terms\\ \medskip \footnotesize such as the \ty{Unit} term\\
    \dt{MkUnit}
  }
  \arrow[l, "\text{\small delaboration}" description, bend right=50]
  \arrow[rr, "\text{\small quotation}" description, bend right=35]
  \arrow[uurr, "\text{\small \emph{internally represented as}}" description, dashed, bend left=25]
  & &
  \mlnode{
    Idris representation of the Haskell representation of Idris core language terms\\
    \medskip \scriptsize such as the \TT\ term\\
    \texttt{\dt{P} (\dt{DCon} \dt{0} \dt{0}) (\dt{UN} \dt{"MkUnit"})\\
    (\dt{P} (\dt{TCon} \dt{0} \dt{0}) (\dt{UN} \dt{"Unit"}) \dt{Erased})}
  }
  \arrow[ll, "\text{\small unquotation}" description, bend right=35]
  \arrow[uu, leftrightarrow, "\text{\small \emph{correspondence}}" description, dashed]
\end{tikzcd}
\caption{The relationship between reflection, reification, quotation,
  unquotation, elaboration and delaboration.}
\label{reflectionGraph}
\end{figure*}

Implementing hard-coded instances of the \Editorable{} type class in the
compiler is challenging to describe since there are many languages involved in
different ways. Idris's compiler is written in Haskell, hence there is a
Haskell data type that represents Idris syntax trees. Idris's
elaborator~\citep{idris} describes a core language that is smaller than Idris's
high-level language, there is also a Haskell data type that represents Idris
core syntax trees.

However, elaborator reflection~\citep{davidphd, elabref} provides new Idris
data types that correspond to the Haskell data types to represent Idris core
language terms.
Outlining how a metaprogramming feature is implemented is one level more meta,
therefore it becomes difficult to use precise terminology.
\autoref{reflectionGraph} describes the relationship between the different
languages and representations and the spells out the specific names for moving
from one to another.

During the execution of \fn{prim\_\_fromEditor} in the elaborator, it is given
\bn{arg}, a Haskell representation of an Idris core language term representing
an \sexp{}, and \bn{ty}, a Haskell representation of an Idris type.
When the elaborator finishes running \fn{prim\_\_fromEditor}, it should create
the Haskell representation of an Idris core language term, which should have the
type \bn{ty}. The elaborator should have a case for each primitive
\Editorable{} implementation. For brevity, we will only consider the case in
which \bn{ty} corresponds to the Idris type \TT{}. Then the elaborator should

\begin{enumerate}
  \item reify \bn{arg} to get a Haskell \sexp{} and make sure it is a string atom
  \item parse the string inside the \sexp{} and get a high-level language term
  \item traverse the high-level language term and resolve namespaces for names, for when it is unambiguous
  \item elaborate the new high-level language term into a core language term, using the local context obtained through section~\ref{ssec:localContext}
  \item reflect the core language term, in order to create a core language term that represents an Idris term of the type \TT{}
  \item normalize the reflected term to get a syntax tree in canonical form, and return
\end{enumerate}

During the execution of \fn{prim\_\_toEditor} in the elaborator, it is given
\bn{ty}, a Haskell representation of an Idris type, and \bn{arg},
\bn{arg}, a Haskell representation of an Idris core language term representing
a term of the type \bn{ty}.
When the elaborator finishes running \fn{prim\_\_toEditor}, it should create
the Haskell representation of an Idris core language term representing an
\sexp{}. The \TT{} case for \fn{prim\_\_toEditor} should

\begin{enumerate}
  \item reify \bn{arg} to get a Haskell representation of an Idris core language term
  \item delaborate and resugar the core language term into a high-level language term
  \item use pretty printing to get a string that is a piece of code
  \item create a string atom \sexp{} with that string
  \item reflect the \sexp{} to get a Haskell term representing an Idris core language term representing an \sexp{}
  \item normalize the reflected term to get a syntax tree in canonical form, and return
\end{enumerate}

%%% Local Variables:
%%% mode: latex
%%% TeX-master: "source"
%%% End:

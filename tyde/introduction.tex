\section{Introduction} \label{sec:introduction}

Rich type systems give programmers a way to express their intentions
as types, statically ruling out many incorrect programs. But rich
types are useful for much more than preventing mistakes: the
information provided by informative types can be used by programming
tools to guide program construction, automating away tedious details
and freeing programmers to concentrate on the parts of their
problem that require human creativity.

Type-driven programming environments are necessarily built according
to language developers' assumptions about how programmers will use
them. These assumptions, however, can never hold for all members of a
diverse community working on a variety of problems. Unfortunately, the
interactive features of Idris and Agda are presently built in to their
respective compilers, and skill in dependently typed programming does
not imply the ability to extend the implementation of dependently
typed languages and maintain those extensions so that they continue to
work as compilers are improved.

The Idris elaborator~\citep{idris} translates programs written in
Idris into a much smaller core type theory. The elaborator is written
in Haskell, making use of an elaboration monad to track the
complicated state that is involved. The high-level Idris language is
extensible using \emph{elaborator reflection}~\citep{davidphd,
  elabref}, which directly exposes the elaboration monad to Idris
programs, so that Idris can be extended in itself. Concretely,
elaborator reflection extends Idris with a primitive monad
\Elab{}. Just as values in \IO{} describe effectful programs to be
executed by the run-time system, \Elab{} actions describe effectful
programs to be run during elaboration.
% \todo{insert ref to lean paper elsewhere}

We have extended Idris's implementation of elaborator reflection with
new primitives that enable it to be used to construct \emph{editor
  actions}. These editor actions have access to the full power of
\Elab{}, but instead of running in the course of elaboration, they are
manually invoked by programmers to modify already-elaborated programs.
With these new primitives, it becomes possible to write
domain-specific editor actions for embedded domain-specific
languages~\citep{dsel} and to replace parts of the compiler with
customizable library code written in Idris. Even more importantly,
users who were previously stuck with whatever the developers provided
are now empowered to make not only their language, but also their
environment, their own.

% \subsection*{Contributions}

We make the following contributions in this paper:

\begin{itemize}[topsep=0pt] % , leftmargin=10pt]
\item We explore the features that are necessary to use
  \mbox{elaborator reflection} to implement editor actions.
\item We describe a concrete realization of this design, and the
  communication protocol that allows it to work in multiple
  interactive environments.
\item We describe a non-trivial editor action that invokes a
  theorem prover for intuitionistic propositional logic to
  interactively fill a hole in an incomplete
  program.
\item We demonstrate that editor actions written in Idris are
  sufficently powerful to replace parts of implementation by
  reimplementing a feature that constructs initial implementations of
  functions, based on their type signatures.
\end{itemize}
%there's space for two more lines on the first page


\subsection{Motivating Example}

Dependently typed languages typically both allow programs to be
incomplete and provide support for making them more complete. A
limited version of this support could be a facility that substitutes
the unit constructor (written \dt{()}, as in Haskell) for a hole of
the unit type (also written \ty{()}, as in Haskell), and the
reflexivity constructor \dt{Refl} when the goal is a reflexive case of
the equality type.

\begin{figure}[b]
\begin{Verbatim}
\kw{%editor}
\fn{easy} : \ty{TTName} -> \ty{Elab} \ty{TT}
\fn{easy} \bn{n} =
  \kw{do} \dt{(}_\dt{,} _\dt{,} \bn{ty}\dt{)} <- \fn{lookupTyExact} \bn{n}
     \kw{case} \bn{ty} \kw{of}
       \kw{`(}\ty{()} \kw{:} \ty{Type}\kw{)} \kw{=>}
         \fn{pure} \kw{`(}\dt{()} \kw{:} \ty{()}\kw{)}
       \kw{`(}\ty{(=)} \{A=\textasciitilde{}\bn{a}\} \{B=\textasciitilde{}\bn{b}\} \textasciitilde{}\bn{x} \textasciitilde{}\bn{y}\kw{)} \kw{=>}
         \kw{do} \fn{converts} \bn{a} \bn{b}
            \fn{converts} \bn{x} \bn{y}
            \fn{pure} \kw{`(}\dt{Refl} \{A=\textasciitilde{}\bn{a}\} \{x=\textasciitilde{}\bn{x}\}\kw{)}
       _ \kw{=>}
         \fn{fail} \dt{[}\dt{TextPart} \dt{"Cannot solve"}\dt{]}
\end{Verbatim}
\hrulefill
\begin{Verbatim}
(\kw{defun} \fn{idris-easy} ()
  \dt{"Invoke the first example action."}
  (\fn{interactive})
  (\fn{idris-elab-hole-arg}
   \dt{"easy"} (\fn{list} (\fn{idris-name-at-point}))))
\end{Verbatim}
  \caption{A simple editor action in Idris (top) and its \mbox{Emacs Lisp} support code (bottom)}
  \label{fig:motivating-example}
\end{figure}

Figure~\ref{fig:motivating-example} presents an implementation of this
editor action. The \kw{\%editor} keyword registers the declaration as
an editor action. Its type states that, when passed a representation
of a name from Idris's core language, it will produce a representation
of a term in Idris's core language, potentially having
elaboration-time side effects. It is passed a name because Idris holes
are identified by name.

The first step is to look up the type of the hole to be replaced,
using \fn{lookupTyExact}, which returns a triple consisting of the
fully-qualified version of the name, whether it is a local variable,
top-level reference, data constructor, or type constructor, and the
name's type. If the name is ambiguous, \fn{lookupTyExact} fails.
Having discovered the name's type, it then pattern-matches on said
type, using Idris's quasiquotation syntax~\citep{idrisQuotation}.

The first case to be considered is the unit type. In this pattern, a
type annotation is needed due to the Haskell-style overloading of the
double-parentheses. If the case is the unit type, the quoted form of
the unit constructor is returned with \fn{pure}, which is analogous to
Haskell's \fn{return}.

The second case to be considered is the equality type, which is
heterogeneous~\citep{mcbridephd} in the Idris standard library.  The
equality type requires two implicit arguments~\citep{pollack}, called
\bn{A} and \bn{B}, as well as explicit arguments \bn{x} and \bn{y}.
When \bn{A} and \bn{B} are the same type, and \bn{x} and \bn{y} are
judgmentally equal according to that type, \dt{Refl} proves the
equality. The \fn{converts} action checks whether two quoted terms are
judgmentally equal, and fails if they are not. Having checked that the
types and their inhabitants are equal, the second case returns \dt{Refl}.

The third and final case matches any other goal, and it
fails. Additional cases could be added on an \emph{ad hoc} basis, or a
more automatic approach could be taken. See \citet{davidphd} or
\citet{elabref} for a description of how to increase the level of
automation; this example is chosen to be easier to understand.

Each of Idris's editor actions requires a small amount of
editor-specific code to provide a user interface, and editor actions
written in Idris are no exception. With a suitable library, most
editing actions can be accommodated with fewer than five lines of
Emacs Lisp, and we expect the burden to be similar for other
extensible editors. Including in-editor documentation, this example
requires five lines of Emacs Lisp.

\begin{figure}[h]
\begin{BVerbatim}
\fn{ex1} : \ty{()}
\fn{ex1} = \hole{ex1_impl}

\fn{ex2} : \fn{not} \dt{False} = \dt{True}
\fn{ex2} = \hole{ex2_impl}

\fn{ex3} : \dt{False} = \dt{True}
\fn{ex3} = \hole{ex3_impl}
\end{BVerbatim}
\hspace{1em}
\begin{BVerbatim}
\fn{ex1} : \ty{()}
\fn{ex1} = \dt{()}

\fn{ex2} : \fn{not} \dt{False} = \dt{True}
\fn{ex2} = \dt{Refl}

\fn{ex3} : \dt{False} = \dt{True}
\fn{ex3} = \IdrisError{\hole{ex3_impl}}
\end{BVerbatim}

  \caption{Before and after invoking \fn{easy}}
  \label{fig:motivating-example-exec}
\end{figure}

Figure~\ref{fig:motivating-example-exec} displays the results of
executing this editor action on three holes. In the first two
examples, the program is completed automatically. In the third
example, however, an error is indicated because the underlying
\ty{Elab} action fails.

% Some of the features we implemented in this paper have already made their way to
% the Idris compiler, and the rest also will once they are reviewed by the
% other Idris contributors.

% We envision three different audiences for this paper:
% \begin{enumerate}[(1)]
% \item Idris programmers who use editor actions in their editor, who will now
%   have access to more editor actions. For them, reading the applications of
% edit-time tactics in \autoref{sec:applications} would be the most helpful.
% \item Advanced Idris programmers who want to write simple editor actions, using
%   the common Idris types and reflected types. More advanced Idris programmers
%     may want to write more complex editor actions that involve data types
%     that they define.
%     They may want to read \autoref{sec:design} in order to understand the
%     design of our feature and what should be taken into account when defining
%     such data types.
% \item Compiler developers and contributors for Idris and other
%   dependently-typed languages.  They may want to read
%     \autoref{sec:implementation} in order to observe what we needed to change
%     in the compiler, and how they can add this feature to a different language.
% \end{enumerate}

%%% Local Variables:
%%% mode: latex
%%% TeX-master: "source"
%%% End:

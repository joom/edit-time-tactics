\section{Related work} \label{sec:relatedwork}

Type-directed editing and metaprogramming are not unique to Idris.


\subsection{In Lean}

Lean~\cite{lean}, which has a tactic metaprogramming
system~\cite{leanmeta} that is similar to Idris's elaborator
reflection, also supports the use of tactics to implement editor
actions.  Lean's editor actions work only in the context of a
hole. They allow the contents of the hole to be transformed into an
arbitrary string, which replaces the hole. Using the pretty-printing
features of the tactic system, terms can be placed in holes.

Because Lean's editor actions only work in the context of a hole, and
can only take quoted terms as arguments, no Emacs Lisp is necessary to
invoke them. The user simply right-clicks a hole, and a list of
commands appears.  In comparison to our custom editor action mechanism
presented in our work, Lean's system is less expressive, but more
convenient.  It only allows editor action that run on holes, but our
system allows any kind of editor action as long as the user writes the
necessary glue code in the editor mode language.
A system like Lean's hole commands could be implemented as a small
extension to Idris's editor actions that allows them to be specially
registered and imposes the same restrictions on the action types.

\subsection{In Haskell}

Template Haskell~\cite{th} is the primary metaprogramming mechanism in
Haskell.  It is similar to elaborator reflection in the sense that
metaprograms are defined in a monad called \ty{Q}, which allows
metaprograms to create fresh names and look up definitions.  Unlike
elaborator reflection, Template Haskell does not expose the
general-purpose elaboration mechanisms (such as GHC's constraint
solver) through \ty{Q}.  Template Haskell metaprograms generate only
expressions and definitions.

Brian McKenna, however, has implemented a
simplifier\footnote{\url{http://hackage.haskell.org/package/th-pprint}}
for the output of Template Haskell and arranged for the simplified
code to be inserted into Emacs automatically. With further
development, this feature could eventually gain the expressive power
of Idris editor actions.

\subsection{In Agda}

\citet{agsy} introduced a term search algorithm called
Agsy that saves users' time by automating
parts of a proof or program that are straightforward but tedious to write.
Agsy is used regularly by Agda users.
\citet{autoinagda} used Agda's prior reflection system to define a
new proof search mechanism in Agda itself.
The Hezarfen editor action we discussed in section~\ref{sec:hezarfen} is not as advanced
as their \fn{auto} function, yet in their paper, they suggested an IDE feature
that replaces a call to their \fn{auto} with the proof terms it generates.
We generalized their suggestion to all \Elab{} procedures, and specified
how the editor/IDE and the compiler should communicate with each other
in order to successfully call a ``tactic'' with inputs of the correct types.

\subsection{In Coq}

Coq has a metaprogramming mechanism called
\texttt{template-coq}\footnote{\url{https://github.com/Template-Coq/template-coq}}
that is based on Malecha's term
reification~\cite{malecha-phd}. Recently, a typed version of this
system was also introduced~\cite{typed-template-coq}, making it easier
to write reliable code that uses quotations.  However, we are not
aware of any work on using template metaprograms in Coq to write new
features for the editor.

\subsection{Other Languages}

Not every new language is conceived of as primarily a mapping from the
set of strings to the disjoint union of machine code and error
messages, with its users and tooling as an afterthought. Some are
designed from the start with a customizable interactive environment in
mind. This tradition dates back to early work on Lisp, particularly
the Lisp machines and Interlisp-D~\citep{LispHist}, as well as
Smalltalk~\citep{Goldberg1984SmalltalkEnv}. These environments are
highly customizable, but they do not allow users to continue to use
their preferred editors. Idris now occupies a space between the total
freedom of Smalltalk and a language such as Haskell for which editor
support is an afterthought.

Racket is a language that focuses on the paradigm of
\emph{language-oriented programming}~\citep{racketManifesto}, in which
problems are solved by first constructing the most appropriate
language to solve them. One part of this process is extensible,
metaprogrammable tooling, especially the DrRacket~\cite{revenge}
IDE. For instance, \citet{feltey2016languages} demonstrate a
surprisingly concise Java-like language, including refactoring tools.
It is certainly possible to implement dependently typed languages in
the Racket ecosystem: both Cur~\cite{cur} and
Pie~\cite{theLittleTyper} already exist, and the latter includes a
simple editor action system that is presently extensible only in
Racket but could support other languages as well.

Structured editors are an alternative means of interacting with a
programming language.  Alfa is a structural proof editor~\cite{alfa},
descended from an earlier system called Alf~\citep{ALF, ALFthesis}.
These structure editors are not, however, customizable using programs
written in their type theories. Likewise, while
Epigram~\citep{epigram} supported type-driven structured editing, it
was not extensible in itself. The ongoing Hazel
project~\cite{hazelnut,hazelEditor} employs the tools of programming
language theory to describe interactions with a type-aware structured
editor; however, they have not yet reflected this language of
interactions back into their object language.

% They also coined the term ``edit-time'' to
% mean when the user is writing a program in the editor, and suggested
% ``edit-time tactics'' as future work, by which they
% meant a separate language in which users can define editor actions, and a
% library of predefined editor actions that the users can compose.


%%% Local Variables:
%%% mode: latex
%%% TeX-master: "source"
%%% End:

\section{Design}\label{sec:design}

When the user invokes an editor action, the editor has to tell the Idris
compiler what to run. Since the editor and the compiler run in different processes,
for each interaction the editor has to send a message to the compiler,
and the compiler has to send a message back to the editor.
These messages are formatted as \citeauthor{mccarthy}'s \sexp{}s~\citep{mccarthy}.

The \Elab{} monad in Idris primitively keeps track of a state
involving a potentially incomplete expression, its type, and any new declarations generated as
side effects during elaboration.
When an \Elab{} script is executed, the incomplete expression is expected to have been completed.
Because these updates to the expression occur via side effects, elaborator reflection scripts have the type \mbox{\ty{Elab ()}}. Since the
desired metaprogramming effects are captured by the elaboration state, there is
nothing interesting to return.

However, \Elab{} scripts that are used as editor actions are not able to effect changes to the program by modifying the elaboration state, because the contents of the text editor are not part of the state.
Thus, editor actions return their results explicitly, and the serialized results are sent to the editor.

If an editor action needs to send back an expression to the editor, then the
action should have the return type \mbox{\ty{Elab TT}}, where \ty{TT} is the type of
quoted core Idris terms.
Similarly, if a user needs to define an action that creates a function definition,
then the action that does that should have the return type \mbox{\ty{Elab FunDefn}},
where \ty{FunDefn} is the type of quoted function definitions.
A simple editor action that only needs to send a number back
to the editor should return an \mbox{\ty{Elab Nat}}, where \ty{Nat} is the
type of natural numbers.

\subsection{The \Editorable{} Type Class}
\label{ssec:editorable}

The problem with allowing editor actions to return inhabitants of any
type is that the compiler cannot serialize values of arbitrary types
as \sexp{}s.  In order
to give users the power to define how each type should
be represented as \sexp{}s, we define a type class\footnote{In Idris,
  type classes are called \emph{interfaces} and instances are called
  \emph{implementations}.}  called \Editorable{}, which outlines what
the compiler needs to know about a type to be able to serialize and
deserialize values of that type.

\begin{figure}[H]
\begin{Verbatim}
\kw{interface} \ty{Editorable} \bn{a} \kw{where}
  \fn{fromEditor} : \ty{SExp} -> \ty{Elab} \bn{a}
  \fn{toEditor} : \bn{a} -> \ty{Elab} \ty{SExp}
\end{Verbatim}
\caption{Definition of the \Editorable{} type class.}
\label{code:editorable}
\end{figure}

Whenever users want to inform the compiler about the \sexp{}
representation of values of a type, they have to define an instance of the
\Editorable{} type class. Later when a user runs an editor action from an
editor, the \Editorable{} instances are used for communication via
\sexp{}s.

\subsubsection{Some \Editorable{} Instances}

The collection of atoms in Idris's \sexp{}s already includes many
primitive types, such as \ty{String}.  Deserializing a string succeeds
when provided with a string, and fails otherwise. The message thrown on failure can be a non-trivial list structure, which allows Idris's pretty printer to be used to render substrings, but here we elide the concrete messages and focus on the successful cases.
Serialization tags
the atom appropriately.

\begin{figure}[H]
\begin{Verbatim}
\kw{implementation} \ty{Editorable} \ty{String} \kw{where}
  \fn{fromEditor} (\dt{StringAtom} \bn{s}) = \fn{pure} \bn{s}
  \fn{fromEditor} \bn{x} = \fn{fail} \dt{[}\cm{\{- elided -\}}\dt{]}
  \fn{toEditor} \bn{x} = \fn{pure} (\dt{StringAtom} \bn{x})
\end{Verbatim}
\caption{\ty{String} instance of the \ty{Editorable} type class.}
\label{code:editorableString}
\end{figure}

Inductive types, such as \mbox{\ty{Maybe} \bn{a}}, can be represented as lists in which the first element is a tag specifying the chosen constructor.
For instance, \mt{\dt{Just "abc"}} can be represented as \mt{(\dt{:Just "abc"})}, a list
\sexp{} with a symbol atom as the first element and then the \sexp{}
representation of a string, and \dt{Nothing} can be represented as
the symbol \dt{:Nothing}. This can be implemented as follows:

\begin{figure}[H]
\begin{Verbatim}
\kw{implementation} \ty{Editorable} \bn{a}
            => \ty{Editorable} (\ty{Maybe} \bn{a}) \kw{where}
  \fn{fromEditor} (\dt{SExpList [SymbolAtom "Nothing"]}) =
    \fn{pure} \dt{Nothing}
  \fn{fromEditor} (\dt{SExpList [SymbolAtom "Just"}, \bn{x}\dt{]}) =
    \kw{do} \bn{x}' <- \fn{fromEditor} \bn{x}
       \fn{pure} (\dt{Just} \bn{x}')
  \fn{fromEditor} \bn{x} = \fn{fail} \dt{[}\cm{\{- elided -\}}\dt{]}
  \fn{toEditor} (\dt{Just} \bn{x}) =
    \kw{do} \bn{x'} <- \fn{toEditor} \bn{x}
       \fn{pure} (\dt{SExpList} \dt{[}\dt{SymbolAtom} \dt{"Just"}, \bn{x'}\dt{]})
  \fn{toEditor} \dt{Nothing} =
    \fn{pure} (\dt{SExpList [SymbolAtom "Nothing"]})
\end{Verbatim}
\label{code:editorableMaybe}
\caption{\ty{Maybe} instance of the \ty{Editorable} type class.}
\end{figure}

The idea that is introduced here can be used to define an \Editorable{}
instance for any data type that has exported constructors. Constructors that do
not take any argument are represented as symbol atoms, and the ones that do
take arguments are represented as a list \sexp{}, in which the first element is
a symbol atom and the other elements represent the arguments that the
constructor takes. We will call this the \emph{constructor-based \sexp{}
representation} of a type.

It is not, however, possible to use the constructor-based representation for every type.
In particular, functions and infinite coinductive datatypes do not, in general, admit finite serializations.

In other cases, the constructor-based representation requires too much work to encode and decode in editors.
For instance, Idris names have a rich structure, but users know them by their syntax rather than by their internal representation.
The \Editorable{} instance for the type of quoted Idris
names, \ty{TTName}, which appeared in \autoref{fig:motivating-example}, represents names using their user-facing syntax.
For instance, the Idris name \fn{Prelude.Bool.not}, which has the
data type representation \mt{\dt{NS "not" ["Bool"}, \dt{"Prelude"]}}, is
represented by a string atom \sexp{}, namely \dt{"Prelude.Bool.not"}.

\subsubsection{Primitive \Editorable{} Instances}
\label{sssec:primitiveEditorable}

The \sexp{} representations of quoted Idris code, such as \TT{}  and
\TyDecl{}, are the most challenging ones. These types mirror the internal
representation of Idris's core language, but they are ordinary inductive data
types defined in Idris, which means that the constructor-based representation
suffices to represent them.

However, that representation is not particularly convenient for
extending editors.  The constructor-based representation would be an
abstract syntax tree of the \textsf{TT} representation of an Idris
expression. Users, however, work with the concrete syntax of Idris
itself. When they use editor actions, they expect to see concrete
syntax put back into the file, and converting from \TT{} to concrete
Idris syntax requires a lot of code that should not be duplicated in
each editor when it already exists in the Idris compiler. Therefore,
for these core datatypes, the editor sends and receive concrete
syntax.

On the other hand, if the compiler receives concrete syntax and needs to run \Elab{}
actions on that, there are many missing steps in between, most important of
which is elaboration from a high-level language to the core language.
Similarly, if the compiler needs to send back concrete syntax after
running \Elab{} actions, then it needs to reverse all those steps.
In other words, there is a colossal gap between concrete syntax and the core
language that needs to be bridged, and this task can be delegated to the
\Editorable{} type class.

When the \sexp{} received by the compiler is a string atom that is a
piece of Idris code, i.e.\ concrete syntax, \fn{fromEditor} should parse that
string into a high-level language term, and then elaborate that into a core
language term. Only after that can the compiler run the \Elab\ editor action.
Similarly, when the \Elab\ action finishes, \fn{toEditor} should convert core
language terms into the high-level language terms, a process called
\emph{delaboration} in the Idris compiler.
Then, the compiler should invoke the pretty printer to get concrete syntax that
represents that term. The resulting string can be sent back from the compiler
to the editor as a string atom \sexp{}.

Bridging this gap requires an \Editorable{} instance for \TT{} that does parsing,
elaboration, conversion from the core language to the surface language, and pretty printing.
Rather than reimplementing this from scratch in Idris itself, we extended \Elab{}
to expose these features of the compiler as primitives, following \citeauthor{barzilayphd}'s program of \emph{direct reflection}~\citep{barzilayphd}.
In particular, these primitives are used to define the instances of \Editorable{}
for the core language types like \TT{}, \TyDecl{} and \FunDefn{}.
Hard-coding the \Editorable{} instances of \ty{TT}, \ty{TyDecl},
\ty{DataDefn}, \ty{FunDefn}, and \ty{FunClause} into the compiler
allows  by making use
of the already existing compiler implementations of the steps listed above.

To achieve this, the existing \Elab{} monad needs to be extended with
primitives that go through the steps mentioned above.  One \Elab{} primitive
for \fn{fromEditor} and one for \fn{toEditor} suffice; polymorphic
primitives constrained by an indexed family provide a principled
way to manage the primitive instances of \Editorable{}.

\begin{figure}[H]
\begin{Verbatim}
\kw{data} \ty{HasPrim} : \ty{Type} -> \ty{Type} \kw{where}
  \dt{HasTT}        : \ty{HasPrim} \ty{TT}
  \dt{HasTyDecl}    : \ty{HasPrim} \ty{TyDecl}
  \dt{HasDataDefn}  : \ty{HasPrim} \ty{DataDefn}
  \dt{HasFunDefn}   : \ty{HasPrim} (\ty{FunDefn} \ty{TT})
  \dt{HasFunClause} : \ty{HasPrim} (\ty{FunClause} \ty{TT})
\end{Verbatim}
\caption{Definition of the \ty{HasPrim} predicate in Idris.}
\end{figure}


\begin{figure}
\begin{Verbatim}
\fn{prim__fromEditor} : \ty{HasPrim} \bn{a} -> \ty{SExp} -> \ty{Elab} \bn{a}
\fn{prim__toEditor} : \ty{HasPrim} \bn{a} -> \bn{a} -> \ty{Elab} \ty{SExp}
\end{Verbatim}
  \caption{The new \Elab\ primitives.}
\label{code:newElabPrims}
\end{figure}

Figure~\ref{code:newElabPrims} uses \ty{HasPrim} to describe the new \Elab{} primitives.
Using these two primitives, the \Editorable{} instances for the core language
types all look alike; an example can be seen in figure~\ref{fig:editorablePrim}.

\begin{figure}
\begin{Verbatim}
\kw{implementation} \ty{Editorable} \ty{TT} \kw{where}
  \fn{fromEditor} \bn{x} = \fn{prim__fromEditor} \dt{HasTT} \bn{x}
  \fn{toEditor} \bn{x} = \fn{prim__toEditor} \dt{HasTT} \bn{x}
\end{Verbatim}
\caption{An \Editorable{} instance depending on the new primitives.}
\label{fig:editorablePrim}
\end{figure}

\subsection{How the Compiler Uses \Editorable{} for Communication}
\label{ssec:communication}

We have extended Idris's IDE protocol to support an additional message that
represents an invocation of a custom editor action.
This message includes the name of the custom action and a list of arguments,
and its reply contains Idris's response.

When the editor fires up Idris in IDE mode and loads the file, then it can send
a custom action message to Idris. If the compiler receives such a message from
the editor, it looks up the name and type of the editor action from the
context. From types of the arguments of the \Elab{} action, it can find the
necessary \Editorable{} instances and use the \fn{fromEditor} definitions in them
to parse the \sexp{}s into Idris values. If the number of arguments in the
action type and the argument list match, and all arguments can be parsed
without any errors, then the compiler can run the \Elab{} action, and use
\fn{toEditor} to serialize the output, and send it back to the editor.

The compiler can use \Elab{} actions whose arguments and return type have
\Editorable{} instances as custom editor actions. The \fn{easy} action from
figure \ref{fig:motivating-example} is an example of this, and its usage can be
seen in figure \ref{fig:motivating-example-exec}.
When the user puts the cursor on \hole{ex1\_impl} and invokes
\fn{idris-easy} in their Emacs session, Emacs sends a message to the compiler
that specifies that it wants to run \fn{easy}, and provides a list of
arguments, \mt{(\fn{list} \dt{"ex1\_impl"})}, which is a singleton list
containing a string atom. When Idris receives this message, it looks up the
name \fn{easy} from the context and finds out that it has the type
\mt{\ty{TTName} -> \ty{Elab TT}}. Therefore it looks up the \Editorable{}
instance of \ty{TTName} and runs its \fn{fromEditor} implementation on \dt{"ex1\_impl"}, which
results in the Idris name for \hole{ex1\_impl}. Then the compiler can execute
\fn{easy} and get a core language term \dt{()} as a result. Since core
language terms are represented by the \TT{} type, Idris has to find the
\Editorable{} instance of \TT{} and run its implementation of \fn{toEditor} on that term,
which produces an \sexp{} to be sent back to the editor.

\subsection{Using \Editorable{} in Type-Checking}
\label{ssec:typechecking}

The motivation behind the \ty{Editorable} type class is twofold:
\begin{enumerate}
\item to use the \fn{fromEditor} and \fn{toEditor} definitions to serialize
  and deserialize data before and after an \Elab{} action is run; and
\item to check whether a given \Elab{} action is suitable to be used as an
  editor action.
\end{enumerate}

The first motivation is already covered in the previous sections.
When Idris encounters a definition that is tagged with the \kw{\%editor}
keyword as an editor action, it first elaborates the type.
The next step is to check whether this type is suitable as an editor action.
It does this by ensuring that each argument type has an \Editorable{} instance,
that the return type has \Elab{} at its head, and that the argument to \Elab{}
is also \Editorable{}.
Note that this rules out dependent types for editor
actions---section~\ref{sssec:universeEncoding} discusses a potential way to lift
this restriction in the future.

%%% Local Variables:
%%% mode: latex
%%% TeX-master: "source"
%%% End:
